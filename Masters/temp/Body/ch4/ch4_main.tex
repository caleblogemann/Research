% !TEX root = ../../thesis.tex

\chapter{Results}\label{chp:results}
\section{Manufactured Solution}\label{ssec:manufactured_solution}
In order to confirm the order of accuracy of our method, we use the method of
manufactured solutions.
The method of manufactured solutions picks an exact solution \(\hat{q}\) and then adds
a source term to the initial partial differential equation to make \(\hat{q}\) the
true solution to the new differential equation.
So we actually solve
\begin{equation}
    q_t + \p{q^2 - q^3}_x = -\p{q^3 q_{xxx}}_x + s
\end{equation}
where
\begin{equation}
    s = \hat{q}_t + \p{\hat{q}^2 - \hat{q}^3}_x + \p{\hat{q}^3 \hat{q}_{xxx}}_x.\label{eq:source_term}
\end{equation}
This new PDE's exact solution is now \(\hat{q}\).
We chose
\begin{equation}
    \hat{q} = 0.1 \times \sin{2 \pi / 20.0 \times (x - t)} + 0.15\label{eq:exact_solution}
\end{equation}
on \(x \in \br{0, 40}\) with periodic boundary conditions to be our manufactured
solution.
This manufactued solution is chosen so that our flux \(q^2 - q^3\) is in it's convex
region.

We solve this problem until \(t = 5.0\) with constant, linear, and quadratic basis
polynomials.
The time step size for these simulations is determined by the
Courant–Friedrichs–Lewy (CFL) condition.
The CFL condition states that
\begin{equation}
  \Delta t = \nu \frac{\Delta x}{\lambda}
\end{equation}
where \(\lambda \) is the wavespeed and \(\nu \) is the CFL number.
For hyperbolic problems solved with explicit Runge Kutta time-stepping, \(\nu \),
typically follows the pattern \(\frac{1}{2n - 1} \) where \(n\) is the order of the
method.
For this example the wavespeed is \(1\), and the timestep is chosen just smaller
then the typical CFL number.
Specifically the timesteps were \(\Delta t = 0.9 \Delta x\),
\(\Delta t = 0.2 \Delta x\), and \(\Delta t = 0.1 \Delta x\)
for first, second and third order respectively.

Table~\ref{tab:convergence_results} shows the error for each simulation as the mesh
is refined.
The table also shows the rate of convergence to the true solution, and in each case
the rate of convergence approaches the expected order.
Note that the first, second, and third order methods only require one, two, and
three Picard iterations respectively.
The error is computed in the \(L^2\) sense.
First if the numerical solution is in the space \(V_h^k\), then the exact solution
is projected onto the space \(V_h^{k+1}\).
Let \(\hat{q}_h\) be this projection, and we will also project the numerical
solution, \(q_h\), onto the space \(V_h^{k+1}\) with the same basis.
Let \(\set{\phi_i}_{i=1}^{N(k+1)}\) be that basis for \(V_h^{k+1}\), then
there exists coefficients, \(\hat{Q}_i\) and \(Q_i\), such that
\begin{equation}
  \hat{q}_h = \sum{i = 1}{N(k+1)}{\hat{Q}_i \phi_i(x)} \text{ and }
  q_h = \sum{i = 1}{N(k+1)}{Q_i \phi_i(x)}.
\end{equation}
The error is then computed as
\begin{equation}
  e_N = \sqrt{\frac{\sum*{i = 1}{N(k+1)}{\p{\hat{Q}_i - Q_i}^2}}{\sum{i=1}{N(k+1)}{\hat{Q}_i^2}}}.
\end{equation}
This is equivalent to leading order to
\begin{equation}
  e = \sqrt{\dintt{a}{b}{\p{\hat{q} - q_h}^2}{x}}.
\end{equation}
The order of convergence is then computed from these errors as
\begin{equation}
  \text{order} = \log[2]{\frac{e_N}{e_{2N}}}.
\end{equation}

\begin{table}
  \centering
  \begin{tabular}{r*{6}l}
    \toprule
    & \multicolumn{2}{c}{1st Order} & \multicolumn{2}{c}{2nd Order} & \multicolumn{2}{c}{3rd Order} \\
    \midrule
    \(n\) & \multicolumn{1}{c}{error} & order & \multicolumn{1}{c}{error} & order & \multicolumn{1}{c}{error} & order\\
    \midrule
      20 &   \(0.136\) &  --- & \(7.34 \times 10^{-3}\) &  --- & \(5.29 \times 10^{-4}\) &  --- \\
      40 &  \(0.0719\) & 0.91 & \(1.99 \times 10^{-3}\) & 1.89 & \(5.38 \times 10^{-5}\) & 3.30 \\
      80 &  \(0.0378\) & 0.93 & \(5.60 \times 10^{-4}\) & 1.83 & \(7.47 \times 10^{-6}\) & 2.85 \\
     160 &  \(0.0191\) & 0.99 & \(1.56 \times 10^{-4}\) & 1.85 & \(9.97 \times 10^{-7}\) & 2.91 \\
     320 & \(0.00961\) & 0.99 & \(3.98 \times 10^{-5}\) & 1.97 & \(1.26 \times 10^{-7}\) & 2.98 \\
     640 & \(0.00483\) & 0.99 & \(1.00 \times 10^{-5}\) & 1.99 & \(1.58 \times 10^{-8}\) & 3.00 \\
    1280 & \(0.00242\) & 1.00 & \(2.50 \times 10^{-6}\) & 2.00 & \(1.98 \times 10^{-9}\) & 3.00 \\
    \bottomrule
  \end{tabular}
  \caption{Convergence table with a constant, linear, quadratic polynomial bases.
  Nonlinear solve uses one, two, or three Picard iterations respectively.
   CFL = 0.9, 0.2, 0.1 respectively}\label{tab:convergence_results}
\end{table}

Table~\ref{tab:iteration_results} shows how the error is affected when the number of
Picard iterations is decreased.

\begin{table}
  \centering
  \begin{tabular}{r*{6}l}
    \toprule
         & \multicolumn{2}{c}{1 Iteration} & \multicolumn{2}{c}{2 Iterations} & \multicolumn{2}{c}{3 Iterations} \\
    \midrule
    \(n\)& \multicolumn{1}{c}{error} & order & \multicolumn{1}{c}{error} & order & \multicolumn{1}{c}{error} & order\\
    \midrule
      20 & & & & & \(5.29 \times 10^{-4}\) & --- \\
      40 & & & & & \(5.38 \times 10^{-5}\) & 3.30 \\
      80 & & & & & \(7.47 \times 10^{-6}\) & 2.85 \\
     160 & & & & & \(9.97 \times 10^{-7}\) & 2.91 \\
     320 & & & & & \(1.26 \times 10^{-7}\) & 2.98 \\
     640 & & & & & \(1.58 \times 10^{-8}\) & 3.00 \\
    \bottomrule
  \end{tabular}
  \caption{Convergence table with a quadratic polynomial basis. One, Two, and Three
  Picard iterations are used for each nonlinear solve. CFL = 0.1}\label{tab:iteration_results}
\end{table}

\section{Traveling Waves}\label{ssec:bertozzi_cases}
In this section we showcase several numerical examples that demonstrate the traveling
wave profiles of equation~\eqref{eq:thin_film_model}.
The traveling wave profiles differ from the standard hyperbolic wave profile in
several ways.
They differ in that they may not be unique and may include undercompressive shocks.
These examples were first shown in~\cite{article:Bertozzi1999} with first order accuracy.

In these examples, we use a moving reference frame to keep the mesh size reasonable.
This is done by actually simulating the following equation,
\begin{equation}
q_t + \p{q^2 - q^3 - sq}_x = -\p{q^3 q_{xxx}}_x \label{eq:thin_film_moving_reference_frame}
\end{equation}
where \(s\) is the Rankine-Hugoniot wavespeed of the original numerical flux, \(f\)
\begin{equation}
s = \frac{f(q_l) - f(q_r)}{q_l - q_r} = q_l + q_r - (q_l^2 + q_l q_r + q_r^2).\label{eq:ranking_hugoniot}
\end{equation}
This modified equation will have zero wavespeed in most cases and simulates the
original PDE in a moving reference frame.

These examples consider Riemann Problems with different left and right states.
Depending on the left and right states they are several different wave profiles.
For these examples we will fix the right state and vary the left state.
The wave profiles would be qualitatively equivalent for different values for the right
state, however the values of the left state would also need to change accordingly.

\subsubsection{Case 1: Unique Weak Lax Shock}\label{sssec:case1}
Consider a Riemann Problem with left state, \(q_l = 0.3\), and right state,
\(q_r = 0.1\).
We will use the following smoothed out profile as an initial condition for this
problem
\begin{equation}
  q_0(x) = \p{\tanh{-x} + 1} \frac{q_l - q_r}{2} + q_r
\end{equation}
For this initial condition the numerical solution approaches a steady wave profile
with a unique Lax type shock.
Figure~\ref{fig:case1} shows a plot of the solution with this initial condition after
enough time for the solution to hit its steady state.
Note that this behavior persists for all \(q_l\) up to some bound which depends on
\(q_r\).
\begin{figure}
  \centering
  \includegraphics[scale=0.5]{figures/case_1_1.pdf}
  \caption{Case 1: Unique Lax Shock}\label{fig:case1}
\end{figure}

\subsubsection{Case 2: Multiple Lax Shocks}\label{sssec:case2}
If the left state is increased to \(q_l = 0.3323\), then the wave profile is no
longer unique.
In this case the steady traveling wave depends on the initial conditions.

For the following initial conditions,
\begin{equation}
  q_0(x) = \p{\tanh{-x} + 1} \frac{q_l - q_r}{2} + q_r\label{eq:case2_1}
\end{equation}
the behavior is the same as in case 1.
The numerical solution for this initial condition is shown in Figure~\ref{fig:case2}

Now consider the following initial conditions,
\begin{equation}
  q_0(x) =
  \begin{cases}
    \p{\p{0.6 - q_l}/2} \tanh{x} + \p{\p{0.6 + q_l}/2} & x < 5 \\
    -\p{\p{0.6 - q_r}/2} \tanh{x - 10} + \p{\p{0.6 + q_r}/2} & x > 5
  \end{cases}.\label{eq:case2_2}
\end{equation}
The reader might expect this initial condition to approach the traveling wave shown
earlier as it has the same far field boundary values, however this is not the case.
The traveling wave profile that is approaches is shown in Figure~\ref{fig:case2}.

\begin{figure}
  \centering
  \includegraphics[scale=0.35]{figures/case_2_1.pdf}
  \includegraphics[scale=0.35]{figures/case_2_2.pdf}
  \caption{Case 2: Multiple Lax Shocks}\label{fig:case2}
\end{figure}

These are not the only possible wave profiles possible with these left and right
states.
Figure~\ref{fig:case2_3} shows the result for the following initial condition,
\begin{equation}
  q_0(x) =
  \begin{cases}
    \p{\p{0.6 - q_l}/2} \tanh{x} + \p{\p{0.6 + q_l}/2} & x < 10 \\
    -\p{\p{0.6 - q_r}/2} \tanh{x - 20} + \p{\p{0.6 + q_r}/2} & x > 10
  \end{cases}\label{eq:case2_3}.
\end{equation}
This initial condition has a larger hump then equation~\eqref{eq:case2_2}, and the
traveling wave reflects this aspect.
In this case the traveling wave is not a steady wave, and there are in fact
two shocks.
The right shock is an undercompressive shock and the left shock is a traditional
compressive shock.
Both shocks travel slower than the moving reference frame.
They also travel at different speeds from one another so the wave profile changes over
time.
\begin{figure}
  \centering
  \includegraphics[scale=0.5]{figures/case_2_3.pdf}
  \caption{Case 2: Multiple Lax Shocks}\label{fig:case2_3}
\end{figure}

\subsubsection{Case 3: Undercompressive Double Shock}\label{sssec:case3}
For the left state in the next regime, there is again a unique shock profile for all
initial conditions.
However this shock profile is not the single Lax shock seen in Case 1.
With the following initial conditions,
\begin{equation}
  q_0(x) = \p{\tanh{-x} + 1} \frac{q_l - q_r}{2} + q_r
\end{equation}
where \(q_l = 0.4\) and \(q_r = 0.1\), figure~\ref{fig:case3} shows a double shock
structure.
Similar to figure~\ref{fig:case2_3}, we see a undercompressive shock on the right
and a Lax shock on the left.
\begin{figure}
  \centering
  \includegraphics[scale=0.5]{figures/case_3_1.pdf}
  \caption{Case 3: Undercompressive Double Shock}\label{fig:case3}
\end{figure}

\subsubsection{Case 4: Rarefaction-Undercompressive Shock}\label{sssec:case4}
The final traveling wave structure appears when the left state is greater than the
undercompressive shock height.
In this case we see a rarefaction wave along with the undercompressive shock.
Figure~\ref{fig:case4} shows the numerical solution for initial condition
\begin{equation}
  q_0(x) = \p{\tanh{-x + 110} + 1} \frac{q_l - q_r}{2} + q_r
\end{equation}
where \(q_l = 0.8\) and \(q_r = 0.1\).
Note that the rarefaction wave and undercompressive shock are traveling at different
speeds so they seperate from each other.
\begin{figure}
  \centering
  \includegraphics[scale=0.5]{figures/case_4_1.pdf}
  \caption{Case 4: Rarefaction-undercompressive shock}\label{fig:case4}
\end{figure}

% \bibliographystyle{plain}
% % \vspace{-20pt}
% \begingroup
%     \setlength{\bibsep}{13.2pt}
%     \linespread{1}\selectfont
%     \bibliography{refs}
% \endgroup
\clearpage
\pagebreak