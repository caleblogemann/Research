% Chapter 1 of the Thesis Template File
\chapter{Introduction}

In this paper we look at the model equation,
\begin{equation}
  q_t + \p{q^2 - q^3}_x = -\p{q^3 q_{xxx}}_x \quad (x, t) \in \br{a, b} \times \br{0, T}. \label{eq:thin_film_model}
\end{equation}
This equation describes the motion of a thin film of liquid flowing over a one-dimensional domain,
where \(q(x, t) \ge 0\) is the height of the liquid.
This fluid is acted upon by gravity, by forces on the surface, and by surface tension.
An equivalent model can be derived using thermocapillary forces and molecular forces.
This model is useful in many different applications including airplane de-icing\cite{}
and industrial coating.
Some experimental study~\cite{article:cazabat1990fingering,
article:kataoka1997theoretical, article:ludviksson1971dynamics} has been done and
numerical results have shown good agreement with those experiments
in~\cite{article:bertozzi1998contact}.

Previous numerical methods for this type of equation have focused on finite difference
approaches.
Bertozzi and Brenner\cite{bertozzi1997linear} used a fully implicit centered finite
difference scheme to explore instabilities.
Ha et al.\cite{article:Ha2008} explored several different finite difference schemes,
some fully implicit and some using the Crank-Nicolson method.
In their analysis the considered several different methods for the hyperbolic terms
including WENO, Godunov, and an adapted Upwind method.
All of these methods were limited to just first or second order, and they required
solving a Newton iteration.
Finite difference methods also lack provable stability.
% other drawbacks to finite difference methods

We chose to use discontinuous Galerkin methods as they allow for high order
convergence.
The discontinuous Galerkin methods were first introduced by Reed and
Hill\cite{techreport:Reed1973}, and then were formalized by Cockburn and Shu
in a series of papers\cite{article:Cockburn1991I, article:Cockburn1989II,
article:Cockburn1989III, article:Cockburn1990IV, article:cockburn1998V}.
We use the original modal discontinuous Galerkin method as well as the local
discontinuous Galerkin method.
The local discontinuous Galerkin method was also formulated by Cockburn and
Shu\cite{article:Cockburn1998LDG} to handle higher order derivatives with the
discontinuous Galerkin method.
We use the modal discontinuous Galerkin method to discretize the convection term and
the local discontinuous Galerkin method to discretize the diffusion term.

The diffusion term is much stiffer of a problem then the hyperbolic convection.
If the diffusion is handled explicitly in time, then a very strict time step
restriction is present for stability.
Therefore implicit schemes should be preferred.
However the hyperbolic term is nonlinear, and so an implicit scheme would require a
Newton iteration.
Thus we chose to use an Implicit-Explicit Runge Kutta scheme for propagating in time.
These schemes were first introduced by Ascher et al.\cite{article:ascher1997implicit}
and have been expanded on by Kennedy and Carpenter\cite{kennedy2003additive} and
Pareschi and Russo\cite{article:pareschi2000IMEX}.
\clearpage
\pagebreak
