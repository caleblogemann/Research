% !TEX root = main.tex

\section{Numerical Methods}\label{sec:numerical}

  \subsection{Implicit-Explicit Runge-Kutta Scheme}\label{ssec:imex}
    The thin-film model we are trying to solve is a very stiff equation due to the
    high order derivatives.
    In order to take reasonable timesteps despite this stiffness we use
    implicit-explicit (IMEX) Runge-Kutta schemes to propagate our solution through time.
    The IMEX Runge-Kutta scheme propagates time for ordinary differential
    equations or systems of equations of the form
    \begin{equation}
      q_t = F(t, q) + G(t, q),
    \end{equation}
    where \(F\) is solely handled explicitly, but \(G\) needs to be solved implicitly.
    A single step of the Runge-Kutta IMEX scheme is computed as follows,
    \begin{align}
      q^{n+1} &= q^n + \Delta t \sum{i = 1}{s}{b_i' F(t_i, u_i)} + \Delta t \sum{i=1}{s}{b_i G(t_i, u_i)} \\
      u_i &= q^n + \Delta t \sum{j = 1}{i-1}{a_{ij}' F(t_j, u_j)} + \Delta t \sum{j=1}{i}{a_{ij} G(t_j, u_j)} \\
      t_i &= t^n + c_i \Delta t
    \end{align}
    where the coefficients \(a_{ij}\), \(b_i\), \(c_i\), \(a_{ij}'\), \(b_i'\), and
    \(c_i'\) are set in the double Butcher tableau
    \begin{center}
      \begin{tabular}{r|l}
        \(c'\) & \(a'\) \\
        \midrule
          & \(\p{b'}^T\)
      \end{tabular}\hspace{0.5cm}
      \begin{tabular}{r|l}
        \(c\) & \(a\) \\
        \midrule
          & \(b^T\)
      \end{tabular}.
    \end{center}

    Specifically we used the IMEX schemes first introduced by Pareschi and
    Russo~\cite{article:pareschi2000IMEX, article:pareschi2005IMEX}.
    These schemes are strong stability preserving (SSP) in the sense of Gottlieb et
    al.~\cite{gottlieb2001strong} and they are designed for stiff systems.
    The double Butcher tableaus are shown below for the one stage, L-stable, 1st order
    method:
    \begin{center}
      \begin{tabular}{r|l}
        0 & 0 \\
        \midrule
          & 1
      \end{tabular}\hspace{0.5cm}
      \begin{tabular}{r|l}
        1 & 1 \\
        \midrule
          & 1
      \end{tabular},
    \end{center}
    the three stage, 2nd order method:
    \begin{center}
      \begin{tabular}{r|lll}
        0 & 0 & 0 & 0 \\
        0 & 0 & 0 & 0 \\
        1 & 0 & 1 & 0 \\
        \midrule
          & 0 & \(\frac{1}{2}\) & \(\frac{1}{2}\) \\
      \end{tabular}\hspace{0.5cm}
      \begin{tabular}{r|lll}
        \(\frac{1}{2}\) & \(\frac{1}{2}\) & 0 & 0 \\
        0 & \(-\frac{1}{2}\) & \(\frac{1}{2}\) & 0 \\
        1 & 0 & \(\frac{1}{2}\) & \(\frac{1}{2}\) \\
        \midrule
          & 0 & \(\frac{1}{2}\) & \(\frac{1}{2}\) \\
      \end{tabular},
    \end{center}
    and the four stage, L-stable, third order method:
    \begin{center}
      \begin{tabular}{r|llll}
        0 & 0 & 0 & 0 & 0 \\
        0 & 0 & 0 & 0 & 0 \\
        1 & 0 & 1 & 0 & 0 \\
        \(\frac{1}{2}\) & 0 & \(\frac{1}{4}\) & \(\frac{1}{4}\) & 0 \\
        \midrule
          & 0 & \(\frac{1}{6}\) & \(\frac{1}{6}\) & \(\frac{2}{3}\) \\
      \end{tabular} \hspace{0.5cm}
      \begin{tabular}{r|llll}
        \(\alpha \) & \(\alpha \) & 0 & 0 & 0 \\
        0 & \(-\alpha \) & \(\alpha \) & 0 & 0 \\
        1 & \(0\) & \(1 - \alpha \) & \(\alpha \) & 0 \\
        \(\frac{1}{2}\) & \(\beta \) & \(\eta \) & \(\zeta \) & \(\alpha \) \\
        \midrule
          & 0 & \(\frac{1}{6}\) & \(\frac{1}{6}\) & \(\frac{2}{3}\) \\
      \end{tabular}, \\
    \end{center}
    where
    \begin{center}
      \begin{align*}
        \alpha &= 0.24169426078821, \\
        \beta &= 0.06042356519705, \\
        \eta &= 0.1291528696059, \\
        \zeta &= \frac{1}{2} - \beta - \eta - \alpha.
      \end{align*}
    \end{center}

    For our thin-film model \(F\) and \(G\) will be the spatial discretizations of
    the hyperbolic convection and parabolic terms respectively, that is
    \begin{align}
      F(t, q) &= -\p{q^2 - q^3}_x \\
      G(t, q) &= -\p{q^3 q_{xxx}}_x.
    \end{align}
    Since \(G\) is the stiffest part of our model, we would like to handle this term
    implicitly so that our time step is less restricted.
    Also handling \(F\) explicitly allows us to capture the nonlinear
    behavior without a nonlinear solve.

  \subsection{Space Discretization}
    We chose to use the discontinuous Galerkin Method to discretize our equation in
    space.
    First let \(\set{x_{j+1/2}}_0^N\) partition the domain, \(\br{a, b}\), and denote
    each interval as \(I_j = \br{x_{j-1/2}, x_{j+1/2}}\) with
    \(\Delta x_j = x_{j+1/2} - x_{j-1/2}\).
    The discontinuous Galerkin solution then exists in the following finite dimensional
    space,
    \begin{equation}
      V_h^k = \set{v \in L^1\p{\br{a, b}}: \eval{v}{I_j} \in P^k(I_j), j = 1, \ldots, N},
    \end{equation}
    where \(P^k(I_j)\) denotes the set of polynomials of degree k or less on \(I_j\).

  \subsubsection{Modal Discontinuous Galerkin for Hyperbolic Convection}\label{ssec:dg}
    First we consider the continuous operator \(F(t, q) = -\p{q^2 - q^3}_x\).
    Note that \(F: \br{0, T} \times C^1(a, b) \to C^0(a, b)\), and we would like to
    form an approximation, \(F_h\) to this continuous operator, such that
    \(F_h: \br{0, T} \times V_h^k \to V_h^k\).
    In order to do this, consider the weak formulation of the continuous operator,
    \(F\).
    The weak form requires finding \(F(t, q) \in C^0(a, b)\) such that
    \begin{equation}
      \dintt{a}{b}{F(t, q) v(x)}{x} = -\dintt{a}{b}{\p{q^2 - q^3}_x v(x)}{x}
    \end{equation}
    for all smooth functions, \(v \in C^{\infty}(a, b)\).
    The discontinuous Galerkin approximation is formed by replacing these function
    spaces with the DG space, \(V_h^k\).
    This gives that the weak formulation of the approximation \(F_h\) is to find
    \(F_h(t, q_h) \in V_h^k\) such that
    \begin{equation}
      \dintt{a}{b}{F_h(t, q_h) v_h(x)}{x} = -\dintt{a}{b}{\p{q_h^2 - q_h^3}_x v_h(x)}{x}
    \end{equation}
    for all \(v_h \in V_h^k\).
    Using integration by parts this is equivalent to
    finding \(F_h(t, q_h) \in V_h^k\) such that
    \begin{align}
      \dintt{I_j}{}{F_h(t, q_h)v_h(x)}{x} &= \dintt{I_j}{}{\p{q_h^2 - q_h^3} v_h'(x)}{x} \\
      &- \hat{f}\p{q_h}_{j+1/2} v_h(x_{j+1/2}) + \hat{f}\p{q_h}_{j-1/2} v_h(x_{j-1/2}) \nonumber
    \end{align}
    for \(j = 1, \ldots, N\) and for all \(v_h \in V_h^k\).
    Since \(q_h\) is discontinuous at the cell interfaces \(x_{j\pm 1/2}\), some
    numerical flux \(\hat{f}\) must be chosen.

    We chose to use the Local Lax-Friedrich's numerical flux, that is
    \begin{equation}
      \hat{f}(q_l, q_r) = \frac{1}{2} \p{f(q_l) + f(q_r) - \lambda_{\text{max}} \p{q_r - q_l}}
    \end{equation}
    where \(f\) is the flux function, \(q_l\) and \(q_r\) and the left and right
    states at the interface, and \(\lambda_{\text{max}}\) is the locally maximum wavespeed,
    that is
    \begin{equation}
      \lambda_{\text{max}} = \max[q \in \br{\nu, \mu}]{f'(q)},
    \end{equation}
    where \(\nu = \min{q_l, q_r}\) and \(\mu = \max{q_l, q_r}\).
    In our case \(f(q) = q^2 - q^3\).
    At the interface \(x_{j+1/2}\) we have \(q_l = q_h^-(x_{j+1/2})\) and
    \(q_r = q_h^+(x_{j+1/2})\).

    % \begin{equation}
    %   \hat{f}\p{q_h}_{j+1/2} = \frac{1}{2} \p{\p{\p{\p{q_h^+(x_{j+1/2})}^2 - \p{q_h^+(x_{j+1/2})}^3} + \p{\p{q_h^-(x_{j+1/2})}^2 - \p{q_h^-(x_{j+1/2})}^3}}
    %   - \lambda_{\max} \p{q_h^+(x_{j+1/2}) - q_h^-(x_{j+1/2})}}
    % \end{equation}
    % where
    % \begin{equation}
    %   \lambda_{\max} = \max[q \in \br{}]{2q - 3q^2}
    % \end{equation}

  \subsubsection{Local Discontinuous Galerkin Scheme for Parabolic Term}\label{ssec:ldg}
    Next we look at the continuous operator \(G(t, q) = -\p{q^3 q_{xxx}}_x\).
    In discretizing this operator we follow the local discontinuous Galerkin (LDG) method
    first introduced by Cockburn and Shu~\cite{article:Cockburn1998LDG} for
    convection-diffusion systems.
    We first introduce three auxilliary variables, \(r, s, u\), and rewrite the equation
    as the following system,
    \begin{align}
      r &= q_x, \\
      s &= r_x, \\
      u &= s_x, \\
      G(t, q) &= -\p{q^3 u}_x.
    \end{align}
    The weak form of this system is solved by finding functions \(r, s, u, G\) such that
    \begin{align}
      \dintt{a}{b}{r(x) w(x)}{x} &= \dintt{a}{b}{q_x(x) w(x)}{x}, \\
      \dintt{a}{b}{s(x) y(x)}{x} &= \dintt{a}{b}{r_x(x) y(x)}{x}, \\
      \dintt{a}{b}{u(x) z(x)}{x} &= \dintt{a}{b}{s_x(x) z(x)}{x}, \\
      \dintt{a}{b}{G(t, q) v(x)}{x} &= -\dintt{a}{b}{\p{q^3(x) u(x)}_x v(x)}{x},
    \end{align}
    for all smooth functions \(w, y, z, v \in C^{\infty}(a, b)\).
    The LDG method arrives from applying the standard DG method to each of these
    equations.
    That is we replace the continuous function spaces with the
    discontinuous finite dimensional DG space, \(V_h^k\).
    The approximate operator \(G_h\) becomes the process of finding
    \(r_h, s_h, u_h, G_h(t, q_h) \in V_h\) such that for all test functions
    \(v_h, w_h, y_h, z_h \in V_h\) the following equations are
    satisfied
    \begin{align}
      \dintt{a}{b}{r_h w_h}{x} &= \dintt{a}{b}{\p{q_h}_x w_h}{x}, \\
      \dintt{a}{b}{s_h y_h}{x} &= \dintt{a}{b}{\p{r_h}_x y_h}{x}, \\
      \dintt{a}{b}{u_h z_h}{x} &= \dintt{a}{b}{\p{s_h}_x z_h}{x}, \\
      \dintt{a}{b}{G_h(t, q_h) v_h}{x} &= -\dintt{a}{b}{\p{q_h^3 u_h}_x v_h}{x},
    \end{align}
    given \(t\) and \(q_h \in V_h^k\).
    This is equivalent to the following equations for all \(j\), if we use integration
    by parts,
    \begin{align}
      \dintt{I_j}{}{r_h w_h}{x} &= \p{\p{\hat{q}_h w^-_h}_{j+1/2}
      - \p{\hat{q}_h w^+_h}_{j-1/2}} - \dintt{I_j}{}{q_h \p{w_h}_x}{x}, \\
      \dintt{I_j}{}{s_h y_h}{x} &= \p{\p{\hat{r}_h y^-_h}_{j+1/2}
      - \p{\hat{r}_h y^+_h}_{j-1/2}} - \dintt{I_j}{}{r_h \p{y_h}_x}{x}, \\
      \dintt{I_j}{}{u_h z_h}{x} &= \p{\p{\hat{s}_h z^-_h}_{j+1/2}
      - \p{\hat{s}_h z^+_h}_{j-1/2}} - \dintt{I_j}{}{s_h \p{z_h}_x}{x}, \\
      \dintt{I_j}{}{G_h(t, q_h) v_h}{x} &= -\p{\p{\widehat{q^3 u}_h v^-_h}_{j+1/2}
      - \p{\widehat{q^3 u}_h v^+_h}_{j-1/2}} + \dintt{I_j}{}{q_h^3 u_h \p{v_h}_x}{x},
    \end{align}
    where \(\hat{q}, \hat{r}, \hat{s}, \widehat{q^3 u}\) are suitably chosen numerical fluxes.
    A common choice of numerical fluxes are the so-called alternating fluxes, shown below
    \begin{align}
      \hat{q}_h &= q^-_h, \\
      \hat{r}_h &= r^+_h, \\
      \hat{s}_h &= s^-_h, \\
      \widehat{q^3 u}_h &= \p{q^3 u}^+_h.
    \end{align}
    These numerical fluxes are one-sided fluxes that alternate sides for each
    derivative.
    They are chosen to make this method stable, and they also allow for the auxilliary
    variables to be locally solved in terms of \(q_h\), hence where the local
    discontinuous Galerkin method gets its name.

  \subsection{Nonlinear Solver}
    Using these discretizations in the Runge-Kutta IMEX scheme, requires solving the
    nonlinear system,
    \begin{equation}
      u_i - \Delta t a_{ii} G_h(t_i, u_i) = q^n
      + \Delta t \sum{j = 1}{i-1}{a_{ij}' F_h(t_j, u_j)}
      + \Delta t \sum{j=1}{i-1}{a_{ij} G_h(t_j, u_j)},
    \end{equation}
    for \(u_i\).
    The standard approach to solving this system would be to use a Newton iteration,
    however that would require the Jacobian of the operator \(I - \Delta t a_{ii} G_h\),
    which would be relatively intractable.
    Therefore we chose to linearize this operator and use a Picard iteration instead,
    which does not require the Jacobian.

    Suppose we are trying to solve the nonlinear equation \(L(u) = b\), where
    \(L'(v, u)\) is the nonlinear operator linearized about \(v\) acting on \(u\).
    The Picard iteration for solving this nonlinear equation starts with some initial
    guess \(u_0\).
    The next solution is found by solving the following linear problem,
    \begin{equation}
      L'(u_i, u_{i+1}) = b.
    \end{equation}
    In other words the next iteration is found by solving the problem linearized about
    the the previous iteration.

    We linearize the operator \(G_h(t, q_h)\) by first linearizing \(G(t, q)\).
    The continuous operator linearized about \(v\) is given by
    \begin{equation}
      G'(v, t, q) = - \p{v^3 q_{xxx}}_x.
    \end{equation}
    The linearized discrete operator is now just the LDG method applied to this
    linearized continuous operator.

    We find that the Picard iteration approach provides good results with relatively few
    iterations.
    In fact in Section~\ref{sec:results} we show that we can achieve first, second, and
    third order accuracy with only one iteration, with an initial guess that is the
    solution from the previous stage of the IMEX Runge-Kutta scheme.
    If we are solving the nonlinear system for the first stage, then the initial guess
    is the solution from the previous timestep instead of the previous stage.
    This approach is more tractable than a Newton iteration and it converges
    quickly to the nonlinear solution.

    These methods were implemented in a python code called
    \textsc{PyDoGPack}~\cite{pydogpack} developed by the authors.
    \textsc{PyDogPack} uses the packages \textsc{NumPy}~\cite{numpy} and
    \textsc{SciPy}~\cite{scipy} extensively.
    In particular they are used to solve linear systems of equations and to integrate
    numerically.
    All of the examples and tests shown in Section~\ref{sec:results} were done using
    this code.
    % Key parts of the code are shown in Appendix~\ref{app:code}.
