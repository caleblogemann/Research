% !TEX root = main.tex

\section{Introduction}\label{sec:intro}
  In this paper we look at the model equation,
  \begin{equation}
    q_t + \p{q^2 - q^3}_x = -\p{q^3 q_{xxx}}_x \quad (x, t) \in \br{a, b} \times \br{0, T}. \label{eq:thin_film_model}
  \end{equation}
  This equation describes the motion of a thin film of liquid flowing over a
  one-dimensional domain, \(\br{a, b}\), where \(q(x, t) \ge 0\) is the height of the
  liquid.
  This fluid is acted upon by gravity, by forces on the surface, and by surface tension.
  The surface forces can have many different causes, including wind shear forces,
  thermocapillary forces, or molecular forces.
  In all cases an equivalent model can be derived.
  This model is useful in many different applications including airplane
  de-icing~\cite{article:myers2002slowly, article:myers2002flow} and industrial coating.
  Some experimental study~\cite{article:cazabat1990fingering,
  article:kataoka1997theoretical, article:ludviksson1971dynamics} has been done and
  numerical results have shown good agreement with those experiments
  in~\cite{article:bertozzi1998contact}.

  Previous numerical methods for this type of equation have focused on finite difference
  approaches.
  Bertozzi and Brenner~\cite{bertozzi1997linear} used a fully implicit centered finite
  difference scheme to explore instabilities.
  Ha et al.~\cite{article:Ha2008} explored several different finite difference schemes,
  some fully implicit and some using the Crank-Nicolson method.
  In their analysis, they considered several different methods for the hyperbolic terms
  including WENO, Godunov, and an adapted upwind method.
  All of these methods were limited to just first or second order, and they required
  solving a Newton iteration.
  Finite difference methods also lack provable stability.
  % other drawbacks to finite difference methods

  We chose to use discontinuous Galerkin methods as they allow for high order
  convergence.
  The discontinuous Galerkin methods were first introduced by Reed and
  Hill~\cite{techreport:Reed1973}, and then were formalized by Cockburn and Shu
  in a series of papers~\cite{article:Cockburn1991I, article:Cockburn1989II,
  article:Cockburn1989III, article:Cockburn1990IV, article:cockburn1998V}.
  We use the original modal discontinuous Galerkin method as well as the local
  discontinuous Galerkin method.
  The local discontinuous Galerkin method was also formulated by Cockburn and
  Shu~\cite{article:Cockburn1998LDG} to handle convection-diffusion equations with the
  discontinuous Galerkin method.
  We use the modal discontinuous Galerkin method to discretize the convection term,
  \(\p{q^2 - q^3}_x\), and
  the local discontinuous Galerkin method to discretize the diffusion term,
  \(-\p{q^3 q_{xxx}}_x\).

  The nonlinear diffusion is much stiffer, as it has an infinite wavespeed, than the
  hyperbolic convection, which only has a finite wavespeed.
  If the diffusion is handled explicitly in time, then a very strict time step
  restriction is needed to insure stability.
  Therefore the nonlinear diffusion should be solved implicitly.
  The whole system could be solved implicitly, however we chose to use an
  implicit-explicit (IMEX) Runge-Kutta scheme for propagating in time.
  These schemes were first introduced by Ascher et al.~\cite{article:ascher1997implicit}
  and have been expanded on by Kennedy and Carpenter~\cite{kennedy2003additive} and
  Pareschi and Russo~\cite{article:pareschi2000IMEX}.
  The IMEX Runge-Kutta schemes allows us not only to solve the nonlinear diffusion
  implicitly, but it also allows for the nonlinear convection to be propagated
  explicitly.
  Solving the convection explicitly fully captures the nonlinear behavior without a
  nonlinear solve.

  The diffusion still requires a nonlinear solve, but this is simpler than solving both
  convection and diffusion nonlinearly.
  We solve the nonlinear system using a Picard iteration, first introduced by {\'E}mile
  Picard and then formalized by Lindel{\"o}f~\cite{lindelof1894application}.
  We find that very minimal number of iterations is required for the iteration to
  converge.
  In fact we find one iteration allows for high order convergence of smooth solutions,
  and that one iteration provides good results for travelling shock solutions.
  With these methods we have demonstrated up to third order accuracy with a manufactured
  solution example.
  We also demonstrate the nonlinear traveling wave behavior through several numerical
  examples introduced by Bertozzi~\cite{article:Bertozzi1999}.
  These examples demonstrate that the steady state traveling wave may not be unique.
  They also show a double shock structure with an undercompressive shock.