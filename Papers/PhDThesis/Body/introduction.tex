% Chapter 1 of the Thesis Template File
\chapter{Introduction}

In this thesis, I present discontinuous Galerkin methods for shallow water moment
models.
The shallow water moment equations (SWME) were first introduced by Kowalski and
Torrilhon\cite{kowalski2017moment} in 2017.
One of the main drawbacks of the standard shallow water equations is the lack of
vertical resolution in the velocity.
In other words the velocity of the fluid is identical at the top and bottom of the
fluid.
This may not always be a reasonable physical assumption.
In particular when the model includes some drag along the bottom topography, the
constant velocity restriction allows the drag to have a disproportionate effect on the
height of the fluid.
However users would like to avoid these issues without needing to go to the
computational expense of a fully vertically resolved model.
The shallow water moment equations propose a solution to this problem, by adding
moments to the velocity profile.
This allows for a more complicated description of the velocity, while still retaining
relatively few equations in a lower dimensional problem.

One shortcoming of SWME model is that when there is more than one moment, the system
is no longer globally hyperbolic.
A modified model introduced by Koellermeier and
Pimentel-Garcia\cite{koellermeier2020steady} in 2020, resolves this issue.
This model is known as the shallow water linearized moment equations (SWLME), and
is provable hyperbolic.
This model is derived by removing some terms from the SWME, while still retaining
many of it's desirable features.
In this work I expand on this model slightly by proving that it is hyperbolic in
two dimensions as well as in one dimension.

I chose to numerically solve these systems using the discontinuous Galerkin method.
The discontinuous Galerkin method was first introduced by Reed and
Hill\cite{techreport:Reed1973} in their 1973 report on neutron transport.
The method was later formalized and popularized by a series of papers by Cockburn and
Shu\cite{article:Cockburn1989II,article:Cockburn1991I,article:Cockburn1989III,article:Cockburn1990IV,article:Cockburn1998V}
in the 1990s.
The discontinuous Galerkin method is a finite element method where the solution space
and test space are discontinuous over the mesh element faces.
% TODO: add more details about DG method

The main numerical challenge in discretizing the SWME or the SWLME, is that they both
contain nonconservative products.
In other words they can't be written conservative or divergence form.
The traditional theory of weak solutions using distributions does not apply to
nonconservative equations.
Instead this thesis relies on the theory laid out by Dal Maso, LeFloch, and
Murat\cite{dal1995definition} in their 1995 paper.
The main idea of this theory is to include a regularization path at any possible
discontinuities.
This regularization path smooths out any discontinuities and then value of the solution
is examined as the limit approaches the discontinuity.
Through this limiting process Dal Maso et. al. are able to define a path integral as
a measure which represents the nonconservative product at discontinuities.

Rhebergen et al.\cite{rhebergen2008discontinuous} were the first to translate the
nonconservative product theory of Dal Maso, LeFloch, and Murat, to the discontinuous
Galerkin method.







%\chapterbib
%\bibliographystyle{apa}
%\bibliography{Reference/mybib}


%\renewcommand{\bibname}{\centerline{BIBLIOGRAPHY}}
%\bibliographystyle{apa}
%\newpage
%\phantomsection
%\addtocontents{toc}{\def\protect\@chapapp{}}
%\addcontentsline{toc}{chapter}{BIBLIOGRAPHY}
%\addtocontents{toc}{\def\protect\@chapapp{CHAPTER\ }}
%\bibliography{mybib}
