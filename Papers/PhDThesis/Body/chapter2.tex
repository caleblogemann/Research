% Chapter 2 of the Thesis Template File
%   which includes bibliographic references.
\chapter{The Models}


\section{Shallow Water Moment Models}

The shallow water moment equations (SWME) were first introduced by
Kowalski and Torrilhon\cite{}

\subsection{Derivation}
  We begin by considering the Navier-Stokes equations,
  \begin{align}
    \div{\v{u}} &= 0 \\
    \v{u}_t + \div*{\v{u}\v{u}} &= - \frac{1}{\rho} \grad{p}
    + \frac{1}{\rho} \div{\sigma} + \v{g},
  \end{align}
  where \(\v{u} = \br{u, v, w}^T\) is the vector of velocities, \(p\) is the pressure,
  \(\rho \) is the constant density, \(\sigma \) is the deviatoric stress tensor, and
  \(\v{g}\) is the gravitational force vector.
  We also have two boundaries, the bottom topography \(h_b(t, x, y)\), and the free
  surface \(h_s(t, x, y)\).
  At both of these boundaries the kinematic boundary conditions are in effect and can
  be expressed as
  \begin{align}
    \p{h_s}_t + \br{u(t, x, y, h_s), v(t, x, y, h_s)}^T \cdot \grad{h_s}
    &= w(t, x, y, h_s) \\
    \p{h_b}_t + \br{u(t, x, y, h_b), v(t, x, y, h_b)}^T \cdot \grad{h_b}
    &= w(t, x, y, h_b).
  \end{align}
  In practice the bottom topography is unchanging in time, but we express \(h_b\) with
  time dependence to allow for a symmetric representation of the boundary conditions.

\subsubsection{Dimensional Analysis}
  Now we consider the characteristic scales of the problem.
  Let \(L\) be the characteristic horizontal length scale, and let \(H\) be the
  characteristic vertical length scale.
  For this problem we assume that \(H << L\) and we denote the ratio of these
  lengths as \(\varepsilon = H/L\).
  With these characteristic lengths we can scale the length variables to a
  nondimensional form
  \begin{equation}
    x = L\hat{x}, \quad y = L\hat{y}, \quad z = H\hat{z}.
  \end{equation}
  Now let \(U\) be the characteristic horizontal velocity, then because of the
  shallowness the characteristic vertical velocity will be \(\varepsilon U\).
  Therefore the velocity variables can be scaled as follows,
  \begin{equation}
    u = U\hat{u}, \quad v = U\hat{v}, \quad w = \varepsilon U \hat{w}.
  \end{equation}
  Now with the characteristic length and velocity, the time scaling can be described
  as
  \begin{equation}
    t = \frac{L}{U}\hat{t}
  \end{equation}
  The pressure will be scaled by the characteristic height, \(H\), and the stresses
  will be scaled by a characteristic stress, \(S\).
  It is assumed that the basal shear stresses, \(\sigma_{xz}\) and \(\sigma_{yz}\) are
   of larger order than the lateral shear stress, \(\sigma_{xy}\), and the normal
  stresses, \(\sigma_{xx}\), \(\sigma_{yy}\), and \(\sigma_{zz}\), so that
  \begin{equation}
    p = \rho g H \hat{p}, \quad \sigma_{xz/yz} = S\hat{\sigma}_{xz/yz}, \quad
    \sigma_{xx/xy/yy/zz} = \varepsilon S \hat{\sigma}_{xx/xy/yy/zz}.
  \end{equation}

  Substituting all of these scaled variables into the Navier-Stokes system gives,
  \begin{align*}
    \hat{u}_{\hat{x}} + \hat{v}_{\hat{y}} + \hat{w}_{\hat{z}} &= 0 \\
    \varepsilon F^2 \p{\hat{u}_{\hat{t}} + \p{\hat{u}^2}_{\hat{x}}
      + \p{\hat{u}\hat{v}}_{\hat{y}} + \p{\hat{u}\hat{w}}_{\hat{z}}}
      &= -\varepsilon \hat{p}_{\hat{x}}
      + G
      \p{\varepsilon^2 \p{\hat{\sigma}_{xx}}_{\hat{x}}
        + \varepsilon^2 \p{\hat{\sigma}_{xy}}_{\hat{y}}
        + \p{\hat{\sigma}_{xz}}_{\hat{z}}}
      + e_x \\
    \varepsilon F^2
      \p{\hat{v}_{\hat{t}}
        + \p{\hat{u}\hat{v}}_{\hat{x}}
        + \p{\hat{v}^2}_{\hat{y}}
        + \p{\hat{v}\hat{w}}_{\hat{z}}
      }
      &=
      -\varepsilon \hat{p}_{\hat{y}}
      + G
      \p{\varepsilon^2 \p{\hat{\sigma}_{xy}}_{\hat{x}}
        + \varepsilon^2 \p{\hat{\sigma}_{yy}}_{\hat{y}}
        + \p{\hat{\sigma}_{yz}}_{\hat{z}}
      } + e_y \\
    \varepsilon^2 F^2
      \p{\hat{w}_{\hat{t}}
        + \p{\hat{u}\hat{w}}_{\hat{x}}
        + \p{\hat{v}\hat{w}}_{\hat{x}}
        + \p{\hat{w}^2}_{\hat{z}}
      }
      &= - \hat{p}_{\hat{z}}
      + \varepsilon G
      \p{\p{\hat{\sigma}_{xz}}_{\hat{x}}
        + \p{\hat{\sigma}_{yz}}_{\hat{y}}
        + \p{\hat{\sigma}_{zz}}_{\hat{z}}
      } + e_z \\
      F = \frac{U}{\sqrt{gH}} \approx 1, &\quad G = \frac{S}{\rho g H} < 1
  \end{align*}

  Drop terms with \(\varepsilon^2\) and \(\varepsilon G\), giving
  \begin{align*}
    \hat{u}_{\hat{x}} + \hat{v}_{\hat{y}} + \hat{w}_{\hat{z}} &= 0 \\
    \varepsilon F^2 \p{\hat{u}_{\hat{t}} + \p{\hat{u}^2}_{\hat{x}}
      + \p{\hat{u}\hat{v}}_{\hat{y}} + \p{\hat{u}\hat{w}}_{\hat{z}}}
      &= -\varepsilon \hat{p}_{\hat{x}}
      + G \p{\hat{\sigma}_{xz}}_{\hat{z}}
      + e_x \\
    \varepsilon F^2
      \p{\hat{v}_{\hat{t}}
        + \p{\hat{u}\hat{v}}_{\hat{x}}
        + \p{\hat{v}^2}_{\hat{y}}
        + \p{\hat{v}\hat{w}}_{\hat{z}}
      }
      &=
      -\varepsilon \hat{p}_{\hat{y}}
      + G \p{\hat{\sigma}_{yz}}_{\hat{z}}
      + e_y \\
      \hat{p}_{\hat{z}} &= e_z
  \end{align*}
  where we can solve for the hydrostatic pressure
  \begin{align*}
    \hat{p}(\hat{t}, \hat{x}, \hat{y}) = \p{\hat{h}_s(\hat{t}, \hat{x}, \hat{y}) - \hat{z}} e_z
  \end{align*}

\subsubsection{Mapping}

% Below \subsubsection
% Sectional commands: \paragraph and \subparagraph may also be used

%\chapterbib

%\bibliographystyle{apa}
%\bibliography{Reference/mybib}

