% Chapter 2 of the Thesis Template File
%   which includes bibliographic references.
\chapter{The Models}

\section{Shallow Water Moment Models}
  The shallow water moment equations (SWME) were first introduced by Kowalski and
  Torrilhon.
  The goal of this new model is to add vertical resolution to the velocity of the shallow
  water equations.
  The standard shallow water equations make several key assumptions.
  The shallow water equations assume hydrostatic pressure and that the horizontal
  velocity is constant in the vertical direction.
  The assumption that the horizontal velocity is constant in the vertical direction
  is particularly restricting.
  One common approach to add vertical resolution the the shallow water models is the
  so-called multilayer shallow water model.
  The multilayer shallow water model assumes that the horizontal velocity consists of
  multiple layers of constant velocity.
  This approach can reflect nature, where the oceans and atmosphere do have multiple
  layers.
  However the multilayer model has a significant numerical downside.
  The multilayer model is not globally hyperbolic, which means that the problem can
  become ill-posed.
  When the velocities of the different layers become too different the system is no
  longer hyperbolic.
  In this case the fluid should create vortices at the interface between the layers.
  However the multilayer shallow water model does not allow for these roll-ups and so
  becomes ill-posed.

  Kowalski and Torrilhon have introduced a new approach to adding vertical resolution
  to the shallow water equations, which has better hyperbolicity properties.
  The main idea of their approach is to approximate the horizontal velocity as
  an Ansatz expansion in the vertical direction, that is the velocities can be represented
  as
  \begin{align}
    u(x, y, z, t) &= u_m(x, y, t) + \sum{j=1}{N}{\alpha_j(x, y, t) \phi_j(z)} \\
    v(x, y, z, t) &= v_m(x, y, t) + \sum{j=1}{N}{\beta_j(x, y, t) \phi_j(z)},
  \end{align}
  where \(u_m(x, y, t)\) and \(v_m(x, y, t)\) are the mean velocities in the \(x\) and
  \(y\) directions respectively.
  In general the functions \(\phi_j\) can be arbitrary.
  In fact if \(\phi_j\) are characteristic functions, then the multilayer shallow water
  model can be derived.
  However in this work we will assume that \(\phi_j\) are polynomials.
  This approach maintains computational efficiency compared with fully vertically resolved
  models.

\subsection{Derivation}
  We begin by considering the Navier-Stokes equations,
  \begin{align}
    \div{\v{u}} &= 0 \\
    \v{u}_t + \div*{\v{u}\v{u}} &= - \frac{1}{\rho} \grad{p}
    + \frac{1}{\rho} \div{\sigma} + \v{g},
  \end{align}
  where \(\v{u} = \br{u, v, w}^T\) is the vector of velocities, \(p\) is the pressure,
  \(\rho \) is the constant density, \(\sigma \) is the deviatoric stress tensor, and
  \(\v{g}\) is the gravitational force vector.
  We also have two boundaries, the bottom topography \(h_b(t, x, y)\), and the free
  surface \(h_s(t, x, y)\).
  At both of these boundaries the kinematic boundary conditions are in effect and can
  be expressed as
  \begin{align}
    \p{h_s}_t + \br{u(t, x, y, h_s), v(t, x, y, h_s)}^T \cdot \grad{h_s}
    &= w(t, x, y, h_s) \\
    \p{h_b}_t + \br{u(t, x, y, h_b), v(t, x, y, h_b)}^T \cdot \grad{h_b}
    &= w(t, x, y, h_b).
  \end{align}
  In practice the bottom topography is unchanging in time, but we express \(h_b\) with
  time dependence to allow for a symmetric representation of the boundary conditions.

\subsubsection{Dimensional Analysis}
  Now we consider the characteristic scales of the problem.
  Let \(L\) be the characteristic horizontal length scale, and let \(H\) be the
  characteristic vertical length scale.
  For this problem we assume that \(H << L\) and we denote the ratio of these
  lengths as \(\varepsilon = H/L\).
  With these characteristic lengths we can scale the length variables to a
  nondimensional form
  \begin{equation}
    x = L\hat{x}, \quad y = L\hat{y}, \quad z = H\hat{z}.
  \end{equation}
  Now let \(U\) be the characteristic horizontal velocity, then because of the
  shallowness the characteristic vertical velocity will be \(\varepsilon U\).
  Therefore the velocity variables can be scaled as follows,
  \begin{equation}
    u = U\hat{u}, \quad v = U\hat{v}, \quad w = \varepsilon U \hat{w}.
  \end{equation}
  Now with the characteristic length and velocity, the time scaling can be described
  as
  \begin{equation}
    t = \frac{L}{U}\hat{t}
  \end{equation}
  The pressure will be scaled by the characteristic height, \(H\), and the stresses
  will be scaled by a characteristic stress, \(S\).
  It is assumed that the basal shear stresses, \(\sigma_{xz}\) and \(\sigma_{yz}\) are
   of larger order than the lateral shear stress, \(\sigma_{xy}\), and the normal
  stresses, \(\sigma_{xx}\), \(\sigma_{yy}\), and \(\sigma_{zz}\), so that
  \begin{equation}
    p = \rho g H \hat{p}, \quad \sigma_{xz/yz} = S\hat{\sigma}_{xz/yz}, \quad
    \sigma_{xx/xy/yy/zz} = \varepsilon S \hat{\sigma}_{xx/xy/yy/zz}.
  \end{equation}

  Substituting all of these scaled variables into the Navier-Stokes system gives,
  \begin{align}
    \hat{u}_{\hat{x}} + \hat{v}_{\hat{y}} + \hat{w}_{\hat{z}} &= 0 \\
    \varepsilon F^2 \p{\hat{u}_{\hat{t}} + \p{\hat{u}^2}_{\hat{x}}
      + \p{\hat{u}\hat{v}}_{\hat{y}} + \p{\hat{u}\hat{w}}_{\hat{z}}}
      &= -\varepsilon \hat{p}_{\hat{x}}
      + G
      \p{\varepsilon^2 \p{\hat{\sigma}_{xx}}_{\hat{x}}
        + \varepsilon^2 \p{\hat{\sigma}_{xy}}_{\hat{y}}
        + \p{\hat{\sigma}_{xz}}_{\hat{z}}}
      + e_x \\
    \varepsilon F^2
      \p{\hat{v}_{\hat{t}}
        + \p{\hat{u}\hat{v}}_{\hat{x}}
        + \p{\hat{v}^2}_{\hat{y}}
        + \p{\hat{v}\hat{w}}_{\hat{z}}
      }
      &=
      -\varepsilon \hat{p}_{\hat{y}}
      + G
      \p{\varepsilon^2 \p{\hat{\sigma}_{xy}}_{\hat{x}}
        + \varepsilon^2 \p{\hat{\sigma}_{yy}}_{\hat{y}}
        + \p{\hat{\sigma}_{yz}}_{\hat{z}}
      } + e_y \\
    \varepsilon^2 F^2
      \p{\hat{w}_{\hat{t}}
        + \p{\hat{u}\hat{w}}_{\hat{x}}
        + \p{\hat{v}\hat{w}}_{\hat{x}}
        + \p{\hat{w}^2}_{\hat{z}}
      }
      &= - \hat{p}_{\hat{z}}
      + \varepsilon G
      \p{\p{\hat{\sigma}_{xz}}_{\hat{x}}
        + \p{\hat{\sigma}_{yz}}_{\hat{y}}
        + \p{\hat{\sigma}_{zz}}_{\hat{z}}
      } + e_z \\
      F = \frac{U}{\sqrt{gH}} \approx 1, &\quad G = \frac{S}{\rho g H} < 1
  \end{align}

  Drop terms with \(\varepsilon^2\) and \(\varepsilon G\), giving
  \begin{align}
    \hat{u}_{\hat{x}} + \hat{v}_{\hat{y}} + \hat{w}_{\hat{z}} &= 0 \\
    \varepsilon F^2 \p{\hat{u}_{\hat{t}} + \p{\hat{u}^2}_{\hat{x}}
      + \p{\hat{u}\hat{v}}_{\hat{y}} + \p{\hat{u}\hat{w}}_{\hat{z}}}
      &= -\varepsilon \hat{p}_{\hat{x}}
      + G \p{\hat{\sigma}_{xz}}_{\hat{z}}
      + e_x \\
    \varepsilon F^2
      \p{\hat{v}_{\hat{t}}
        + \p{\hat{u}\hat{v}}_{\hat{x}}
        + \p{\hat{v}^2}_{\hat{y}}
        + \p{\hat{v}\hat{w}}_{\hat{z}}
      }
      &=
      -\varepsilon \hat{p}_{\hat{y}}
      + G \p{\hat{\sigma}_{yz}}_{\hat{z}}
      + e_y \\
      \hat{p}_{\hat{z}} &= e_z
  \end{align}
  where we can solve for the hydrostatic pressure
  \begin{align}
    \hat{p}(\hat{t}, \hat{x}, \hat{y}) = \p{\hat{h}_s(\hat{t}, \hat{x}, \hat{y}) - \hat{z}} e_z
  \end{align}

  For the rest of the derivation we will transform back into dimensional variables for
  readability purposes.
  \begin{align}
    u_x + v_y + w_z &= 0 \\
    u_t + \p{u^2}_x + \p{uv}_y + \p{uw}_z
      &= -\frac{1}{\rho} p_x + \frac{1}{\rho} \p{\sigma_{xz}}_z + g e_x \\
    v_t + \p{uv}_x + \p{v^2}_y + \p{vw}_z
      &= -\frac{1}{\rho} p_y + \frac{1}{\rho} \p{\sigma_{yz}}_z + g e_y \\
    p(t, x, y, z) &= \p{h_s(t, x, y) - z} \rho g e_z
  \end{align}

\subsubsection{Mapping}
  In order to make this system more accessible we will map the vertical variable \(z\)
  to the normalized variable \(\zeta \), through the transformation
  \begin{gather}
    \zeta(t, x, y, z) = \frac{z - h_b(t, x, y)}{h(t, x, y)},
  \end{gather}
  or equivalently
  \begin{gather}
    z(t, x, y, \zeta) = h(t, x, y) \zeta + h_b(t, x, y)
  \end{gather}
  where \(h(t, x, y) = h_s(t, x, y) - h_b(t, x, y)\).
  This transformation maps the vertical variable, \(z\) onto \(\zeta \in \br{0, 1}\).
  In order to transform the partial differential equations we consider a function
  \(\Psi(t, x, y, z)\), then it's mapped counterpart \(\tilde{\Psi}(t, x, y, \zeta)\)
  can be described as
  \[
    \tilde{\Psi}(t, x, y, \zeta) = \Psi\p{t, x, y, z(t, x, y, \zeta)}
      = \Psi\p{t, x, y, h(t, x, y) \zeta + h_b(t, x, y)},
  \]
  or equivalently
  \[
    \Psi(t, x, y, z) = \tilde{\Psi}\p{t, x, y, \zeta(t, x, y, z)}
      = \tilde{\Psi}\p{t, x, y, \frac{z - h_b(t, x, y)}{h(t, x, y)}}.
  \]
  We also need to be able to map derivatives of functions in order to be able to map
  the differential equations.
  This can be described
  \begin{align}
    \Psi_z(t, x, y, z) &= \p{\tilde{\Psi}\p{t, x, y, \zeta(t, x, y, z)}}_z \\
    \Psi_z(t, x, y, z) &= \tilde{\Psi}_{\zeta}\p{t, z, y, \zeta(t, x, y, z)} \zeta_z(t, x, y, z) \\
    \Psi_z(t, x, y, z) &= \tilde{\Psi}_{\zeta}\p{t, z, y, \zeta(t, x, y, z)} \frac{1}{h(t, x, y)} \\
    h(t, x, y) \Psi_z(t, x, y, z) &= \tilde{\Psi}_{\zeta}\p{t, z, y, \zeta(t, x, y, z)} \\
    h \Psi_z &= \tilde{\Psi}_{\zeta}
  \end{align}

  For the other variables, \(\set{t, x, y}\), the partial derivatives are identical.
  Let \(s \in \set{t, x, y}\), then
  \begin{align}
    \zeta_s(t, x, y, z) &= \p{\frac{z - h_b(t, x, y)}{h(t, x, y)}}_s \\
    &= -\frac{\p{z - h_b(t, x, y)}h_s(t, x, y)}{h\p{t, x, y}^2} - \frac{\p{h_b}_s(t, x, y)}{h(t, x, y)} \\
    &= -\zeta(t, x, y, z)\frac{h_s(t, x, y)}{h\p{t, x, y}} - \frac{\p{h_b}_s(t, x, y)}{h(t, x, y)} \\
    &= -\frac{\zeta(t, x, y, z)h_s(t, x, y) + \p{h_b}_s(t, x, y)}{h\p{t, x, y}}
  \end{align}
  and
  \begin{align}
    \Psi_s(t, x, y, z) &= \p{\tilde{\Psi}\p{t, x, y, \zeta(t, x, y, z)}}_s \\
    \Psi_s(t, x, y, z) &= \tilde{\Psi}_s\p{t, x, y, \zeta(t, x, y, z)}
      + \tilde{\Psi}_{\zeta}(t, x, y, \zeta(t, x, y, z)) \zeta_s(t, x, y, z) \\
    \Psi_s(t, x, y, z) &= \tilde{\Psi}_s\p{t, x, y, \zeta}
      - \tilde{\Psi}_{\zeta}(t, x, y, \zeta)
      \p{\frac{\zeta h_s(t, x, y) + \p{h_b}_s(t, x, y)}{h\p{t, x, y}}} \\
    h(t, x, y)\Psi_s(t, x, y, z) &= h(t, x, y)\tilde{\Psi}_s\p{t, x, y, \zeta}
      - \tilde{\Psi}_{\zeta}(t, x, y, \zeta) \p{\zeta h_s(t, x, y) + \p{h_b}_s(t, x, y)} \\
    h(t, x, y)\Psi_s(t, x, y, z) &= h(t, x, y)\tilde{\Psi}_s\p{t, x, y, \zeta}
      - \tilde{\Psi}_{\zeta}(t, x, y, \zeta) \p{\zeta h_s(t, x, y) + \p{h_b}_s(t, x, y)} \\
    h\Psi_s &= h\tilde{\Psi}_s - \tilde{\Psi}_{\zeta} \p{\zeta h_s + \p{h_b}_s} \\
    h\Psi_s &= h\tilde{\Psi}_s + h_s\tilde{\Psi}
      - h_s\tilde{\Psi} - \tilde{\Psi}_{\zeta} \p{\zeta h + h_b}_s \\
    h\Psi_s &= \p{h\tilde{\Psi}}_s - \p{h_s \tilde{\Psi}
      + \tilde{\Psi}_{\zeta} \p{\zeta h + h_b}_s} \\
    h\Psi_s &= \p{h\tilde{\Psi}}_s
      - \p{\p{\p{\zeta h + h_b}_{\zeta}}_s \tilde{\Psi} + \tilde{\Psi}_{\zeta} \p{\zeta h + h_b}_s} \\
    h\Psi_s &= \p{h\tilde{\Psi}}_s
      - \p{\p{\p{\zeta h + h_b}_s}_{\zeta} \tilde{\Psi} + \tilde{\Psi}_{\zeta} \p{\zeta h + h_b}_s} \\
    h\Psi_s &= \p{h\tilde{\Psi}}_s
      - \p{\p{\zeta h + h_b}_s \tilde{\Psi}}_{\zeta}
  \end{align}

\paragraph{Mapping of the Mass Balance Equation}
  Now we can use these differential transformations to map the continuity equation
  or mass balance equation onto the normalized space.
  We begin by multiplying the continuity equation by \(h\)
  \begin{gather}
    h\p{u_x + v_y + w_z} = 0,
  \end{gather}
  and then transforming from \(z\) to \(\zeta \)
  \begin{gather}
    % \p{h\tilde{u}}_x - \p{\p{\zeta h + h_b}_x \tilde{u}}_{\zeta}
    %   + \p{h\tilde{v}}_y - \p{\p{\zeta h + h_b}_y \tilde{v}}_{\zeta}
    %   + \p{\tilde{w}}_{\zeta} = 0 \\
    \p{h\tilde{u}}_x + \p{h\tilde{v}}_y
      + \p{\tilde{w} - \p{\zeta h + h_b}_x \tilde{u} - \p{\zeta h + h_b}_y \tilde{v}}_{\zeta} = 0.
  \end{gather}

  We can then integrate over \(\zeta \) to find an explicit expression for \(w\) the
  vertical velocity.
  \begin{gather}
    \tilde{w}(t, x, y, \zeta) - \tilde{w}(t, x, y, 0) = \nonumber \\
    -\dintt{0}{\zeta}{\p{h\tilde{u}}_x}{\zeta'}
      - \dintt{0}{\zeta}{\p{h\tilde{v}}_y}{\zeta'}
      + \p{\zeta h + h_b}_x \tilde{u} + \p{\zeta h + h_b}_y \tilde{v}
      - \p{h_b}_x \tilde{u} - \p{h_b}_y \tilde{v}.
  \end{gather}
  This can be simplified using the kinematic boundary condition at the bottom surface,
  to show that the vertical velocity can be expressed as
  \begin{gather}
    \tilde{w}(t, x, y, \zeta) =
    -\dintt{0}{\zeta}{\p{h\tilde{u}}_x}{\zeta'}
      - \dintt{0}{\zeta}{\p{h\tilde{v}}_y}{\zeta'}
      + \p{\zeta h + h_b}_x \tilde{u} + \p{\zeta h + h_b}_y \tilde{v}.
      \label{eq:vertical_velocity}
  \end{gather}
  Lastly by consider the vertical velocity at the free surface and using the kinematic
  boundary condition at that surface we arrive at the mass conservation equation,
  \begin{align}
    h_t + \p{hu_m}_x + \p{hv_m}_y = 0, \label{eq:mass_conservation}
  \end{align}
  where \(u_m = \dintt{0}{1}{\tilde{u}}{\zeta}\) and
  \(v_m = \dintt{0}{1}{\tilde{v}}{\zeta}\) are the mean velocities in the \(x\) and
  \(y\) directions respectively.
  This mass conservation equation is identical to the corresponding equation in the
  standard shallow water equations.

\paragraph{Mapping of the Momentum Equations}
  Next we map the conservation of momentum equations.
  Again we multiply by \(h\),
  \begin{gather}
      hu_t + h\p{u^2}_x + h\p{uv}_y + h\p{uw}_z + \frac{1}{\rho} hp_x
        = \frac{1}{\rho} h\p{\sigma_{xz}}_z + g h e_x
  \end{gather}
  and transform from \(z\) to \(\zeta \),
  \begin{gather}
      \p{h\tilde{u}}_t + \p{h\tilde{u}^2}_x + \p{h\tilde{u}\tilde{v}}_y
        + \p{\tilde{u} \p{\tilde{w} - \p{\zeta h + h_b}_t
        - \p{\zeta h + h_b}_x \tilde{u} - \p{\zeta h + h_b}_y \tilde{v}}}_{\zeta} \\
        + \frac{1}{\rho}\p{h\tilde{p}}_x
        - \frac{1}{\rho}\p{\p{\zeta h + h_b}_x \tilde{p}}_{\zeta}
        = \frac{1}{\rho} \p{\tilde{\sigma}_{xz}}_{\zeta} + g h e_x \\
  \end{gather}
% Below \subsubsection
% Sectional commands: \paragraph and \subparagraph may also be used

%\chapterbib

%\bibliographystyle{apa}
%\bibliography{Reference/mybib}

