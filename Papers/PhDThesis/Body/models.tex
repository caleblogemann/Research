% Chapter 2 of the Thesis Template File
%   which includes bibliographic references.

\newcommand{\nmom}[0]{N}

\chapter{The Models}

\section{Shallow Water Moment Models}
  The shallow water moment equations (SWME) were first introduced by Kowalski and
  Torrilhon.
  The goal of this new model is to add vertical resolution to the velocity of the shallow
  water equations.
  The standard shallow water equations make several key assumptions.
  The shallow water equations assume hydrostatic pressure and that the horizontal
  velocity is constant in the vertical direction.
  The assumption that the horizontal velocity is constant in the vertical direction
  is particularly restricting.
  One common approach to add vertical resolution the the shallow water models is the
  so-called multilayer shallow water model.
  The multilayer shallow water model assumes that the horizontal velocity consists of
  multiple layers of constant velocity.
  This approach can reflect nature, where the oceans and atmosphere do have multiple
  layers.
  However the multilayer model has a significant numerical downside.
  The multilayer model is not globally hyperbolic, which means that the problem can
  become ill-posed.
  When the velocities of the different layers become too different the system is no
  longer hyperbolic.
  In this case the fluid should create vortices at the interface between the layers.
  However the multilayer shallow water model does not allow for these roll-ups and so
  becomes ill-posed.

  Kowalski and Torrilhon have introduced a new approach to adding vertical resolution
  to the shallow water equations, which has better hyperbolicity properties.
  The main idea of their approach is to approximate the horizontal velocity as
  an Ansatz expansion in the vertical direction, that is the velocities can be represented
  as
  \begin{align}
    u(x, y, z, t) &= u_m(x, y, t) + \sum{j=1}{\nmom}{\alpha_j(x, y, t) \phi_j(z)} \\
    v(x, y, z, t) &= v_m(x, y, t) + \sum{j=1}{\nmom}{\beta_j(x, y, t) \phi_j(z)},
  \end{align}
  where \(u_m(x, y, t)\) and \(v_m(x, y, t)\) are the mean velocities in the \(x\) and
  \(y\) directions respectively.
  In general the functions \(\phi_j\) can be arbitrary.
  In fact if \(\phi_j\) are characteristic functions, then the multilayer shallow water
  model can be derived.
  However in this work we will assume that \(\phi_j\) are polynomials.
  This approach maintains computational efficiency compared with fully vertically resolved
  models.

\subsection{Derivation}
  We begin by considering the Navier-Stokes equations,
  \begin{align}
    \div{\v{u}} &= 0 \\
    \v{u}_t + \div*{\v{u}\v{u}} &= - \frac{1}{\rho} \grad{p}
    + \frac{1}{\rho} \div{\sigma} + \v{g},
  \end{align}
  where \(\v{u} = \br{u, v, w}^T\) is the vector of velocities, \(p\) is the pressure,
  \(\rho \) is the constant density, \(\sigma \) is the deviatoric stress tensor, and
  \(\v{g}\) is the gravitational force vector.
  We also have two boundaries, the bottom topography \(h_b(t, x, y)\), and the free
  surface \(h_s(t, x, y)\).
  At both of these boundaries the kinematic boundary conditions are in effect and can
  be expressed as
  \begin{align}
    \p{h_s}_t + \br{u(t, x, y, h_s), v(t, x, y, h_s)}^T \cdot \grad{h_s}
    &= w(t, x, y, h_s) \\
    \p{h_b}_t + \br{u(t, x, y, h_b), v(t, x, y, h_b)}^T \cdot \grad{h_b}
    &= w(t, x, y, h_b).
  \end{align}
  In practice the bottom topography is unchanging in time, but we express \(h_b\) with
  time dependence to allow for a symmetric representation of the boundary conditions.

\subsubsection{Dimensional Analysis}
  Now we consider the characteristic scales of the problem.
  Let \(L\) be the characteristic horizontal length scale, and let \(H\) be the
  characteristic vertical length scale.
  For this problem we assume that \(H << L\) and we denote the ratio of these
  lengths as \(\varepsilon = H/L\).
  With these characteristic lengths we can scale the length variables to a
  nondimensional form
  \begin{equation}
    x = L\hat{x}, \quad y = L\hat{y}, \quad z = H\hat{z}.
  \end{equation}
  Now let \(U\) be the characteristic horizontal velocity, then because of the
  shallowness the characteristic vertical velocity will be \(\varepsilon U\).
  Therefore the velocity variables can be scaled as follows,
  \begin{equation}
    u = U\hat{u}, \quad v = U\hat{v}, \quad w = \varepsilon U \hat{w}.
  \end{equation}
  Now with the characteristic length and velocity, the time scaling can be described
  as
  \begin{equation}
    t = \frac{L}{U}\hat{t}
  \end{equation}
  The pressure will be scaled by the characteristic height, \(H\), and the stresses
  will be scaled by a characteristic stress, \(S\).
  It is assumed that the basal shear stresses, \(\sigma_{xz}\) and \(\sigma_{yz}\) are
   of larger order than the lateral shear stress, \(\sigma_{xy}\), and the normal
  stresses, \(\sigma_{xx}\), \(\sigma_{yy}\), and \(\sigma_{zz}\), so that
  \begin{equation}
    p = \rho g H \hat{p}, \quad \sigma_{xz/yz} = S\hat{\sigma}_{xz/yz}, \quad
    \sigma_{xx/xy/yy/zz} = \varepsilon S \hat{\sigma}_{xx/xy/yy/zz}.
  \end{equation}

  Substituting all of these scaled variables into the Navier-Stokes system gives,
  \begin{align}
    \hat{u}_{\hat{x}} + \hat{v}_{\hat{y}} + \hat{w}_{\hat{z}} &= 0 \\
    \varepsilon F^2 \p{\hat{u}_{\hat{t}} + \p{\hat{u}^2}_{\hat{x}}
      + \p{\hat{u}\hat{v}}_{\hat{y}} + \p{\hat{u}\hat{w}}_{\hat{z}}}
      &= -\varepsilon \hat{p}_{\hat{x}}
      + G
      \p{\varepsilon^2 \p{\hat{\sigma}_{xx}}_{\hat{x}}
        + \varepsilon^2 \p{\hat{\sigma}_{xy}}_{\hat{y}}
        + \p{\hat{\sigma}_{xz}}_{\hat{z}}}
      + e_x \\
    \varepsilon F^2
      \p{\hat{v}_{\hat{t}}
        + \p{\hat{u}\hat{v}}_{\hat{x}}
        + \p{\hat{v}^2}_{\hat{y}}
        + \p{\hat{v}\hat{w}}_{\hat{z}}
      }
      &=
      -\varepsilon \hat{p}_{\hat{y}}
      + G
      \p{\varepsilon^2 \p{\hat{\sigma}_{xy}}_{\hat{x}}
        + \varepsilon^2 \p{\hat{\sigma}_{yy}}_{\hat{y}}
        + \p{\hat{\sigma}_{yz}}_{\hat{z}}
      } + e_y \\
    \varepsilon^2 F^2
      \p{\hat{w}_{\hat{t}}
        + \p{\hat{u}\hat{w}}_{\hat{x}}
        + \p{\hat{v}\hat{w}}_{\hat{x}}
        + \p{\hat{w}^2}_{\hat{z}}
      }
      &= - \hat{p}_{\hat{z}}
      + \varepsilon G
      \p{\p{\hat{\sigma}_{xz}}_{\hat{x}}
        + \p{\hat{\sigma}_{yz}}_{\hat{y}}
        + \p{\hat{\sigma}_{zz}}_{\hat{z}}
      } + e_z \\
      F = \frac{U}{\sqrt{gH}} \approx 1, &\quad G = \frac{S}{\rho g H} < 1
  \end{align}

  Drop terms with \(\varepsilon^2\) and \(\varepsilon G\), giving
  \begin{align}
    \hat{u}_{\hat{x}} + \hat{v}_{\hat{y}} + \hat{w}_{\hat{z}} &= 0 \\
    \varepsilon F^2 \p{\hat{u}_{\hat{t}} + \p{\hat{u}^2}_{\hat{x}}
      + \p{\hat{u}\hat{v}}_{\hat{y}} + \p{\hat{u}\hat{w}}_{\hat{z}}}
      &= -\varepsilon \hat{p}_{\hat{x}}
      + G \p{\hat{\sigma}_{xz}}_{\hat{z}}
      + e_x \\
    \varepsilon F^2
      \p{\hat{v}_{\hat{t}}
        + \p{\hat{u}\hat{v}}_{\hat{x}}
        + \p{\hat{v}^2}_{\hat{y}}
        + \p{\hat{v}\hat{w}}_{\hat{z}}
      }
      &=
      -\varepsilon \hat{p}_{\hat{y}}
      + G \p{\hat{\sigma}_{yz}}_{\hat{z}}
      + e_y \\
      \hat{p}_{\hat{z}} &= e_z
  \end{align}
  where we can solve for the hydrostatic pressure
  \begin{align}
    \hat{p}(\hat{t}, \hat{x}, \hat{y}) = \p{\hat{h}_s(\hat{t}, \hat{x}, \hat{y}) - \hat{z}} e_z
  \end{align}

  For the rest of the derivation we will transform back into dimensional variables for
  readability purposes.
  \begin{align}
    u_x + v_y + w_z &= 0 \\
    u_t + \p{u^2}_x + \p{uv}_y + \p{uw}_z
      &= -\frac{1}{\rho} p_x + \frac{1}{\rho} \p{\sigma_{xz}}_z + g e_x \\
    v_t + \p{uv}_x + \p{v^2}_y + \p{vw}_z
      &= -\frac{1}{\rho} p_y + \frac{1}{\rho} \p{\sigma_{yz}}_z + g e_y \\
    p(t, x, y, z) &= \p{h_s(t, x, y) - z} \rho g e_z
  \end{align}

\subsubsection{Mapping}
  In order to make this system more accessible we will map the vertical variable \(z\)
  to the normalized variable \(\zeta \), through the transformation
  \begin{gather}
    \zeta(t, x, y, z) = \frac{z - h_b(t, x, y)}{h(t, x, y)},
  \end{gather}
  or equivalently
  \begin{gather}
    z(t, x, y, \zeta) = h(t, x, y) \zeta + h_b(t, x, y)
  \end{gather}
  where \(h(t, x, y) = h_s(t, x, y) - h_b(t, x, y)\).
  This transformation maps the vertical variable, \(z\) onto \(\zeta \in \br{0, 1}\).
  In order to transform the partial differential equations we consider a function
  \(\Psi(t, x, y, z)\), then it's mapped counterpart \(\tilde{\Psi}(t, x, y, \zeta)\)
  can be described as
  \begin{gather}
    \tilde{\Psi}(t, x, y, \zeta) = \Psi\p{t, x, y, z(t, x, y, \zeta)}
      = \Psi\p{t, x, y, h(t, x, y) \zeta + h_b(t, x, y)},
  \end{gather}
  or equivalently
  \begin{gather}
    \Psi(t, x, y, z) = \tilde{\Psi}\p{t, x, y, \zeta(t, x, y, z)}
      = \tilde{\Psi}\p{t, x, y, \frac{z - h_b(t, x, y)}{h(t, x, y)}}.
  \end{gather}
  We also need to be able to map derivatives of functions in order to be able to map
  the differential equations.
  This can be described
  \begin{align}
    \Psi_z(t, x, y, z) &= \p{\tilde{\Psi}\p{t, x, y, \zeta(t, x, y, z)}}_z \\
    \Psi_z(t, x, y, z) &= \tilde{\Psi}_{\zeta}\p{t, z, y, \zeta(t, x, y, z)} \zeta_z(t, x, y, z) \\
    \Psi_z(t, x, y, z) &= \tilde{\Psi}_{\zeta}\p{t, z, y, \zeta(t, x, y, z)} \frac{1}{h(t, x, y)} \\
    h(t, x, y) \Psi_z(t, x, y, z) &= \tilde{\Psi}_{\zeta}\p{t, z, y, \zeta(t, x, y, z)} \\
    h \Psi_z &= \tilde{\Psi}_{\zeta}
  \end{align}

  For the other variables, \(\set{t, x, y}\), the partial derivatives are identical.
  Let \(s \in \set{t, x, y}\), then
  \begin{align}
    \zeta_s(t, x, y, z) &= \p{\frac{z - h_b(t, x, y)}{h(t, x, y)}}_s \\
    &= -\frac{\p{z - h_b(t, x, y)}h_s(t, x, y)}{h\p{t, x, y}^2} - \frac{\p{h_b}_s(t, x, y)}{h(t, x, y)} \\
    &= -\zeta(t, x, y, z)\frac{h_s(t, x, y)}{h\p{t, x, y}} - \frac{\p{h_b}_s(t, x, y)}{h(t, x, y)} \\
    &= -\frac{\zeta(t, x, y, z)h_s(t, x, y) + \p{h_b}_s(t, x, y)}{h\p{t, x, y}}
  \end{align}
  and
  \begin{align}
    \Psi_s(t, x, y, z) &= \p{\tilde{\Psi}\p{t, x, y, \zeta(t, x, y, z)}}_s \\
    \Psi_s(t, x, y, z) &= \tilde{\Psi}_s\p{t, x, y, \zeta(t, x, y, z)}
      + \tilde{\Psi}_{\zeta}(t, x, y, \zeta(t, x, y, z)) \zeta_s(t, x, y, z) \\
    \Psi_s(t, x, y, z) &= \tilde{\Psi}_s\p{t, x, y, \zeta}
      - \tilde{\Psi}_{\zeta}(t, x, y, \zeta)
      \p{\frac{\zeta h_s(t, x, y) + \p{h_b}_s(t, x, y)}{h\p{t, x, y}}} \\
    h(t, x, y)\Psi_s(t, x, y, z) &= h(t, x, y)\tilde{\Psi}_s\p{t, x, y, \zeta}
      - \tilde{\Psi}_{\zeta}(t, x, y, \zeta) \p{\zeta h_s(t, x, y) + \p{h_b}_s(t, x, y)} \\
    h(t, x, y)\Psi_s(t, x, y, z) &= h(t, x, y)\tilde{\Psi}_s\p{t, x, y, \zeta}
      - \tilde{\Psi}_{\zeta}(t, x, y, \zeta) \p{\zeta h_s(t, x, y) + \p{h_b}_s(t, x, y)} \\
    h\Psi_s &= h\tilde{\Psi}_s - \tilde{\Psi}_{\zeta} \p{\zeta h_s + \p{h_b}_s} \\
    h\Psi_s &= h\tilde{\Psi}_s + h_s\tilde{\Psi}
      - h_s\tilde{\Psi} - \tilde{\Psi}_{\zeta} \p{\zeta h + h_b}_s \\
    h\Psi_s &= \p{h\tilde{\Psi}}_s - \p{h_s \tilde{\Psi}
      + \tilde{\Psi}_{\zeta} \p{\zeta h + h_b}_s} \\
    h\Psi_s &= \p{h\tilde{\Psi}}_s
      - \p{\p{\p{\zeta h + h_b}_{\zeta}}_s \tilde{\Psi} + \tilde{\Psi}_{\zeta} \p{\zeta h + h_b}_s} \\
    h\Psi_s &= \p{h\tilde{\Psi}}_s
      - \p{\p{\p{\zeta h + h_b}_s}_{\zeta} \tilde{\Psi} + \tilde{\Psi}_{\zeta} \p{\zeta h + h_b}_s} \\
    h\Psi_s &= \p{h\tilde{\Psi}}_s
      - \p{\p{\zeta h + h_b}_s \tilde{\Psi}}_{\zeta}
  \end{align}

\paragraph{Mapping of the Mass Balance Equation}
  Now we can use these differential transformations to map the continuity equation
  or mass balance equation onto the normalized space.
  We begin by multiplying the continuity equation by \(h\)
  \begin{gather}
    h\p{u_x + v_y + w_z} = 0,
  \end{gather}
  and then transforming from \(z\) to \(\zeta \)
  \begin{gather}
    % \p{h\tilde{u}}_x - \p{\p{\zeta h + h_b}_x \tilde{u}}_{\zeta}
    %   + \p{h\tilde{v}}_y - \p{\p{\zeta h + h_b}_y \tilde{v}}_{\zeta}
    %   + \p{\tilde{w}}_{\zeta} = 0 \\
    \p{h\tilde{u}}_x + \p{h\tilde{v}}_y
      + \p{\tilde{w} - \p{\zeta h + h_b}_x \tilde{u} - \p{\zeta h + h_b}_y \tilde{v}}_{\zeta} = 0.
  \end{gather}
  We can then integrate over \(\zeta \) to find an explicit expression for \(w\) the
  vertical velocity.
  \begin{gather}
    \tilde{w}(t, x, y, \zeta) - \tilde{w}(t, x, y, 0) = \nonumber \\
    -\dintt{0}{\zeta}{\p{h\tilde{u}}_x}{\zeta'}
      - \dintt{0}{\zeta}{\p{h\tilde{v}}_y}{\zeta'}
      + \p{\zeta h + h_b}_x \tilde{u} + \p{\zeta h + h_b}_y \tilde{v}
      - \p{h_b}_x \tilde{u} - \p{h_b}_y \tilde{v}.
  \end{gather}
  This can be simplified using the kinematic boundary condition at the bottom surface,
  to show that the vertical velocity can be expressed as
  \begin{gather}
    \tilde{w}(t, x, y, \zeta) =
    -\dintt{0}{\zeta}{\p{h\tilde{u}}_x}{\zeta'}
      - \dintt{0}{\zeta}{\p{h\tilde{v}}_y}{\zeta'}
      + \p{\zeta h + h_b}_x \tilde{u} + \p{\zeta h + h_b}_y \tilde{v}.
      \label{eq:vertical_velocity}
  \end{gather}
  Lastly by consider the vertical velocity at the free surface and using the kinematic
  boundary condition at that surface we arrive at the mass conservation equation,
  \begin{align}
    h_t + \p{hu_m}_x + \p{hv_m}_y = 0, \label{eq:mass_conservation}
  \end{align}
  where \(u_m = \dintt{0}{1}{\tilde{u}}{\zeta}\) and
  \(v_m = \dintt{0}{1}{\tilde{v}}{\zeta}\) are the mean velocities in the \(x\) and
  \(y\) directions respectively.
  This mass conservation equation is identical to the corresponding equation in the
  standard shallow water equations.

\paragraph{Mapping of the Momentum Equations}
  Next we map the conservation of momentum equations.
  Again we multiply by \(h\),
  \begin{gather}
      hu_t + h\p{u^2}_x + h\p{uv}_y + h\p{uw}_z + \frac{1}{\rho} hp_x
        = \frac{1}{\rho} h\p{\sigma_{xz}}_z + g h e_x
  \end{gather}
  and transform from \(z\) to \(\zeta \),
  \begin{gather}
      \p{h\tilde{u}}_t + \p{h\tilde{u}^2}_x + \p{h\tilde{u}\tilde{v}}_y
        + \p{\tilde{u} \p{\tilde{w} - \p{\zeta h + h_b}_t
        - \p{\zeta h + h_b}_x \tilde{u} - \p{\zeta h + h_b}_y \tilde{v}}}_{\zeta} \\
        + \frac{1}{\rho}\p{h\tilde{p}}_x
        - \frac{1}{\rho}\p{\p{\zeta h + h_b}_x \tilde{p}}_{\zeta}
        = \frac{1}{\rho} \p{\tilde{\sigma}_{xz}}_{\zeta} + g h e_x.
  \end{gather}
  The hydrostatic pressure can be mapped onto \(\zeta \) as
  \begin{gather}
      \tilde{p}(t, x, y, \zeta) = h(t, x, y) \p{1 - \zeta} \rho g e_z,
  \end{gather}
  and then the pressure terms in the momentum equation can be simplified
  in the following way,
  \begin{gather}
    \frac{1}{\rho}\p{h\tilde{p}}_x
    - \frac{1}{\rho}\p{\p{\zeta h + h_b}_x \tilde{p}}_{\zeta}
    = \p{\frac{1}{2} h^2 g e_z}_x + \p{h_b}_x h g e_z.
  \end{gather}
  The resulting momentum balance equation is
  \begin{gather}
    \p{h\tilde{u}}_t + \p{h\tilde{u}^2 + \frac{1}{2} h^2 g e_z}_x
      + \p{h\tilde{u}\tilde{v}}_y
      + \p{\tilde{u} \p{\tilde{w} - \p{\zeta h + h_b}_t
      - \p{\zeta h + h_b}_x \tilde{u} - \p{\zeta h + h_b}_y \tilde{v}}}_{\zeta} \\
      = \frac{1}{\rho} \p{\tilde{\sigma}_{xz}}_{\zeta} + g h \p{e_x - \p{h_b}_x e_z}
  \end{gather}
  Next we consider the vertical coupling term, \(\omega \)
  \begin{gather}
    \omega = \tilde{w} - \p{\zeta h + h_b}_t - \p{\zeta h + h_b}_x \tilde{u}
    - \p{\zeta h + h_b}_y \tilde{v},
  \end{gather}
  Using the expression for vertical velocity in~\eqref{eq:vertical_velocity}, we find
  that,
  \begin{gather}
    \omega = -\p{h\dintt{0}{\zeta}{\tilde{u}}{\zeta'}}_x
        - \p{h\dintt{0}{\zeta}{\tilde{v}}{\zeta'}}_y
        - \zeta h_t,
  \end{gather}
  and then using~\eqref{eq:mass_conservation}, the vertical coupling becomes
  \begin{gather}
    \omega = -\p{h\dintt{0}{\zeta}{\tilde{u}_d}{\zeta'}}_x
        - \p{h\dintt{0}{\zeta}{\tilde{v}_d}{\zeta'}}_y,
  \end{gather}
  where
  \begin{gather}
      \tilde{u}_d = \tilde{u} - u_m \quad \tilde{v}_d = \tilde{v} - v_m.
  \end{gather}
  Thus the x momentum equation can be concisely written as
  \begin{gather}
    \p{h\tilde{u}}_t + \p{h\tilde{u}^2 + \frac{1}{2} h^2 g e_z}_x
      + \p{h\tilde{u}\tilde{v}}_y
      + \p{\tilde{u} \omega}_{\zeta}
      = \frac{1}{\rho} \p{\tilde{\sigma}_{xz}}_{\zeta} + g h \p{e_x - \p{h_b}_x e_z}
  \end{gather}
  Similarly the y-momentum equation can be written
  \begin{gather}
    \p{h\tilde{v}}_t + \p{h\tilde{u}\tilde{v}}_x
      + \p{h\tilde{v}^2 + \frac{1}{2} h^2 g e_z}_y
      + \p{\tilde{u} \omega}_{\zeta}
      = \frac{1}{\rho} \p{\tilde{\sigma}_{yz}}_{\zeta} + g h \p{e_y - \p{h_b}_y e_z}
  \end{gather}

\paragraph{Newtonian Closure}
  For the remainder of the derivation, a Newtonian flow is considered.
  In this case the deviatoric stress tensor terms are related to the velocities
  according to
  \begin{gather}
    \sigma_{xz} = \mu u_z, \qquad \sigma_{yz} = \mu v_z
  \end{gather}
  where \(\mu \) is the dynamic viscosity of the fluid.
  These terms can be mapped from the z domain to the zeta domain as follows,
  \begin{gather}
    \frac{1}{\rho} \tilde{\sigma}_{xz} = \frac{\nu}{h} \tilde{u}_{\zeta}, \qquad
    \frac{1}{\rho} \tilde{\sigma}_{yz} = \frac{\nu}{h} \tilde{v}_{\zeta},
  \end{gather}
  where \(\nu = \frac{\mu}{\rho}\) is the kinematic viscosity.
  Boundary conditions also need to be specified at the bottom topography and at the
  free surface.
  Stress free conditions will be assumed at the free surface,
  \begin{gather}
    \eval{u_z}{z = h_s} = 0 \qquad \eval{v_z}{z = h_s} = 0.
  \end{gather}
  At the bottom topography I will use a slip boundary condition of the form
  \begin{gather}
    \eval{\p{u - \lambda u_z}}{z = h_b} = 0 \qquad \eval{\p{v - \lambda v_z}}{z = h_b} = 0.
  \end{gather}
  Mapping these boundary conditions from z to zeta gives
  \begin{gather}
    \eval{\tilde{u}_{\zeta}}{\zeta = 1} = 0
    \qquad
    \eval{\tilde{v}_{\zeta}}{\zeta = 1} = 0 \\
    \eval{\tilde{u}_{\zeta}}{\zeta = 0} = \frac{h}{\lambda} \eval{\tilde{u}}{\zeta = 0}
    \qquad
    \eval{\tilde{v}_{\zeta}}{\zeta = 0} = \frac{h}{\lambda} \eval{\tilde{v}}{\zeta = 0}.
  \end{gather}

\paragraph{Moment Closure}
  Now that we have mapped the mass, momentum, and boundary equations onto the new
  reference frame of \(\zeta \in \br{0, 1}\), I will drop the tilde notation for
  readability.
  We can see the vertically resolved reference system, has the form,
  \begin{gather}
    h_t + \p{hu_m}_x + \p{hv_m}_y = 0 \\
    \p{hu}_t + \p{hu^2 + \frac{1}{2} h^2 g e_z}_x
    + \p{huv}_y
    + \p{u \omega}_{\zeta}
    = \frac{\nu}{h}\p{u_{\zeta}}_{\zeta} + g h \p{e_x - \p{h_b}_x e_z} \\
    \p{hv}_t + \p{huv}_x
    + \p{hv^2 + \frac{1}{2} h^2 g e_z}_y
    + \p{u \omega}_{\zeta}
    = \frac{\nu}{h} \p{v_{\zeta}}_{\zeta} + g h \p{e_y - \p{h_b}_y e_z}
  \end{gather}

  The final step in deriving this system of equations is depth-averaging the equations.
  This removes the explicit dependence on \(\zeta \), and creates evolution equations
  for all of the individual moments.
  This is done by using the moment expansion
  \begin{align}
    u(x, y, \zeta, t) &= u_m(x, y, t) + \sum{j=1}{\nmom}{\alpha_j(x, y, t) \phi_j(\zeta)} \\
    v(x, y, \zeta, t) &= v_m(x, y, t) + \sum{j=1}{\nmom}{\beta_j(x, y, t) \phi_j(\zeta)},
  \end{align}
  for the vertically resolved velocities.
  The velocity is approximated by this polynomial expansion, and the conservation of
  momentum equation can be used to derive time evolution equations for the mean
  velocities, \(u_m\) and \(v_m\), and the moment coefficients, \(\alpha_j\) and
  \(\beta_j\).
  The polynomials that are used in this work are Legendre polynomials orthogonal on
  \(\br{0, 1}\) and scaled so that \(\phi_j(0) = 1\).
  The first few polynomials are given by
  \begin{gather}
    \phi_0(\zeta) = 1, \qquad \phi_1(\zeta) = 1 - 2\zeta, \qquad \phi_2(\zeta) = 1 - 6\zeta + 6\zeta^2.
  \end{gather}
  Note that the mean velocities can be interpreted as \(\alpha_0\) and \(\beta_0\) if
  desired.
  By multiplying the momentum equations by the legendre polynomials and depth averaging,
  orthogonality will give equations for the mean velocities and moments.
  Note that the boundary conditions are weakly enforced by the conditions
  \begin{gather}
    -\frac{\nu}{h} \eval{u_{\zeta}}{\zeta = 0}{\zeta = 1} = \frac{\nu}{\lambda} \eval{u}{\zeta = 0}.
  \end{gather}
  Also the vertical coupling \(\omega \) disappears because it is zero at the bottom
  topography and at the free surface.
  The resulting system after depth averaging is
  \begin{equation}
    h_t + \p{hu}_x + \p{hv}_x = 0
  \end{equation}
  \begin{gather}
    \p{hu}_t + \p{hu^2 + h \sum{j = 1}{\nmom}{\frac{1}{2j + 1} \alpha_j^2}
    + \frac{1}{2} g e_z h^2}_x
    + \p{huv + h \sum{j = 1}{\nmom}{\frac{1}{2j + 1} \alpha_j \beta_j}}_y \nonumber \\
    = - \frac{\nu}{\lambda} \p{u + \sum{j = 1}{\nmom}{\alpha_j}}
    + h g e_x - h g e_z \p{h_b}_x
  \end{gather}
  \begin{gather}
    \p{hv}_t + \p{huv + h \sum{j = 1}{\nmom}{\frac{1}{2j + 1} \alpha_j \beta_j}}_x
    + \p{hv^2 + h \sum{j = 1}{\nmom}{\frac{1}{2j + 1} \beta_j^2}
    + \frac{1}{2} g e_z h^2}_y \nonumber \\
    = - \frac{\nu}{\lambda} \p{v + \sum{j = 1}{\nmom}{\beta_j}}
    + h g e_y - h g e_z \p{h_b}_y
  \end{gather}
  \begin{gather}
    \p{h\alpha_i}_t + \p{2hu\alpha_i
    + h\sum{j = 1}{\nmom}{\sum{k = 1}{\nmom}{A_{ijk} \alpha_j \alpha_k}}}_x
    + \p{hu\beta_i + hv\alpha_i
    + h \sum{j = 1}{\nmom}{\sum{k = 1}{\nmom}{A_{ijk} \alpha_j \beta_k}}}_y \nonumber \\
    = u_m D_i - \sum{j = 1}{\nmom}{D_j \sum{k = 1}{\nmom}{B_{ijk} \alpha_k}}
    - (2i + 1) \frac{\nu}{\lambda}\p{u
    + \sum{j = 1}{\nmom}{\p{1 + \frac{\lambda}{h}C_{ij}}\alpha_j}}
  \end{gather}
  \begin{gather}
    \p{h\beta_i}_t + \p{hu\beta_i + hv\alpha_i
    + h\sum{j = 1}{\nmom}{\sum{k = 1}{\nmom}{A_{ijk} \alpha_j \beta_k}}}_x
    + \p{2hv\beta_i
    + h\sum{j = 1}{\nmom}{\sum{k = 1}{\nmom}{A_{ijk} \beta_j \beta_k}}}_y \nonumber \\
    = v_m D_i - \sum{j = 1}{\nmom}{D_j \sum{k = 1}{\nmom}{B_{ijk} \beta_k}}
    - (2i + 1) \frac{\nu}{\lambda}\p{v
    + \sum{j = 1}{\nmom}{\p{1 + \frac{\lambda}{h}C_{ij}}\beta_j}}
  \end{gather}
  where
  \begin{gather}
    A_{ijk} = \p{2i + 1} \dintt{0}{1}{\phi_i \phi_j \phi_k}{\zeta} \\
    B_{ijk} = \p{2i + 1} \dintt{0}{1}{\phi_i' \p{\dintt{0}{\zeta}{\phi_j}{\hat{\zeta}}} \phi_k}{\zeta}\\
    C_{ij} = \dintt{0}{1}{\phi_i' \phi_j'}{\zeta} \\
    D_i = \p{h\alpha_i}_x + \p{h \beta_i}_y.
  \end{gather}

\subsection{Example Systems}

\subsubsection{One Dimensional Equations}
  In one dimension the generalized shallow water equations will have the following
  form,
  \begin{gather}
    \v{q}_t + \v{f}\p{\v{q}}_x = g(\v{q}) \v{q}_x + \v{p}.
  \end{gather}
  In this case the unknown \(\v{q}\) will have the form
  \begin{gather}
    \v{q} = \br{h, hu, h\alpha_1, h\alpha_2, \ldots}^T,
  \end{gather}
  where the number of components depends on the number of moments in the velocity
  profiles.

  The wavespeed of this system is given by the the eigenvalues of the matrix
  when the system is in quasilinear form.
  For this system we need to look at the eigenvalues of the matrix
  \(f'(\v{q}) - g(\v{q})\).
  If all of the eigenvalues are real, then this system is considered
  hyperbolic.
  Also if the gradient of the eigenvalue with respect to the conserved
  variables dotted with the corresponding eigenvector is always positive
  or always negative, then we say that the system is convex.
  That is if
  \[
    \grad \lambda_i . \v{v}_i < 0 \quad \text{ or } \quad
    \grad \lambda_i . \v{v}_i > 0,
  \]
  then the system is convex.
  If the dot product is zero, then we have a degenerate wave.

\paragraph{Zeroth Order}
  The Zeroth order system is equivalent to standard shallow water equations.
  The flux function and source function are given by,
  \begin{gather}
    \v{f}\p{\v{q}} =
    \begin{pmatrix}
      h u \\
      \frac{1}{2} e_{z} g h^{2} + h u^{2}
    \end{pmatrix} \\
    \v{p}\p{\v{q}} =
    \begin{pmatrix}
      0 \\
      -{(e_{z} \frac{\partial}{\partial x}h_{b} - e_{x})} g h - \frac{\nu u}{\lambda}
    \end{pmatrix}.
  \end{gather}
  Thy hyperbolicity of the system can be checked with the flux jacobian, which is
  equivalent to the quasilinear matrix, as this system is conservative.
  The flux jacobian and its eigenvalues, \(\lambda \), are shown below as
  \begin{gather}
    \v{f}'(\v{q}) = A =
    \begin{pmatrix}
      0 & 1 \\
      e_{z} g h - u^{2} & 2u
    \end{pmatrix}
    \intertext{Quasilinear Matrix Eigenvalues}
    \lambda = u \pm \sqrt{g h}
  \end{gather}
  Also this system is convex, which can be verified as follows,
  \begin{gather}
    \grad \lambda_i \cdot \v{v}_i = \pm \frac{3}{2} \frac{g}{\sqrt{gh}}.
  \end{gather}

\paragraph{First Order}
  For the first order system, one moment is introduced into the velocity.
  In this case the velocity can be described as a line in the vertical direction.
  With the addition of this moment the nonconservative matrix is introduced, and is
  shown below with the flux function and source function.
  \begin{gather}
    \v{f}\p{\v{q}} =
    \begin{pmatrix}
      h u \\
      \frac{1}{2} e_{z} g h^{2} + \frac{1}{3} \alpha_{1}^{2} h + h u^{2} \\
      2 \alpha_{1} h u \\
    \end{pmatrix} \\
    \M{g}\p{\v{q}} =
    \begin{pmatrix}
      0 & 0 & 0 \\
      0 & 0 & 0 \\
      0 & 0 & u
    \end{pmatrix} \\
    \v{p}\p{\v{q}} =
    \begin{pmatrix}
      0 \\
      -{(e_{z} \frac{\partial}{\partial x}h_{b} - e_{x})} g h - \frac{\nu {(\alpha_{1} + u)}}{\lambda} \\
      -\frac{3 {({(\frac{4 \lambda}{h} + 1)} \alpha_{1} + u)} \nu}{\lambda} \\
    \end{pmatrix}
  \end{gather}
  The hyperbolicity of this system can be checked by computing the flux jacobian
  and the quasilinear matrix, \(A = \v{f}'(\v{q}) - g(q)\), of this system.
  \begin{gather}
    \v{f}'(\v{q}) =
    \begin{pmatrix}
      0 & 1 & 0 \\
      e_{z} g h - \frac{1}{3} \alpha_{1}^{2} - u^{2} & 2 u & \frac{2}{3} \alpha_{1} \\
      -2 \alpha_{1} u & 2 \alpha_{1} & 2 u
    \end{pmatrix} \\
    A =
    \begin{pmatrix}
      0 & 1 & 0 \\
      e_{z} g h - \frac{1}{3} \alpha_{1}^{2} - u^{2} & 2 u & \frac{2}{3} \alpha_{1} \\
      -2 \alpha_{1} u & 2 \alpha_{1} & u
    \end{pmatrix}
  \end{gather}
  The eigenvalues of the quasilinear matrix can be computed as
  \begin{gather}
    \lambda = u \pm \sqrt{g h + \alpha_1^2}, u.
  \end{gather}
  Also the convexity of the system can be checked with the following dot products.
  \begin{gather}
    \grad \lambda_1 \cdot \v{v}_1 = -\frac{1}{2} \p{3 g h + 4 \alpha_1^2} \frac{\sqrt{g h + \alpha_1^2}}{g h^2 + h \alpha_1^2} \\
    \grad \lambda_2 \cdot \v{v}_2 = \frac{\sqrt{g h + \alpha_1^2}}{h} \\
    \grad \lambda_3 \cdot \v{v}_3 = -\frac{1}{2} \p{2 g h + \alpha_1^2} \frac{\sqrt{g h + \alpha_1^2}}{g h^2 + h \alpha_1^2}
  \end{gather}

\paragraph{Second Order}
  For the second order system there are two moments, and the velocity in the vertical
  direction is parabolic.
  The flux function, nonconservative matrix, and source function are given as
  \begin{gather}
    \v{f}\p{\v{q}} =
    \begin{pmatrix}
      h u \\
      \frac{1}{2} e_{z} g h^{2} + h u^{2} + \frac{1}{15} {(5 \alpha_{1}^{2} + 3 \alpha_{2}^{2})} h \\
      \frac{4}{5} \alpha_{1} \alpha_{2} h + 2 \alpha_{1} h u \\
      2 \alpha_{2} h u + \frac{2}{21} {(7 \alpha_{1}^{2} + 3 \alpha_{2}^{2})} h
    \end{pmatrix} \\
    g\p{\v{q}} =
    \begin{pmatrix}
      0 & 0 & 0 & 0 \\
      0 & 0 & 0 & 0 \\
      0 & 0 & -\frac{1}{5} \alpha_{2} + u & \frac{1}{5} \alpha_{1} \\
      0 & 0 & \alpha_{1} & \frac{1}{7} \alpha_{2} + u
    \end{pmatrix} \\
    \v{p}\p{\v{q}} =
    \begin{pmatrix}
      0 \\
      -{(e_{z} \frac{\partial}{\partial x}h_{b} - e_{x})} g h - \frac{\nu {(\alpha_{1} + \alpha_{2} + u)}}{\lambda} \\
      -\frac{3 {({(\frac{4 \lambda}{h} + 1)} \alpha_{1} + \alpha_{2} + u)} \nu}{\lambda} \\
      -\frac{5 {({(\frac{12 \lambda}{h} + 1)} \alpha_{2} + \alpha_{1} + u)} \nu}{\lambda}
    \end{pmatrix}
  \end{gather}
  The flux jacobian and quasilinear matrix can be computed to be
  \begin{gather}
    \v{f}'(\v{q}) =
    \begin{pmatrix}
      0 & 1 & 0 & 0 \\
      e_{z} g h - \frac{1}{3} \alpha_{1}^{2} - \frac{1}{5} \alpha_{2}^{2} - u^{2} & 2 u & \frac{2}{3} \alpha_{1} & \frac{2}{5} \alpha_{2} \\
      -\frac{4}{5} \alpha_{1} \alpha_{2} - 2 \alpha_{1} u & 2 \alpha_{1} & \frac{4}{5} \alpha_{2} + 2 u & \frac{4}{5} \alpha_{1} \\
      -\frac{2}{3} \alpha_{1}^{2} - \frac{2}{7} \alpha_{2}^{2} - 2 \alpha_{2} u & 2 \alpha_{2} & \frac{4}{3} \alpha_{1} & \frac{4}{7} \alpha_{2} + 2 u
    \end{pmatrix} \\
    A =
    \begin{pmatrix}
      0 & 1 & 0 & 0 \\
      e_{z} g h - \frac{1}{3} \alpha_{1}^{2} - \frac{1}{5} \alpha_{2}^{2} - u^{2} & 2 u & \frac{2}{3} \alpha_{1} & \frac{2}{5} \alpha_{2} \\
      -\frac{4}{5} \alpha_{1} \alpha_{2} - 2 \alpha_{1} u & 2 \alpha_{1} & \alpha_{2} + u & \frac{3}{5} \alpha_{1} \\
      -\frac{2}{3} \alpha_{1}^{2} - \frac{2}{7} \alpha_{2}^{2} - 2 \alpha_{2} u & 2 \alpha_{2} & \frac{1}{3} \alpha_{1} & \frac{3}{7} \alpha_{2} + u
    \end{pmatrix}
  \end{gather}
  This system is no longer globally hyperbolic.
  The eigenvalues of the system can be computed numerically, but there are values of
  \(\alpha_1\) and \(\alpha_2\), which result in a non hyperbolic system.
  Intuitively this occurs when the values of \(\alpha_1\) and \(\alpha_2\) would
  physically result in a vortex, turbulent behavior, or some other physical behavior,
  that can't be captured or described by the system.
  In this case the problem becomes ill-posed.

\paragraph{Third Order}
  For three moments, the velocity in the vertical direction is cubic, and the flux
  function, nonconservative matrix, and source function are shown below,
  \begin{gather}
    \v{f}\p{\v{q}} =
    \begin{pmatrix}
      h u \\
      \frac{1}{2} e_{z} g h^{2} + h u^{2} + \frac{1}{105} {(35 \alpha_{1}^{2} + 21 \alpha_{2}^{2} + 15 \alpha_{3}^{2})} h \\
      2 \alpha_{1} h u + \frac{2}{35} {(14 \alpha_{1} \alpha_{2} + 9 \alpha_{2} \alpha_{3})} h \\
      2 \alpha_{2} h u + \frac{2}{21} {(7 \alpha_{1}^{2} + 3 \alpha_{2}^{2} + 9 \alpha_{1} \alpha_{3} + 2 \alpha_{3}^{2})} h \\
      2 \alpha_{3} h u + \frac{2}{15} {(9 \alpha_{1} \alpha_{2} + 4 \alpha_{2} \alpha_{3})} h
    \end{pmatrix} \\
    g\p{\v{q}} =
    \begin{pmatrix}
      0 & 0 & 0 & 0 & 0 \\
      0 & 0 & 0 & 0 & 0 \\
      0 & 0 & -\frac{1}{5} \alpha_{2} + u & \frac{1}{5} \alpha_{1} - \frac{3}{35} \alpha_{3} & \frac{3}{35} \alpha_{2} \\
      0 & 0 & \alpha_{1} - \frac{3}{7} \alpha_{3} & \frac{1}{7} \alpha_{2} + u & \frac{2}{7} \alpha_{1} + \frac{1}{21} \alpha_{3} \\
      0 & 0 & \frac{6}{5} \alpha_{2} & \frac{4}{5} \alpha_{1} + \frac{2}{15} \alpha_{3} & \frac{1}{5} \alpha_{2} + u
    \end{pmatrix} \\
    \v{p}\p{\v{q}} =
    \begin{pmatrix}
      0 \\
      -{(e_{z} \frac{\partial}{\partial x}h_{b} - e_{x})} g h - \frac{\nu {(\alpha_{1} + \alpha_{2} + \alpha_{3} + u)}}{\lambda} \\
      -\frac{3 {({(\frac{4 \lambda}{h} + 1)} \alpha_{1} + {(\frac{4 \lambda}{h} + 1)} \alpha_{3} + \alpha_{2} + u)} \nu}{\lambda} \\
      -\frac{5 {({(\frac{12 \lambda}{h} + 1)} \alpha_{2} + \alpha_{1} + \alpha_{3} + u)} \nu}{\lambda} \\
      -\frac{7 {({(\frac{4 \lambda}{h} + 1)} \alpha_{1} + {(\frac{24 \lambda}{h} + 1)} \alpha_{3} + \alpha_{2} + u)} \nu}{\lambda}
    \end{pmatrix}
  \end{gather}
  The flux jacobian and quasilinear matrix can be computed as
  \begin{gather}
    \v{f}'(\v{q}) =
    \begin{pmatrix}
      0 & 1 & 0 & 0 & 0 \\
      e_{z} g h - \frac{1}{3} \alpha_{1}^{2} - \frac{1}{5} \alpha_{2}^{2} - \frac{1}{7} \alpha_{3}^{2} - u^{2} & 2 u & \frac{2}{3} \alpha_{1} & \frac{2}{5} \alpha_{2} & \frac{2}{7} \alpha_{3} \\
      -\frac{4}{5} \alpha_{1} \alpha_{2} - \frac{18}{35} \alpha_{2} \alpha_{3} - 2 \alpha_{1} u & 2 \alpha_{1} & \frac{4}{5} \alpha_{2} + 2 u & \frac{4}{5} \alpha_{1} + \frac{18}{35} \alpha_{3} & \frac{18}{35} \alpha_{2} \\
      -\frac{2}{3} \alpha_{1}^{2} - \frac{2}{7} \alpha_{2}^{2} - \frac{6}{7} \alpha_{1} \alpha_{3} - \frac{4}{21} \alpha_{3}^{2} - 2 \alpha_{2} u & 2 \alpha_{2} & \frac{4}{3} \alpha_{1} + \frac{6}{7} \alpha_{3} & \frac{4}{7} \alpha_{2} + 2 u & \frac{6}{7} \alpha_{1} + \frac{8}{21} \alpha_{3} \\
      -\frac{6}{5} \alpha_{1} \alpha_{2} - \frac{8}{15} \alpha_{2} \alpha_{3} - 2 \alpha_{3} u & 2 \alpha_{3} & \frac{6}{5} \alpha_{2} & \frac{6}{5} \alpha_{1} + \frac{8}{15} \alpha_{3} & \frac{8}{15} \alpha_{2} + 2 u
    \end{pmatrix} \\
    A =
    \begin{pmatrix}
      0 & 1 & 0 & 0 & 0 \\
      e_{z} g h - \frac{1}{3} \alpha_{1}^{2} - \frac{1}{5} \alpha_{2}^{2} - \frac{1}{7} \alpha_{3}^{2} - u^{2} & 2 u & \frac{2}{3} \alpha_{1} & \frac{2}{5} \alpha_{2} & \frac{2}{7} \alpha_{3} \\
      -\frac{4}{5} \alpha_{1} \alpha_{2} - \frac{18}{35} \alpha_{2} \alpha_{3} - 2 \alpha_{1} u & 2 \alpha_{1} & \alpha_{2} + u & \frac{3}{5} \alpha_{1} + \frac{3}{5} \alpha_{3} & \frac{3}{7} \alpha_{2} \\
      -\frac{2}{3} \alpha_{1}^{2} - \frac{2}{7} \alpha_{2}^{2} - \frac{6}{7} \alpha_{1} \alpha_{3} - \frac{4}{21} \alpha_{3}^{2} - 2 \alpha_{2} u & 2 \alpha_{2} & \frac{1}{3} \alpha_{1} + \frac{9}{7} \alpha_{3} & \frac{3}{7} \alpha_{2} + u & \frac{4}{7} \alpha_{1} + \frac{1}{3} \alpha_{3} \\
      -\frac{6}{5} \alpha_{1} \alpha_{2} - \frac{8}{15} \alpha_{2} \alpha_{3} - 2 \alpha_{3} u & 2 \alpha_{3} & 0 & \frac{2}{5} \alpha_{1} + \frac{2}{5} \alpha_{3} & \frac{1}{3} \alpha_{2} + u
    \end{pmatrix}
  \end{gather}
  The eigenvalues of the quasilinear matrix need to be computed numerically, and
  as in the second order case they are no longer always real.
  There are cases where the system is no longer hyperbolic.

\subsubsection{2D Equations}
  In two dimensions the generalized shallow water equations will have the following
  form,
  \begin{gather}
    \v{q}_t + \v{f}_1\p{\v{q}}_x + \v{f}_2\p{\v{q}}_y
    = g_1(\v{q}) \v{q}_x + g_2(\v{q}) \v{q}_y + \v{p}.
  \end{gather}
  In this case the unknown \(\v{q}\) will have the form
  \begin{gather}
    \v{q} = \br{h, hu, hv, h\alpha_1, h\beta_1, h\alpha_2, h \beta_2, \ldots}^T,
  \end{gather}
  where the number of components depends on the number of moments in the velocity
  profiles.

  The wavespeeds of the two dimensional system in the direction
  \(\v{n} = \br{n_1, n_2}\), are given by the eigenvalues of the matrix
  \begin{gather}
    n_1 \p{\v{f}_1'(\v{q}) - g_1(\v{q})} + n_2 \p{\v{f}_2'(\v{q}) - g_2(\v{q})}.
  \end{gather}
  If this matrix is diagonalizable with real eigenvalues for all directions
  \(\v{n}\), then this system is considered hyperbolic.

\paragraph{Zeroth Order}
  The zeroth order system is exactly the standard shallow water equations, where only
  the average velocity is considered.
  This velocity profiles in this system only consider the constant moment.
  In this case the nonconservative product disappears and the equation has the
  following form.
  \begin{gather}
    \v{q}_t + \v{f}_1\p{\v{q}}_x + \v{f}_2\p{\v{q}}_y = \v{p}.
  \end{gather}
  where
  \begin{gather}
    \v{f}_1(\v{q}) =
    \begin{pmatrix}
      h u \\
      \frac{1}{2} e_{z} g h^{2} + h u^{2} \\
      h u v
    \end{pmatrix}, \qquad
    \v{f}_2(\v{q}) =
    \begin{pmatrix}
      h v \\
      h u v \\
      \frac{1}{2} e_{z} g h^{2} + h v^{2}
    \end{pmatrix} \\
    \v{p} =
    \begin{pmatrix}
      0 \\
      -\p{e_{z} \pda{h_b}{x} - e_{x}} g h - \frac{\nu}{\lambda} u \\
      -\p{e_{z} \pda{h_b}{y} - e_{y}} g h - \frac{\nu}{\lambda} v
    \end{pmatrix}
  \end{gather}
  The flux jacobians and quasilinear matrices,
  \(A = \v{f}_1'(\v{q}) - g_1(\v{q}), B = \v{f}_2'(\v{q}) - g_2(\v{q})\), are given
  below,
  \begin{gather}
    \v{f}_1'(\v{q}) =
    \begin{pmatrix}
      0 & 1 & 0 \\
      e_{z} g h - u^{2} & 2 u & 0 \\
      -u v & v & u
    \end{pmatrix}, \qquad
    \v{f}_2'(\v{q}) =
    \begin{pmatrix}
      0 & 0 & 1 \\
      -u v & v & u \\
      e_{z} g h - v^{2} & 0 & 2 v
    \end{pmatrix} \\
    A =
    \begin{pmatrix}
      0 & 1 & 0 \\
      e_{z} g h - u^{2} & 2 u & 0 \\
      -u v & v & u
    \end{pmatrix}, \qquad
    B =
    \begin{pmatrix}
      0 & 0 & 1 \\
      -u v & v & u \\
      e_{z} g h - v^{2} & 0 & 2 v
    \end{pmatrix}
  \end{gather}
  The wavespeed of this system in the direction of \(\v{n} = \br{n_1, n_2}\) is given
  by the following eigenvalues,
  \begin{gather}
    \lambda_{1,2} = n_1 u + n_2 v \pm \sqrt{e_z g h \p{n_0^2 + n_1^2}} \\
    \lambda_3 = n_1 u + n_2 v
  \end{gather}

\paragraph{First Order}
  The first order system describes the velocity in the vertical direction as a line
  and adds two additional moments, one in the x-direction and one in the y-direction.
  The flux function, nonconservative matrices, and source term are given below.
  \begin{gather}
    \v{f}_1(\v{q}) =
    \begin{pmatrix}
      h u \\
      \frac{1}{2} e_{z} g h^{2} + \frac{1}{3} \alpha_{1}^{2} h + h u^{2} \\
      \frac{1}{3} \alpha_{1} \beta_{1} h + h u v \\
      2 \alpha_{1} h u \\
      \beta_{1} h u + \alpha_{1} h v
    \end{pmatrix}, \quad
    \v{f}_2(\v{q}) =
    \begin{pmatrix}
      h v \\
      \frac{1}{3} \alpha_{1} \beta_{1} h + h u v \\
      \frac{1}{2} e_{z} g h^{2} + \frac{1}{3} \beta_{1}^{2} h + h v^{2} \\
      \beta_{1} h u + \alpha_{1} h v \\
      2  \beta_{1} h v
    \end{pmatrix} \\
    g_1(\v{q}) =
    \begin{pmatrix}
      0 & 0 & 0 & 0 & 0 \\
      0 & 0 & 0 & 0 & 0 \\
      0 & 0 & 0 & 0 & 0 \\
      0 & 0 & 0 & u & 0 \\
      0 & 0 & 0 & v & 0
    \end{pmatrix}, \quad
    g_2(\v{q}) =
    \begin{pmatrix}
      0 & 0 & 0 & 0 & 0 \\
      0 & 0 & 0 & 0 & 0 \\
      0 & 0 & 0 & 0 & 0 \\
      0 & 0 & 0 & 0 & u \\
      0 & 0 & 0 & 0 & v
    \end{pmatrix} \\
    \v{p} =
    \begin{pmatrix}
      0 \\
      -{(e_{z} \frac{\partial}{\partial x}h_{b} - e_{x})} g h - \frac{\nu {(\alpha_{1} + u)}}{\lambda} \\
      -{(e_{z} \frac{\partial}{\partial y}h_{b} - e_{y})} g h - \frac{\nu {(\beta_{1} + v)}}{\lambda} \\
      -\frac{3 {({(\frac{4 \lambda}{h} + 1)} \alpha_{1} + u)} \nu}{\lambda} \\
      -\frac{3 {({(\frac{4 \lambda}{h} + 1)} \beta_{1} + v)} \nu}{\lambda}
    \end{pmatrix}
  \end{gather}
  The flux jacobians and quasilinear matrices for this system are
  \begin{gather}
    \v{f}_1'(\v{q}) =
    \begin{pmatrix}
      0 & 1 & 0 & 0 & 0 \\
      e_{z} g h - \frac{1}{3} \alpha_{1}^{2} - u^{2} & 2 u & 0 & \frac{2}{3} \alpha_{1} & 0 \\
      -\frac{1}{3} \alpha_{1} \beta_{1} - u v & v & u & \frac{1}{3} \beta_{1} & \frac{1}{3} \alpha_{1} \\
      -2 \alpha_{1} u & 2 \alpha_{1} & 0 & 2 u & 0 \\
      -\beta_{1} u - \alpha_{1} v & \beta_{1} & \alpha_{1} & v & u
    \end{pmatrix} \\
    \v{f}_2'(\v{q}) =
    \begin{pmatrix}
      0 & 0 & 1 & 0 & 0 \\
      -\frac{1}{3} \alpha_{1} \beta_{1} - u v & v & u & \frac{1}{3} \beta_{1} & \frac{1}{3} \alpha_{1} \\
      e_{z} g h - \frac{1}{3} \beta_{1}^{2} - v^{2} & 0 & 2 v & 0 & \frac{2}{3} \beta_{1} \\
      -\beta_{1} u - \alpha_{1} v & \beta_{1} & \alpha_{1} & v & u \\
      -2 \beta_{1} v & 0 & 2 \beta_{1} & 0 & 2 v
    \end{pmatrix} \\
    A =
    \begin{pmatrix}
      0 & 1 & 0 & 0 & 0 \\
      e_{z} g h - \frac{1}{3} \alpha_{1}^{2} - u^{2} & 2 u & 0 & \frac{2}{3} \alpha_{1} & 0 \\
      -\frac{1}{3} \alpha_{1} \beta_{1} - u v & v & u & \frac{1}{3} \beta_{1} & \frac{1}{3} \alpha_{1} \\
      -2 \alpha_{1} u & 2 \alpha_{1} & 0 & u & 0 \\
      -\beta_{1} u - \alpha_{1} v & \beta_{1} & \alpha_{1} & 0 & u
    \end{pmatrix} \\
    B =
    \begin{pmatrix}
      0 & 0 & 1 & 0 & 0 \\
      -\frac{1}{3} \alpha_{1} \beta_{1} - u v & v & u & \frac{1}{3} \beta_{1} & \frac{1}{3} \alpha_{1} \\
      e_{z} g h - \frac{1}{3} \beta_{1}^{2} - v^{2} & 0 & 2 v & 0 & \frac{2}{3} \beta_{1} \\
      -\beta_{1} u - \alpha_{1} v & \beta_{1} & \alpha_{1} & v & 0 \\
      -2 \beta_{1} v & 0 & 2 \beta_{1} & 0 & v
    \end{pmatrix}.
  \end{gather}
  The wavespeed of the system in the direction of \(\v{n} = \br{n_1, n_2}\) can be
  computed as the following eigenvalues of \(n_1 A + n_2 B\),
  \begin{gather}
    \lambda_{1,2} = n_0 u + n_1 v \pm \sqrt{e_z g h \p{n_0^2 + n_1^2} + \p{\alpha_1 n_0 + \beta_1 n_1}^2} \\
    \lambda_3 = n_0 u + n_1 v \\
    \lambda_{4,5} = n_0 u + n_1 v \pm \frac{\sqrt{3}}{3} \p{\alpha_1 n_0 + \beta_1 n_1}
  \end{gather}

\paragraph{Second Order}
  The second order system has seven equations and four additional moments from the
  standard shallow water model.
  The velocity in the vertical direction can be described as parabolas,
  The flux functions, nonconservative matrices, and source function are shown below,
  \begin{gather}
    \v{f}_1(\v{q}) =
    \begin{pmatrix}
      h u \\
      \frac{1}{2} e_{z} g h^{2} + h u^{2} + \frac{1}{15} {(5 \alpha_{1}^{2} + 3 \alpha_{2}^{2})} h \\
      h u v + \frac{1}{15} {(5 \alpha_{1} \beta_{1} + 3 \alpha_{2} \beta_{2})} h \\
      \frac{4}{5} \alpha_{1} \alpha_{2} h + 2 \alpha_{1} h u \\
      \beta_{1} h u + \alpha_{1} h v + \frac{2}{5} {(\alpha_{2} \beta_{1} + \alpha_{1} \beta_{2})} h \\
      2 \alpha_{2} h u + \frac{2}{21} {(7 \alpha_{1}^{2} + 3 \alpha_{2}^{2})} h \\
      \beta_{2} h u + \alpha_{2} h v + \frac{2}{21} {(7 \alpha_{1} \beta_{1} + 3 \alpha_{2} \beta_{2})} h
    \end{pmatrix}, \quad
    \v{f}_2(\v{q}) =
    \begin{pmatrix}
      h v \\
      h u v + \frac{1}{15} {(5 \alpha_{1} \beta_{1} + 3 \alpha_{2} \beta_{2})} h \\
      \frac{1}{2} e_{z} g h^{2} + h v^{2} + \frac{1}{15} {(5 \beta_{1}^{2} + 3 \beta_{2}^{2})} h \\
      \beta_{1} h u + \alpha_{1} h v + \frac{2}{5} {(\alpha_{1} \beta_{1} + \alpha_{2} \beta_{2})} h \\
      \frac{4}{5} \beta_{1} \beta_{2} h + 2 \beta_{1} h v \\
      \beta_{2} h u + \alpha_{2} h v + \frac{2}{21} {(7 \alpha_{1} \beta_{1} + 3 \alpha_{2} \beta_{2})} h \\
      2 \beta_{2} h v + \frac{2}{21} {(7 \beta_{1}^{2} + 3 \beta_{2}^{2})} h
    \end{pmatrix} \\
    g_1(\v{q}) =
    \begin{pmatrix}
      0 & 0 & 0 & 0 & 0 & 0 & 0 \\
      0 & 0 & 0 & 0 & 0 & 0 & 0 \\
      0 & 0 & 0 & 0 & 0 & 0 & 0 \\
      0 & 0 & 0 & -\frac{1}{5} \alpha_{2} + u & 0 & \frac{1}{5} \alpha_{1} & 0 \\
      0 & 0 & 0 & -\frac{1}{5} \beta_{2} + v & 0 & \frac{1}{5} \beta_{1} & 0 \\
      0 & 0 & 0 & \alpha_{1} & 0 & \frac{1}{7} \alpha_{2} + u & 0 \\
      0 & 0 & 0 & \beta_{1} & 0 & \frac{1}{7} \beta_{2} + v & 0
    \end{pmatrix}, \quad
    g_2(\v{q}) =
    \begin{pmatrix}
      0 & 0 & 0 & 0 & 0 & 0 & 0 \\
      0 & 0 & 0 & 0 & 0 & 0 & 0 \\
      0 & 0 & 0 & 0 & 0 & 0 & 0 \\
      0 & 0 & 0 & 0 & -\frac{1}{5} \alpha_{2} + u & 0 & \frac{1}{5} \alpha_{1} \\
      0 & 0 & 0 & 0 & -\frac{1}{5} \beta_{2} + v & 0 & \frac{1}{5} \beta_{1} \\
      0 & 0 & 0 & 0 & \alpha_{1} & 0 & \frac{1}{7} \alpha_{2} + u \\
      0 & 0 & 0 & 0 & \beta_{1} & 0 & \frac{1}{7} \beta_{2} + v
    \end{pmatrix} \\
    \v{p} =
    \begin{pmatrix}
      0 \\
      -{(e_{z} \frac{\partial}{\partial x}h_{b} - e_{x})} g h - \frac{\nu {(\alpha_{1} + \alpha_{2} + u)}}{\lambda} \\
      -{(e_{z} \frac{\partial}{\partial y}h_{b} - e_{y})} g h - \frac{\nu {(\beta_{1} + \beta_{2} + v)}}{\lambda} \\
      -\frac{3 {({(\frac{4 \lambda}{h} + 1)} \alpha_{1} + \alpha_{2} + u)} \nu}{\lambda} \\
      -\frac{3 {({(\frac{4 \lambda}{h} + 1)} \beta_{1} + \beta_{2} + v)} \nu}{\lambda} \\
      -\frac{5 {({(\frac{12 \lambda}{h} + 1)} \alpha_{2} + \alpha_{1} + u)} \nu}{\lambda} \\
      -\frac{5 {({(\frac{12 \lambda}{h} + 1)} \beta_{2} + \beta_{1} + v)} \nu}{\lambda}
    \end{pmatrix}
  \end{gather}
  The flux jacobians are computed below as,
  \begin{gather}
    \v{f}_1'(\v{q}) =
    \begin{pmatrix}
      0 & 1 & 0 & 0 & 0 & 0 & 0 \\
      e_{z} g h - \frac{1}{3} \alpha_{1}^{2} - \frac{1}{5} \alpha_{2}^{2} - u^{2} & 2 u & 0 & \frac{2}{3} \alpha_{1} & 0 & \frac{2}{5} \alpha_{2} & 0 \\
      -\frac{1}{3} \alpha_{1} \beta_{1} - \frac{1}{5} \alpha_{2} \beta_{2} - u v & v & u & \frac{1}{3} \beta_{1} & \frac{1}{3} \alpha_{1} & \frac{1}{5} \beta_{2} & \frac{1}{5} \alpha_{2} \\
      -\frac{4}{5} \alpha_{1} \alpha_{2} - 2 \alpha_{1} u & 2 \alpha_{1} & 0 & \frac{4}{5} \alpha_{2} + 2 u & 0 & \frac{4}{5} \alpha_{1} & 0 \\
      -\frac{2}{5} \alpha_{2} \beta_{1} - \frac{2}{5} \alpha_{1} \beta_{2} - \beta_{1} u - \alpha_{1} v & \beta_{1} & \alpha_{1} & \frac{2}{5} \beta_{2} + v & \frac{2}{5} \alpha_{2} + u & \frac{2}{5} \beta_{1} & \frac{2}{5} \alpha_{1} \\
      -\frac{2}{3} \alpha_{1}^{2} - \frac{2}{7} \alpha_{2}^{2} - 2 \alpha_{2} u & 2 \alpha_{2} & 0 & \frac{4}{3} \alpha_{1} & 0 & \frac{4}{7} \alpha_{2} + 2 u & 0 \\
      -\frac{2}{3} \alpha_{1} \beta_{1} - \frac{2}{7} \alpha_{2} \beta_{2} - \beta_{2} u - \alpha_{2} v & \beta_{2} & \alpha_{2} & \frac{2}{3} \beta_{1} & \frac{2}{3} \alpha_{1} & \frac{2}{7} \beta_{2} + v & \frac{2}{7} \alpha_{2} + u
    \end{pmatrix} \\
    \v{f}_2'(\v{q}) =
    \begin{pmatrix}
      0 & 0 & 1 & 0 & 0 & 0 & 0 \\
      -\frac{1}{3} \alpha_{1} \beta_{1} - \frac{1}{5} \alpha_{2} \beta_{2} - u v & v & u & \frac{1}{3} \beta_{1} & \frac{1}{3} \alpha_{1} & \frac{1}{5} \beta_{2} & \frac{1}{5} \alpha_{2} \\
      e_{z} g h - \frac{1}{3} \beta_{1}^{2} - \frac{1}{5} \beta_{2}^{2} - v^{2} & 0 & 2 v & 0 & \frac{2}{3} \beta_{1} & 0 & \frac{2}{5} \beta_{2} \\
      -\frac{2}{5} \alpha_{1} \beta_{1} - \frac{2}{5} \alpha_{2} \beta_{2} - \beta_{1} u - \alpha_{1} v & \beta_{1} & \alpha_{1} & \frac{2}{5} \beta_{1} + v & \frac{2}{5} \alpha_{1} + u & \frac{2}{5} \beta_{2} & \frac{2}{5} \alpha_{2} \\
      -\frac{4}{5} \beta_{1} \beta_{2} - 2 \beta_{1} v & 0 & 2 \beta_{1} & 0 & \frac{4}{5} \beta_{2} + 2 v & 0 & \frac{4}{5} \beta_{1} \\
      -\frac{2}{3} \alpha_{1} \beta_{1} - \frac{2}{7} \alpha_{2} \beta_{2} - \beta_{2} u - \alpha_{2} v & \beta_{2} & \alpha_{2} & \frac{2}{3} \beta_{1} & \frac{2}{3} \alpha_{1} & \frac{2}{7} \beta_{2} + v & \frac{2}{7} \alpha_{2} + u \\
      -\frac{2}{3} \beta_{1}^{2} - \frac{2}{7} \beta_{2}^{2} - 2 \beta_{2} v & 0 & 2 \beta_{2} & 0 & \frac{4}{3} \beta_{1} & 0 & \frac{4}{7} \beta_{2} + 2 v
    \end{pmatrix}.
  \end{gather}
  The quasilinear matrices are
  \begin{gather}
    A =
    \begin{pmatrix}
      0 & 1 & 0 & 0 & 0 & 0 & 0 \\
      e_{z} g h - \frac{1}{3} \alpha_{1}^{2} - \frac{1}{5} \alpha_{2}^{2} - u^{2} & 2 u & 0 & \frac{2}{3} \alpha_{1} & 0 & \frac{2}{5} \alpha_{2} & 0 \\
      -\frac{1}{3} \alpha_{1} \beta_{1} - \frac{1}{5} \alpha_{2} \beta_{2} - u v & v & u & \frac{1}{3} \beta_{1} & \frac{1}{3} \alpha_{1} & \frac{1}{5} \beta_{2} & \frac{1}{5} \alpha_{2} \\
      -\frac{4}{5} \alpha_{1} \alpha_{2} - 2 \alpha_{1} u & 2 \alpha_{1} & 0 & \alpha_{2} + u & 0 & \frac{3}{5} \alpha_{1} & 0 \\
      -\frac{2}{5} \alpha_{2} \beta_{1} - \frac{2}{5} \alpha_{1} \beta_{2} - \beta_{1} u - \alpha_{1} v & \beta_{1} & \alpha_{1} & \frac{3}{5} \beta_{2} & \frac{2}{5} \alpha_{2} + u & \frac{1}{5} \beta_{1} & \frac{2}{5} \alpha_{1} \\
      -\frac{2}{3} \alpha_{1}^{2} - \frac{2}{7} \alpha_{2}^{2} - 2 \alpha_{2} u & 2 \alpha_{2} & 0 & \frac{1}{3} \alpha_{1} & 0 & \frac{3}{7} \alpha_{2} + u & 0 \\
      -\frac{2}{3} \alpha_{1} \beta_{1} - \frac{2}{7} \alpha_{2} \beta_{2} - \beta_{2} u - \alpha_{2} v & \beta_{2} & \alpha_{2} & -\frac{1}{3} \beta_{1} & \frac{2}{3} \alpha_{1} & \frac{1}{7} \beta_{2} & \frac{2}{7} \alpha_{2} + u
    \end{pmatrix} \\
    B =
    \begin{pmatrix}
      0 & 0 & 1 & 0 & 0 & 0 & 0 \\
      -\frac{1}{3} \alpha_{1} \beta_{1} - \frac{1}{5} \alpha_{2} \beta_{2} - u v & v & u & \frac{1}{3} \beta_{1} & \frac{1}{3} \alpha_{1} & \frac{1}{5} \beta_{2} & \frac{1}{5} \alpha_{2} \\
      e_{z} g h - \frac{1}{3} \beta_{1}^{2} - \frac{1}{5} \beta_{2}^{2} - v^{2} & 0 & 2 v & 0 & \frac{2}{3} \beta_{1} & 0 & \frac{2}{5} \beta_{2} \\
      -\frac{2}{5} \alpha_{1} \beta_{1} - \frac{2}{5} \alpha_{2} \beta_{2} - \beta_{1} u - \alpha_{1} v & \beta_{1} & \alpha_{1} & \frac{2}{5} \beta_{1} + v & \frac{2}{5} \alpha_{1} + \frac{1}{5} \alpha_{2} & \frac{2}{5} \beta_{2} & -\frac{1}{5} \alpha_{1} + \frac{2}{5} \alpha_{2} \\
      -\frac{4}{5} \beta_{1} \beta_{2} - 2 \beta_{1} v & 0 & 2 \beta_{1} & 0 & \beta_{2} + v & 0 & \frac{3}{5} \beta_{1} \\
      -\frac{2}{3} \alpha_{1} \beta_{1} - \frac{2}{7} \alpha_{2} \beta_{2} - \beta_{2} u - \alpha_{2} v & \beta_{2} & \alpha_{2} & \frac{2}{3} \beta_{1} & -\frac{1}{3} \alpha_{1} & \frac{2}{7} \beta_{2} + v & \frac{1}{7} \alpha_{2} \\
      -\frac{2}{3} \beta_{1}^{2} - \frac{2}{7} \beta_{2}^{2} - 2 \beta_{2} v & 0 & 2 \beta_{2} & 0 & \frac{1}{3} \beta_{1} & 0 & \frac{3}{7} \beta_{2} + v
    \end{pmatrix}.
  \end{gather}
  As in the one dimensional case the second order wavespeeds need to be computed
  numerically and the system is no longer globally hyperbolic.

% \section{Shallow Water Linearized Moment Equations}


% \section{Spherical Shallow Water Moment Equations}
%   We would like to construct the shallow water moment models on the sphere.
%   Consider a thin film of fluid on a sphere with a
%   In this work, we will use the standard mathematical notation for spherical
%   coordinates, that is
%   \begin{gather}
%     r = \sqrt{x^2 + y^2 + z^2} \quad
%     \theta = \arctan{y/x} \quad
%     \phi = \arctan{\frac{\sqrt{x^2 + y^2}}{z}}
%   \end{gather}
%   or equivalently
%   \begin{gather}
%     x = r \cos{\theta} \sin{\phi} \quad
%     y = r \sin{\theta} \sin{\phi} \quad
%     z = r \cos{\phi}.
%   \end{gather}

% \subsection{Derivation}
%   We begin by considering the incompressible Navier-Stokes equations in spherical
%   coordinates,
%   \begin{gather}
%     \frac{1}{r^2}\pda{r^2 u_r}{r} + \frac{1}{r \sin{\phi}} \pd{u_{\theta}}{\theta}
%       + \frac{1}{r \sin{\phi}} \pda{\sin{\phi} u_{\phi}}{\phi} = 0 \\
%     \pd{u_r}{t} + \pda{u_r^2}{r} + \frac{1}{r \sin{\phi}} \pda{u_r u_\theta}{\theta}
%       + \frac{1}{r} \pda{u_r u_{\phi}}{\phi}
%       + \frac{2 u_r^2 + u_r u_{\phi} \cot{\phi} - u_{\theta}^2 - u_{\phi}^2}{r}
%       = - \frac{1}{\rho_0} \pd{p}{r} + g_r \\
%     \pd{u_{\theta}}{t} + \pda{u_r u_{\theta}}{r}
%       + \frac{1}{r \sin{\phi}} \pda{u_{\theta}^2}{\theta}
%       + \frac{1}{r} \pda{u_{\theta} u_{\phi}}{\phi}
%       + 3\frac{u_r u_{\theta}}{r}
%       + 2\frac{u_{\theta} u_{\phi} \cot{\phi}}{r}
%       = - \frac{1}{\rho_0 r \sin{\phi}} \pd{p}{\theta} + g_{\theta} \\
%     \pd{u_{\phi}}{t} + \pda{u_r u_{\phi}}{r}
%       + \frac{1}{r \sin{\phi}} \pda{u_{\theta} u_{\phi}}{\theta}
%       + \frac{1}{r} \pda{u_{\phi}^2}{\phi}
%       + 3 \frac{u_r u_{\phi}}{r}
%       + \frac{\p{u_{\phi}^2 - u_{\theta}^2} \cot{\phi}}{r}
%       = - \frac{1}{\rho_0 r} \pd{p}{\phi} + g_{\phi}
%   \end{gather}
%   where \(u_r\) is the fluid velocity in the radial direction, \(u_{\theta}\) and
%   \(u_{\phi}\) are the fluid velocities in the azimuthal and polar directions,
%   \(\rho_0\) is the constant fluid density, \(p\) is the pressure, \(g\) is the gravity
%   constant, and \(e_r, e_{\theta}, e_{\phi}\) are the components of the direction of
%   gravitational force.

%   We will assume kinematic boundary conditions at both the bottom topography and the
%   free surface.
%   The kinematic boundary condition states that the material derivative at the surface
%   is zero.
%   Thus in spherical coordinates the kinematic boundary conditions are
%   \begin{gather}
%     \pd{h_s}{t} + \frac{u_{\theta}(t, \theta, \phi, h_s)}{r \sin{\phi}} \pd{h_s}{\theta}
%       + \frac{u_{\phi}(t, \theta, \phi, h_s)}{r} \pd{h_s}{\phi}
%       = u_r(t, \theta, \phi, h_s) \\
%     \pd{h_b}{t} + \frac{u_{\theta}(t, \theta, \phi, h_b)}{r \sin{\phi}} \pd{h_b}{\theta}
%       + \frac{u_{\phi}(t, \theta, \phi, h_b)}{r} \pd{h_b}{\phi}
%       = u_r(t, \theta, \phi, h_b),
%   \end{gather}
%   where \(h_s\) is the radius of the free surface, and \(h_b\) is the radius of the
%   bottom topography.
%   Note that \(\pd{h_b}{t} = 0\), but is included for symmetry.

% \subsubsection{Dimensional Analysis}
%   First we assume that the radius of the sphere is much larger than the height of the
%   fluid, and we replace the instances of the variable \(r\) with the radius of the
%   sphere \(r_0\) in the Navier-Stokes equations.
%   So the system we are considering is
%   \begin{gather}
%     \pda{u_r}{r} + \frac{1}{r_0 \sin{\phi}} \pd{u_{\theta}}{\theta}
%       + \frac{1}{r_0 \sin{\phi}} \pda{\sin{\phi} u_{\phi}}{\phi} = 0 \\
%     \pd{u_r}{t} + \pda{u_r^2}{r} + \frac{1}{r_0 \sin{\phi}} \pda{u_r u_\theta}{\theta}
%       + \frac{1}{r_0} \pda{u_r u_{\phi}}{\phi}
%       + \frac{2 u_r^2 + u_r u_{\phi} \cot{\phi} - u_{\theta}^2 - u_{\phi}^2}{r_0}
%       = - \frac{1}{\rho_0} \pd{p}{r} + g e_r \\
%     \pd{u_{\theta}}{t} + \pda{u_r u_{\theta}}{r}
%       + \frac{1}{r_0 \sin{\phi}} \pda{u_{\theta}^2}{\theta}
%       + \frac{1}{r_0} \pda{u_{\theta} u_{\phi}}{\phi}
%       + 3\frac{u_r u_{\theta}}{r_0}
%       + 2\frac{u_{\theta} u_{\phi} \cot{\phi}}{r_0}
%       = - \frac{1}{\rho_0 r_0 \sin{\phi}} \pd{p}{\theta} + g e_{\theta} \\
%     \pd{u_{\phi}}{t} + \pda{u_r u_{\phi}}{r}
%       + \frac{1}{r_0 \sin{\phi}} \pda{u_{\theta} u_{\phi}}{\theta}
%       + \frac{1}{r_0} \pda{u_{\phi}^2}{\phi}
%       + 3 \frac{u_r u_{\phi}}{r_0}
%       + \frac{\p{u_{\phi}^2 - u_{\theta}^2} \cot{\phi}}{r_0}
%       = - \frac{1}{\rho_0 r_0} \pd{p}{\phi} + g e_{\phi},
%   \end{gather}
%   with boundary conditions
%   \begin{gather}
%     \pd{h_s}{t} + \frac{u_{\theta}(t, \theta, \phi, h_s)}{r_0 \sin{\phi}} \pd{h_s}{\theta}
%       + \frac{u_{\phi}(t, \theta, \phi, h_s)}{r_0} \pd{h_s}{\phi}
%       = u_r(t, \theta, \phi, h_s) \\
%     \pd{h_b}{t} + \frac{u_{\theta}(t, \theta, \phi, h_b)}{r_0 \sin{\phi}} \pd{h_b}{\theta}
%       + \frac{u_{\phi}(t, \theta, \phi, h_b)}{r_0} \pd{h_b}{\phi}
%       = u_r(t, \theta, \phi, h_b).
%   \end{gather}

%   Now we nondimensionalize the variables.
%   Let \(R\) be the characteristic radius and \(H\) be the characteristic height of the
%   fluid.
%   We are interested in situations where the aspect ratio \(\varepsilon = H/R\) is small.
%   We will introduce the following nondimensional variables
%   \begin{gather}
%     r = R \hat{r} \qquad h = H \hat{h}.
%   \end{gather}
%   The variables \(\theta \) and \(\phi \) are already nondimensional as they are angles
%   in radians.
%   We will scale the velocities by a characteristic velocity \(U\), i.e.
%   \begin{gather}
%     u_{\theta} = U \hat{u}_{\theta} \quad u_{\phi} = U \hat{u}_{\phi}
%     \quad u_r = \varepsilon U \hat{u}_r,
%   \end{gather}
%   where the vertical velocity is smaller due to the shallowness of the flow.
%   This



% \subsubsection{Mapping}

% \paragraph{Mapping of Mass Balance}

% \paragraph{Mapping of Momentum Balance}

%\chapterbib

%\bibliographystyle{apa}
%\bibliography{Reference/mybib}

