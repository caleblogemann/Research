% Chapter 3 from the thesis template file
%   that contains an example table and figure.
\chapter{Discontinuous Galerkin Method}

In this chapter, I describe the standard discontinuous Galerkin method for hyperbolic
balance laws.
I introduce the notation that I will be using throughout the thesis, and I describe
some of the keys ideas needed for the implementation of these methods.

\section{Generic Formulation}
  Consider a partial differential equation of the form
  \begin{gather}
    \v{q}_t + \div_{\v{x}} \M{f}\p{\v{q}, \v{x}, t} = \v{s}\p{\v{q}, \v{x}, t} \quad
    \text{for } \v{x} \in \Omega \subset \RR^d
  \end{gather}
  where \(\v{q}\) is a vector of \(N_e\) equations, \(\M{f}\) is the flux function,
  and \(\v{s}\) is the source function.
  The flux function maps values in \(\RR^{N_e} \times \RR^d \times \RR^+\) into
  matrices in \(\RR^{N_e \times d}\).
  Sometimes the flux function is considered as a set of vector functions, where there
  is one vector for each spatial dimension.
  I will however use the matrix notation.
  The divergence of the flux function is the sum of the spatial derivatives of the
  columns of \(\M{f}\), or in other words the divergence is over the last index of the
  matrix.
  The source function is a vector function from \(\RR^{N_e} \times \RR^d \times \RR^+\)
  into \(\RR^{N_e}\).
  These type of equations are known as balance laws and if the source function is zero,
  then they are called conservation laws.
  These equations need initial conditions and boundary conditions at all inflow points
  on the boundary \(\partial \Omega \) to be well-defined.
  In other words we also have
  \begin{gather}
    \v{q}(\v{x}, 0) = \v{q}_0(\v{x}) \\
    \v{q}(\v{x}, t) = \v{q}_b(\v{x}, t), \quad \v{x} \in \partial \Omega
  \end{gather}
  A boundary point is an inflow point if the eigenvalues of the jacobian of the flux
  function dotted into the outward point normal vector, \(\v{n} \cdot \M{f}'\),
  are negative.
  Specifically I am interested in when these type of equations are hyperbolic.
  Equations of this form are hyperbolic when the flux jacobian along any normal vector
  has real eigenvalues and is diagonalizable, that is when \(\v{n} \cdot \M{f}'\) is
  diagonalizable.

  One interesting feature of hyperbolic equations is that they may form
  discontinuities even when the initial condition and boundary conditions are smooth.
  In contrast this is not true for elliptic and parabolic partial differential
  equations, which have much stricter regularity theory.
  Because the solutions of these equations may contain discontinuities, the theory
  focuses on what are known as weak solutions instead of pointwise solutions, which are
  also known as strong solutions.
  The discontinuous Galerkin method is based on the idea of weak solutions to these
  PDEs.
  Finding weak solutions to the original PDE require searching an infinite dimensional
  space of functions.
  The discontinuous Galerkin method instead approximates the solution using a finite
  dimensional space.
  The way the DG method does this is by partitioning the domain \(\Omega \) as the set
  of elements \(K_i\) which I will label as \(\Omega_h = \set{K_i}_{i=1}^{N}\).
  The DG method then tries to find a solution that is polynomial on each element.
  Mathematically we denote the set of possible solutions as
  \begin{gather}
    V^k_h = \set{\v{q} \in L^1(\Omega \times \RR^+)| \eval{\v{q}}{K_i} \in \PP^k(K_i)}.
  \end{gather}
  Another way of writing this is with a basis expansion on each element,
  \begin{gather}
    \eval{\v{q}\p{\v{x}, t}}{K_i} = \sum{j=1}{k}{\v{Q}_i^j \phi_i^j(\v{x})}
      = \M{Q}_i \v{\phi}_i(\v{x}).
  \end{gather}
  To specify the DG method we need a set of linearly independent polynomials to
  form a basis on each element.
  To make things simpler I will use a single basis on a canonical element, \(\mcK \),
  and linear transformations from each mesh element and the canonical element.
  Let the spatial dimensions on the canonical element be denoted as \(\v{\xi}\),
  and I will denote the linear transformation from the mesh elements to the canonical
  elements and back as \(\v{c}_i(\v{x}): K_i \to \mcK \) and
  \(\v{b}_i(\v{\xi}): \mcK \to K_i\).
  Then if \(\set{\phi}\) is a basis of \(\PP^k(\mcK)\), we can describe a basis on each
  element with the linear transformations as follows,
  \begin{equation}
    \phi_i^k(\v{x}) = \phi^k(\v{c}_i(\v{x}))
    \text{ and } \phi^k(\v{\xi}) = \phi_i^k(b_i(\v{\xi})).
  \end{equation}

  % TODO: could expand on how to get the local statements

  The local statements of the discontinuous galerkin method
  \begin{equation}
    \dintt{K_i}{}{\v{q}_t \v{\phi}_i^T(\v{x})}{\v{x}}
    = \dintt{K_i}{}{\M{f}\p{\v{q}, \v{x}, t} \p{\v{\phi}_i'(\v{x})}^T}{\v{x}}
    - \dintt{\partial K_i}{}{\M{f}^* \v{n} \v{\phi}_i^T(\v{x})}{s}
    + \dintt{K_i}{}{\v{s}\p{\v{q}, \v{x}, t} \v{\phi}_i^T\p{\v{x}}}{\v{x}}
  \end{equation}
  On each element, \(K_i\) the discontinuous Galerkin solution can be written as an
  expansion of the basis, that is \(\eval{\v{q}}{K_i} = \M{Q}_i \v{\phi}_i(\v{x})\).
  Substituting this expression into the statement of the method gives,
  \begin{equation}
    \dintt{K_i}{}{\M{Q}_{i,t} \v{\phi}_i(\v{x}) \v{\phi}_i^T(\v{x})}{\v{x}}
    = \dintt{K_i}{}{\M{f}\p{\M{Q}_i \v{\phi}_i(\v{x}), \v{x}, t}
      \p{\v{\phi}_i'(\v{x})}^T}{\v{x}}
    - \dintt{\partial K_i}{}{\M{f}^* \v{n} \v{\phi}_i^T(\v{x})}{s}
    + \dintt{K_i}{}{\v{s}\p{\M{Q}_i \v{\phi}_i(\v{x}), \v{x}, t}
      \v{\phi}_i^T\p{\v{x}}}{\v{x}}
  \end{equation}
  Ideally we would like to only work with the basis functions on the canonical element,
  therefore using the function \(\v{b}_i(\v{\xi})\), the integrals can be transformed
  onto the canonical element with a change of variables.
  The integral of the numerical flux on the boundary of the element, will be left on
  the mesh element as in each dimension this integral looks very different.
  More details are given in future sections.
  The DG formulation is now
  \begin{gather}
    \dintt{\mcK}{}{\M{Q}_{i,t} \v{\phi}(\v{\xi}) \v{\phi}^T(\v{\xi}) m_i}{\v{\xi}}
    = \dintt{\mcK}{}{\M{f}\p{\M{Q}_i \v{\phi}(\v{\xi}), \v{b}_i(\v{\xi}), t}
      \p{\v{\phi}'(\v{\xi}) \v{c}_i'(\v{b}_i(\v{\xi}))}^T m_i}{\v{\xi}} \\
    - \dintt{\partial K_i}{}{\M{f}^* \v{n} \v{\phi}_i^T(\v{x})}{s}
    + \dintt{\mcK}{}{\v{s}\p{\M{Q}_i \v{\phi}(\v{\xi}), \v{b}_i(\v{\xi}), t}
      \v{\phi}^T\p{\v{\xi}} m_i}{\v{\xi}},
  \end{gather}
  where \(m_i = \frac{\abs{K_i}}{\abs{\mcK}} = \abs{b_i'(\v{xi})}\) is the element
  metric and satisfies
  \begin{equation}
    \dintt{K_i}{}{}{\v{x}} = \dintt{\mcK}{}{m_i}{\v{\xi}}.
  \end{equation}
  Simplifying and solving for \(\M{Q}_{i,t}\) gives
  \begin{gather}
    \M{Q}_{i,t}
    = \dintt{\mcK}{}{\M{f}\p{\M{Q}_i \v{\phi}(\v{\xi}), \v{b}_i(\v{\xi}), t}
      \p{\v{\phi}'(\v{\xi}) \v{c}_i'(\v{b}_i(\v{\xi}))}^T}{\v{\xi}} \M{M}^{-1} \\
    - \dintt{\partial K_i}{}{\M{f}^* \v{n} \v{\phi}_i^T(\v{x})}{s} \M{M}^{-1} \frac{1}{m_i}
    + \dintt{\mcK}{}{\v{s}\p{\M{Q}_i \v{\phi}(\v{\xi}), \v{b}_i(\v{\xi}), t}
      \v{\phi}^T\p{\v{\xi}}}{\v{\xi}} \M{M}^{-1},
  \end{gather}
  where \(M\) is the mass matrix on the canonical element.
  The mass matrix of a given basis on the canonical element is given by
  \begin{equation}
    M_{ij} = \dintt{\mcK}{}{\phi^i(\v{\xi}) \phi^k(\v{\xi})}{\v{\xi}}
  \end{equation}
  or
  \begin{equation}
    M = \dintt{\mcK}{}{\v{\phi}(\v{\xi}) \v{\phi}^T(\v{\xi})}{\v{\xi}}.
  \end{equation}
  In order to specify the discontinuous Galerkin method for a specify dimension and
  type of mesh element, a canonical element, \(\mcK \), the linear transformations,
  \(\v{c}_i\) and \(\v{b}_i\), the basis \(\v{\phi}\), and boundary integral all
  need to be described.

\section{One Dimension}
  Consider the one dimensional balance law given below.
  \begin{equation}
    \v{q}_t + \v{f}\p{\v{q}, x, t}_x = \v{s}(\v{q}, x, t)
  \end{equation}
  In one dimension the elements are \(K_i = \br{x_{i-1/2}, x_{i+1/2}}\), where the
  center of the element is given by \(x_i\) and
  \(\Delta x_i = \abs{K_i} = x_{i+1/2} - x_{i-1/2}\).
  The canonical element is \(\mcK = \br{-1, 1}\), and the linear transformations are
  \(c_i(x) = \p{x - x_i} \frac{2}{\Delta x_i}\) and
  \(b_i(\xi) = \frac{\Delta x_i}{2} \xi + x_i\).
  Then the element metric will be \(m_i = \frac{\Delta x_i}{2}\).
  The boundary integral of the numerical flux is just the point value at the two
  boundary points.

  The the DG method in one dimension can be expressed as
  \begin{gather}
    \M{Q}_{i,t}
    = \frac{2}{\Delta x_i}\dintt{-1}{1}{\M{f}\p{\M{Q}_i \v{\phi}(\xi), \v{b}_i(\xi), t}
      \v{\phi}_{\xi}^T(\xi)}{\xi} \M{M}^{-1} \\
    - \frac{2}{\Delta x_i} \p{\v{f}^*_{i+1/2} \v{\phi}^T(1) - \v{f}^*_{i-1/2} \v{\phi}^T(-1)} \M{M}^{-1}
    + \dintt{-1}{1}{\v{s}\p{\M{Q}_i \v{\phi}(\xi), b_i(\xi), t}
      \v{\phi}^T\p{\xi}}{\xi} \M{M}^{-1}.
  \end{gather}
  If the basis on the canonical element is orthonormal with orthogonality condition,
  \begin{gather}
    \frac{1}{2}\dintt{-1}{1}{\phi^j(\xi) \phi^k(\xi)}{\xi} = \delta_{jk},
  \end{gather}
  then the mass matrix and its inverse are given by \(M = 2I\) and
  \(M^{-1} = \frac{1}{2} I\).
  The DG method can then be simplified even further as
  \begin{gather}
    \M{Q}_{i,t}
    = \frac{1}{\Delta x_i}\dintt{-1}{1}{\M{f}\p{\M{Q}_i \v{\phi}(\xi), \v{b}_i(\xi), t}
      \v{\phi}_{\xi}^T(\xi)}{\xi}  \\
    - \frac{1}{\Delta x_i} \p{\v{f}^*_{i+1/2} \v{\phi}^T(1) - \v{f}^*_{i-1/2} \v{\phi}^T(-1)}
    + \frac{1}{2}\dintt{-1}{1}{\v{s}\p{\M{Q}_i \v{\phi}(\xi), b_i(\xi), t}
      \v{\phi}^T\p{\xi}}{\xi}.
  \end{gather}
  The integrals can be evaluated easily using gaussian quadrature.

\section{Two Dimensions}
  In two dimensions the flux function is a matrix function of size \(N_e \times 2\).
  Often it is denoted as two vector functions \(\v{f}_1\) and \(\v{f}_2\) or \(\v{f}\)
  and \(\v{g}\), however I will denote it as the matrix function
  \(\M{f} = \br{\v{f}_1, \v{f}_2} = \br{\v{f}, \v{g}}\).
  Also in two dimensions the boundary integral of the numerical flux is a line integral.
  A line integral can be expressed as a one dimensional integral through a
  parameterization of that line.
  Suppose we have a line \(L(\v{x}) = 0\), that can be parameterized by
  \(\v{l}(t) = \v{x}\) for \(t \in \br{t_1, t_2}\).
  Then the line integral can be written as
  \begin{equation}
    \dintt{L}{}{h(\v{x})}{s} = \dintt{t_1}{t_2}{h(\v{l}(t)) \norm{\v{l}'(t)}}{t}.
  \end{equation}

  In two dimensions the canonical element will have a set of faces, \(\mcF = \set{f_j}\).
  I will have a parameterization of each face of the canonical element, \(r_j(s)\), with
  \(s \in \br{-1, 1}\).
  Having \(s \in \br{-1, 1}\) is convenient as 1D quadrature rules won't need to be
  transformed from their canonical intervals.
  The actual integral is over the faces of the mesh element, so the actual
  parameterization for the faces of the mesh element will be \(\v{b}_i(\v{r}_j(t))\).
  In this way I will handle the transformation to the canonical element and the
  parameterization of the line in one step.
  Therefore the boundary integral of the numerical flux can be written as
  \begin{equation}
    \dintt{\partial K_i}{}{\M{f}^* \v{n} \v{\phi}_i^T(\v{x})}{s}
    = \sum{f_j \in \mcF}{}{\dintt{-1}{1}{\M{f}^* \v{n} \v{\phi}^T(\v{r}_j(s))
      \norm{\v{b}_i'(\v{r}_j(s)) \v{r}_j'(s)}}{s}}
  \end{equation}

  In two dimensions the discontinuous galerkin formulation is therefore
  \begin{gather}
    \M{Q}_{i,t}
    = \dintt{\mcK}{}{\M{f}\p{\M{Q}_i \v{\phi}(\v{\xi}), \v{b}_i(\v{\xi}), t}
      \p{\v{\phi}'(\v{\xi}) \v{c}_i'(\v{b}_i(\v{\xi}))}^T}{\v{\xi}} \M{M}^{-1} \\
    - \sum{f_j \in \mcF}{}{\dintt{-1}{1}{\M{f}^* \v{n} \v{\phi}^T(\v{r}_j(s))
      \norm{\v{b}_i'(\v{r}_j(s)) \v{r}_j'(s)}}{s}} \M{M}^{-1} \frac{1}{m_i}
    + \dintt{\mcK}{}{\v{s}\p{\M{Q}_i \v{\phi}(\v{\xi}), \v{b}_i(\v{\xi}), t}
      \v{\phi}^T\p{\v{\xi}}}{\v{\xi}} \M{M}^{-1}
  \end{gather}

\subsection{Rectangular Elements}
  Consider if the mesh contain rectangular elements, then
  \(K_i = \br{x_{i-1/2}, x_{i+1/2}} \times \br{y_{i-1/2}, y_{i+1/2}}\).
  The center of the element is \(\p{x_i, y_i}\) with
  \(\Delta x_i = x_{i+1/2} - x_{i-1/2}\) and \(\Delta y_i = y_{i+1/2} - y_{i-1/2}\).
  The canonical element is \(\mcK = \br{-1, 1} \times \br{-1, 1}\) with coordinates
  \(\v{\xi} = \br{\xi, \eta}\).
  The linear transformations are given by
  \begin{gather}
    \v{b}_i(\v{\xi}) = \br{\frac{\Delta x_i}{2} \xi + x_i, \frac{\Delta y_i}{2} \eta + y_i}^T \\
    \v{c}_i(\v{x}) = \br{\frac{2}{\Delta x_i} \p{x - x_i}, \frac{2}{\Delta y_i} \p{y - y_i}}^T
  \end{gather}
  with Jacobians
  \begin{gather}
    \M{b}_i' =
    \begin{pmatrix}
      \frac{\Delta x_i}{2} & 0 \\
      0 & \frac{\Delta y_i}{2}
    \end{pmatrix} \\
    \M{c}_i' =
    \begin{pmatrix}
      \frac{2}{\Delta x_i} & 0 \\
      0 & \frac{2}{\Delta y_i}
    \end{pmatrix}
  \end{gather}

  The metric of element i is \(m_i = \frac{\Delta x_i \Delta y_i}{4}\).
  Also the parameterizations of the left, right, bottom, and top faces,
  \(r_l, r_r, r_b, r_t\) respectively, are given by
  \begin{gather}
    r_l(s) = \br{-1, s} \\
    r_r(s) = \br{1, s} \\
    r_b(s) = \br{s, -1} \\
    r_t(s) = \br{s, 1}
  \end{gather}
  for \(s \in \br{-1, 1}\).
  We can easily compute \(\norm{\M{b}_i'(\v{r}_f(s)) \v{r}_f'(s)}\) for each face as well
  \begin{gather}
    \norm{\M{b}_i'(\v{r}_l(s)) \v{r}_l'(s)} = \frac{\Delta y_i}{2} \\
    \norm{\M{b}_i'(\v{r}_r(s)) \v{r}_r'(s)} = \frac{\Delta y_i}{2} \\
    \norm{\M{b}_i'(\v{r}_b(s)) \v{r}_b'(s)} = \frac{\Delta x_i}{2} \\
    \norm{\M{b}_i'(\v{r}_t(s)) \v{r}_t'(s)} = \frac{\Delta x_i}{2}
  \end{gather}
  Substituting all these into the formulation gives,
  \begin{gather}
    \M{Q}_{i, t}
    = \dintt{\mcK}{}{\frac{2}{\Delta x_i}\v{f}_1\p{\M{Q}_i \v{\phi}, \v{b}_i(\v{\xi}), t} \v{\phi}^T_{\xi} +\frac{2}{\Delta y_i}\v{f}_2\p{\M{Q}_i \v{\phi}, \v{b}_i(\v{\xi}), t} \v{\phi}^T_{\eta}}{\v{\xi}} \M{M}^{-1} \\
    + \frac{2}{\Delta x_i} \dintt{-1}{1}{\v{f}^*_1\p{b_i(\xi=-1, \eta)} \v{\phi}^T(\xi=-1, \eta)}{s} \M{M}^{-1}\\
    - \frac{2}{\Delta x_i} \dintt{-1}{1}{\v{f}^*_1\p{b_i(\xi=1, \eta)} \v{\phi}^T(\xi=1, \eta)}{s} \M{M}^{-1}\\
    + \frac{2}{\Delta y_i} \dintt{-1}{1}{\v{f}^*_2\p{b_i(\xi, \eta=-1)} \v{\phi}^T(\xi, \eta=-1)}{s} \M{M}^{-1}\\
    - \frac{2}{\Delta y_i} \dintt{-1}{1}{\v{f}^*_2\p{b_i(\xi, \eta=1)} \v{\phi}^T(\xi, \eta=1)}{s} \M{M}^{-1}
  \end{gather}
  For the case of a legendre orthogonal basis with orthogonality condition
  \[
    \frac{1}{4}\dintt{\mcK}{}{\phi^i(\v{\xi}) \phi^j(\v{\xi})}{\xi} = \delta_{ij},
  \]
  then the mass matrix and it's inverse become \(M = 4I\) and \(M^{-1} = \frac{1}{4}I\).
  So the full method becomes,
  \begin{gather}
    \M{Q}_{i, t}
    = \dintt{\mcK}{}{\frac{1}{2\Delta x_i}\v{f}_1\p{\M{Q}_i \v{\phi}, \v{b}_i(\v{\xi}), t} \v{\phi}^T_{\xi}
    + \frac{1}{2\Delta y_i}\v{f}_2\p{\M{Q}_i \v{\phi}, \v{b}_i(\v{\xi}), t} \v{\phi}^T_{\eta}}{\v{\xi}} \\
    + \frac{1}{2\Delta x_i} \dintt{-1}{1}{\v{f}^*_1\p{b_i(\xi=-1, \eta)} \v{\phi}^T(\xi=-1, \eta)}{s} \\
    - \frac{1}{2\Delta x_i} \dintt{-1}{1}{\v{f}^*_1\p{b_i(\xi=1, \eta)} \v{\phi}^T(\xi=1, \eta)}{s} \\
    + \frac{1}{2\Delta y_i} \dintt{-1}{1}{\v{f}^*_2\p{b_i(\xi, \eta=-1)} \v{\phi}^T(\xi, \eta=-1)}{s} \\
    - \frac{1}{2\Delta y_i} \dintt{-1}{1}{\v{f}^*_2\p{b_i(\xi, \eta=1)} \v{\phi}^T(\xi, \eta=1)}{s}
  \end{gather}

\subsection{Triangular Elements}
  Consider a mesh with triangular elements.
  That is each mesh element is given by three vertices in \(\RR^2\),
  \(\set{\v{v}_1, \v{v}_2, \v{v}_3}\).
  The coordinates of each vertex are given by \(\v{v}_i = \br{x_i, y_i}\).
  The canonical element that I will use is a right triangle with vertices,
  \(\br{-1, 1}\), \(\br{-1, -1}\), \(\br{1, -1}\).
  The linear transformations between mesh elements and the canonical element are
  given by
  \begin{gather}
    \v{b}_i(\v{\xi}) = \br{b_{00} \xi + b_{01} \eta + b_{02}, b_{10} \xi + b_{11} \eta + b_{12}} \\
    \v{c}_i(\v{x}) = \br{c_{00} x + c_{01} y + c_{02}, c_{10} x + c_{11} y + c_{12}}
  \end{gather}
  where the coefficients are
  \begin{gather}
    b_{00} = \frac{1}{2} \p{x_3 - x_2} \\
    b_{01} = \frac{1}{2} \p{x_1 - x_2} \\
    b_{02} = \frac{1}{2} \p{x_1 + x_3} \\
    b_{10} = \frac{1}{2} \p{y_3 - y_2} \\
    b_{11} = \frac{1}{2} \p{y_1 - y_2} \\
    b_{12} = \frac{1}{2} \p{y_1 + y_3} \\
    c_{00} = \frac{-2(y_1 - y_2)}{y_1 (x_2 - x_3) - x_1 (y_2 - y_3) + y_2 x_3 - x_2 y_3} \\
    c_{01} = \frac{2(x_1 - x_2)}{y_1 (x_2 - x_3) - x_1 (y_2 - y_3) + y_2 x_3 - x_2 y_3} \\
    c_{02} = \frac{y_1 (x_2 + x_3) - x_1 (y_2 + y_3) - y_2 x_3 + x_2 y_3}{y_1 (x_2 -
      x_3) - x_1 (y_2 - y_3) + y_2 x_3 - x_2 y_3} \\
    c_{10} = \frac{-2(y_2 - y_3)}{y_1 (x_2 - x_3) - x_1 (y_2 - y_3) + y_2 x_3 - x_2 y_3} \\
    c_{11} = \frac{2(x_2 - x_3)}{y_1 (x_2 - x_3) - x_1 (y_2 - y_3) + y_2 x_3 - x_2 y_3} \\
    c_{12} = \frac{x_1 (y_2 - y_3) - y_1 (x_2 - x_3) + y_2 x_3 - x_2 y_3}{y_1 (x_2 -
      x_3) - x_1 (y_2 - y_3) + y_2 x_3 - x_2 y_3}
  \end{gather}
  These coefficients were found by doing a linear solve such that the vertices of the
  mesh element would be transformed to the vertices of the canonical element.

  The jacobians of the linear transformations are
  \begin{gather}
    \v{b}_i'(\v{\xi}) =
    \begin{pmatrix}
      b_{00} & b_{01} \\
      b_{10} & b_{11}
    \end{pmatrix} \\
    \v{c}_i'(\v{x}) =
    \begin{pmatrix}
      c_{00} & c_{01} \\
      c_{10} & c_{11}
    \end{pmatrix}
  \end{gather}
  The metric of the element will be
  \(m_i = \det{\v{b}_i'} = b_{00}b_{11} - b_{10}b_{01}\).

  Also we can parameterize the left, bottom and hypotenuse faces of the canonical
  element as
  \begin{gather}
    r_l(s) = \br{-1, s} \\
    r_b(s) = \br{s, -1} \\
    r_h(s) = \br{s, -s}
  \end{gather}
  for \(s \in \br{-1, 1}\).
  We can easily compute \(\norm{\v{b}_i'(\v{r}_f(s)) \v{r}_f'(s)}\) for each face as
  well
  \begin{gather}
    \norm{\v{b}_i'(\v{r}_l(s)) \v{r}_l'(s)} = \sqrt{b_{01}^2 + b_{11}^2}
      = \frac{1}{2}\sqrt{\p{x_1 - x_2}^2 + \p{y_1 - y_2}^2} \\
    \norm{\v{b}_i'(\v{r}_b(s)) \v{r}_b'(s)} = \sqrt{b_{00}^2 + b_{10}^2}
      = \frac{1}{2}\sqrt{\p{x_3 - x_2}^2 + \p{y_3 - y^2}^2} \\
    \norm{\v{b}_i'(\v{r}_h(s)) \v{r}_h'(s)}
      = \sqrt{\p{b_{00} - b_{01}}^2 + \p{b_{10} - b_{11}}^2}
      = \frac{1}{2}\sqrt{\p{x_3 - x_1}^2 + \p{y_3 - y_1}^2}
  \end{gather}
  For the case of an orthonormal modal basis with orthogonality condition,
  \begin{gather}
    \frac{1}{2} \dintt{\mcK}{}{\phi^i(\v{\xi}) \phi^j(\v{\xi})}{\v{\xi}} = \delta_{ij}
  \end{gather}
  then the mass matrix and it's inverse will be \(\M{M} = 2I\) and
  \(\M{M}^{-1} = \frac{1}{2} I\).

  % \subsection{ODE solver}

%\chapterbib

%\bibliographystyle{apa}
%\bibliography{Reference/mybib}
