\documentclass[oneside]{article}
\usepackage[letterpaper, margin=2cm]{geometry}
\usepackage{Notes}

\begin{document}
  \begin{center}
    \textbf{\Large{Derivation of Shallow Water Equations}} \\
  \end{center}

  We begin by considering the Navier-Stokes equations,
  \begin{align}
    \div{\v{u}} &= 0 \\
    \v{u}_t + \div*{\v{u}\v{u}} &= - \frac{1}{\rho} \grad{p} + \frac{1}{\rho} \div{\sigma} + \v{g},
  \end{align}
  where \(\v{u} = \br{u, v, w}^T\) is the vector of velocities, \(p\) is the pressure,
  \(\rho \) is the constant density, \(\sigma \) is the deviatoric stress tensor, and
  \(\v{g}\) is the gravitational force vector.
  We also have two boundaries, the bottom topography \(h_b(t, x, y)\), and the free
  surface \(h_s(t, x, y)\).
  At both of these boundaries the kinimatic boundary conditions are in effect and can
  be expressed as
  \begin{align}
    \p{h_s}_t + \br{u(t, x, y, h_s), v(t, x, y, h_s)}^T \cdot \grad{h_s} &= w(t, x, y, h_s) \\
    \p{h_b}_t + \br{u(t, x, y, h_b), v(t, x, y, h_b)}^T \cdot \grad{h_b} &= w(t, x, y, h_b).
  \end{align}
  In practice the bottom topography is unchanging in time, but we express \(h_b\) with
  time dependence to allow for a symmetric representation of the boundary conditions.

  \subsection{Dimensional Analysis}
    Now we consider the characteristic scales of the problem.
    Let \(L\) be the characteristic horizontal length scale, and let \(H\) be the
    characteristic vertical length scale.
    For this problem we assume that \(H << L\) and we denote the ratio of these
    lengths as \(\varepsilon = H/L\).
    With these characteristic lengths we can scale the length variables to a
    nondimensional form
    \begin{equation}
      x = L\hat{x}, \quad  y = L\hat{y}, \quad z = H\hat{z}.
    \end{equation}
    Now let \(U\) be the characteristic horizontal velocity, then because of the
    shallowness the characteristic vertical velocity will be \(\varepsilon U\).
    Therefore the velocity variables can be scaled as follows,
    \begin{equation}
      u = U\hat{u}, \quad v = U\hat{v}, \quad w = \varepsilon U \hat{w}.
    \end{equation}
    Now with the characteristic length and velocity, the time scaling can be described
    as
    \begin{equation}
      t = \frac{L}{U}\hat{t}
    \end{equation}



\end{document}