\documentclass[11pt, oneside]{article}
\usepackage[letterpaper, margin=2cm]{geometry}
\usepackage{Notes}

\begin{document}
\begin{center}
\textbf{\Large{Local Discontinuous Galerkin Method for the Diffusion
Equation
}}
\end{center}
  We would like to solve the 1D diffusion equation with a Discontinuous Galerkin
  Method.
  The 1D diffusion equation is given as
  \[
    q_t = q_{xx}.
  \]

  If we were to naively apply DG methods, we could discretize the domain and
  consider piecewise polynomial approximation.
  We would then multiply by a test function and integrate by parts.
  \begin{gather*}
    \dintt{I_j}{}{u_t v}{x} = \dintt{I_j}{}{u_{xx} v}{x} \\
    \dintt{I_j}{}{u_t v}{x} = \p{\p{\hat{u}_x v^-}_{j+1/2} - \p{\hat{u}_x v^+}_{j-1/2}} - \dintt{I_j}{}{u_x v_x}{x}
    %\dintt{I_j}{}{u_t v}{x} = \p{\p{\hat{u}_x v^-}_{j+1/2} - \p{\hat{u}_x v^+}_{j-1/2}} 
      %- \p{\p{\hat{u} v^-_x}_{j+1/2} - \p{\hat{u} v^+_x}_{j-1/2}}
      %+ \dintt{I_j}{}{u v_{xx}}{x}
  \end{gather*}
  and we use the average numerical flux
  \begin{gather*}
    \p{\hat{u}_x}_{j+1/2} = \frac{\p{u^-_x}_{j+1/2} + \p{u^+_x}_{j+1/2}}{2}
  \end{gather*}
  This method is convergent and stable but it converges to the wrong solution.

\textbf{\large{Local Discontinuous Galerkin Method}}
  The Local Discontinuous Galerkin method proposes a different approach.
  First rewrite the diffusion equation as a system of first order equations.
  \begin{align*}
    r &= q_x \\
    q_t &= r_x
  \end{align*}

  The LDG method becomes the process of finding $q_h, r_h \in V_h$ in the DG solution space, such that
  for all test functions $v_h, w_h \in V_h$ and for all $j$ the following equations are satisfied
  \begin{align*}
    \dintt{I_j}{}{r_h w_h}{x} &= \dintt{I_j}{}{(q_h)_x w_h}{x} \\
    \dintt{I_j}{}{(q_h)_t v_h}{x} &= \dintt{I_j}{}{(r_h)_x v_h}{x}
  \end{align*}
  After integrating by parts, these equations are
  \begin{align*}
    \dintt{I_j}{}{r_h w_h}{x} &= \p{\p{\hat{q}_h w^-_h}_{j+1/2} - \p{\hat{q}_j w^+_h}_{j-1/2}}- \dintt{I_j}{}{q_h (w_h)_x}{x} \\
    \dintt{I_j}{}{(q_h)_t v_h}{x} &= \p{\p{\hat{r}_h v^-_h}_{j+1/2} - \p{\hat{r}_h v^+_h}_{j-1/2}} - \dintt{I_j}{}{r_h (v_h)_x}{x}
  \end{align*}
  A common choice of numerical fluxes are the so-called alternating fluxes.
  \begin{align*}
    \hat{r}_h &= r^-_h \\
    \hat{q}_h &= q^+_h
  \end{align*}

\textbf{\large{Implementation}}
  If we consider a single cell $I_j$, do a linear transformation from
  $x \in \br{x_{j-1/2}, x_{j+1/2}}$ to $\xi \in \br{-1, 1}$, and consider
  specifically the Legendre polynomial basis $\set{\phi^k(\xi)}$ with the
  following orthogonality property
  \[
    \frac{1}{2}\dintt{-1}{1}{\phi^j(\xi) \phi^k(\xi)}{\xi} = \delta_{jk}
  \]
  we can form a more concrete LDG method for implementing.
  The linear transformation can be expressed as
  \[
    x = \frac{\Delta x}{2} \xi + \frac{x_{j-1/2} + x_{j+1/2}}{2}
  \]
  or
  \[
    \xi = \frac{2}{\Delta x} \p{x - \frac{x_{j-1/2} + x_{j+1/2}}{2}}
  \]
  After this tranformation the diffusion equation become
  \[
    q_t = \frac{4}{\Delta x^2} q_{\xi\xi}
  \]
  on the cell $I_j$.
  We can then write this as the following system of first order equations.
  \begin{align*}
    r &= \frac{2}{\Delta x} q_{\xi} \\
    q_t &= \frac{2}{\Delta x} r_{\xi}
  \end{align*}
  With the Legendre basis, the numerical solution on $I_j$ can be written as
  \begin{align*}
    \eval{q}{I_i}{} &\approx \eval{q_h}{I_i}{} = \sum{l = 1}{M}{Q_i^l \phi^l(\xi)} \\
    \eval{r}{I_i}{} &\approx \eval{r_h}{I_i}{} = \sum{l = 1}{M}{R_i^l \phi^l(\xi)}
  \end{align*}
  Now plugging these into the system and multiplying by a Legendre basis and integrating gives.
  I will use the following shorthand for numerical fluxes using one of the alternating flux options.
  \begin{align*}
    \hat{Q}_{i+1/2} &= \sum{l = 1}{M}{Q_{i+1}^l \phi^l(-1)} \\
    \hat{R}_{i+1/2} &= \sum{l = 1}{M}{R_{i}^l \phi^l(1)}
  \end{align*}
  \begin{align*}
    r &= \frac{2}{\Delta x} q_{\xi} \\
    \sum{l = 1}{M}{R_i^l \phi^l(\xi)} &= \frac{2}{\Delta x} \sum{l = 1}{M}{Q_i^l \phi_{\xi}^l(\xi)} \\
    \frac{1}{2}\dintt{-1}{1}{\sum{l = 1}{M}{R_i^l \phi^l(\xi)} \phi^k(\xi)}{\xi} &= \frac{1}{\Delta x} \dintt{-1}{1}{\sum{l = 1}{M}{Q_i^l \phi_{\xi}^l(\xi)} \phi^k(\xi)}{\xi} \\
    R_i^k &= \frac{1}{\Delta x} \dintt{-1}{1}{\sum{l = 1}{M}{Q_i^l \phi_{\xi}^l(\xi)} \phi^k(\xi)}{\xi} \\
    R_i^k &= -\frac{1}{\Delta x} \dintt{-1}{1}{\sum{l = 1}{M}{Q_i^l \phi^l(\xi)} \phi_{\xi}^k(\xi)}{\xi} + \frac{1}{\Delta x}\p{\phi^k(1)\hat{Q}_{i+1/2} - \phi^k(-1)\hat{Q}_{i-1/2}} \\
  \end{align*}
  \begin{align*}
    q_t &= \frac{2}{\Delta x} r_{\xi} \\
    \sum{l = 1}{M}{\dot{Q}_i^l \phi^l(\xi)}  &= \frac{2}{\Delta x} \sum{l = 1}{M}{R_i^l \phi_{\xi}^l(\xi)} \\
    \frac{1}{2}\dintt{-1}{1}{\sum{l = 1}{M}{\dot{Q}_i^l \phi^l(\xi)} \phi^k(\xi)}{\xi}  &= \frac{1}{\Delta x} \dintt{-1}{1}{\sum{l = 1}{M}{R_i^l \phi_{\xi}^l(\xi)}\phi^k(\xi)}{\xi} \\
    \dot{Q}_i^k &= -\frac{1}{\Delta x} \dintt{-1}{1}{\sum{l = 1}{M}{R_i^l \phi^l(\xi)}\phi_{\xi}^k(\xi)}{\xi} + \frac{1}{\Delta x}\p{\phi^k(1)\hat{R}_{i+1/2} - \phi^k(-1)\hat{R}_{i-1/2}} \\
  \end{align*}
  Now this is a system of ODEs, there are $M \times N$ ODEs if $M$ is the spacial
  order and $N$ is the number of cells.

\textbf{\large{Matrix Representation}}
  Some common matrices and vectors that appear in these equations are
  \begin{align*}
    \v{Q}_{i} &= \br{Q_i^l}_{l=1}^M \\
    \v{\phi}(\xi) &= \br{\phi^k(\xi)}_{k = 1}^{M} \\
    \Phi(\xi_1,\xi_2) &= \v{\phi}(\xi_1) \v{\phi}^T(\xi_2) \\
    A &= \br{a_{kl}}_{k,l=1}^M \\
    a_{kl} &= \dintt{-1}{1}{\phi_{\xi}^k(\xi) \phi^l(\xi)}{\xi} \\
  \end{align*}

  Also the numerical fluxes can be written as the following dot products
  \begin{align*}
    \hat{Q}_{i+1/2} &= \sum{l = 1}{M}{Q_{i+1}^l \phi^l(-1)} \\
    &= \v{\phi}^T(-1)\v{Q}_{i+1} \\
    \hat{R}_{i+1/2} &= \sum{l = 1}{M}{R_{i}^l \phi^l(1)} \\
    &= \v{\phi}^T(1)\v{R}_{i} \\
  \end{align*}

  \begin{align*}
    R_i^k &= -\frac{1}{\Delta x} \dintt{-1}{1}{\sum{l = 1}{M}{Q_i^l \phi^l(\xi)} \phi_{\xi}^k(\xi)}{\xi} + \frac{1}{\Delta x}\p{\phi^k(1)\hat{Q}_{i+1/2} - \phi^k(-1)\hat{Q}_{i-1/2}} \\
    R_i^k &= -\frac{1}{\Delta x} \sum{l=1}{M}{Q_i^l\dintt{-1}{1}{\phi^l(\xi) \phi_{\xi}^k(\xi)}{\xi}} + \frac{1}{\Delta x}\p{\phi^k(1)\hat{Q}_{i+1/2} - \phi^k(-1)\hat{Q}_{i-1/2}} \\
    R_i^k &= -\frac{1}{\Delta x} \p{A \v{Q}_i}_k + \frac{1}{\Delta x}\p{\phi^k(1)\v{\phi}^T(-1)\v{Q}_{i+1} - \phi^k(-1)\v{\phi}^T(-1)\v{Q}_{i}} \\
    \v{R}_i &= -\frac{1}{\Delta x} A \v{Q}_i + \frac{1}{\Delta x}\p{\v{\phi}(1)\v{\phi}^T(-1)\v{Q}_{i+1} - \v{\phi}(-1)\v{\phi}^T(-1)\v{Q}_{i}} \\
    \v{R}_i &= -\frac{1}{\Delta x} A \v{Q}_i + \frac{1}{\Delta x}\p{\Phi(1, -1)\v{Q}_{i+1} - \Phi(-1, -1)\v{Q}_{i}} \\
    \v{R}_i &= -\frac{1}{\Delta x} \p{A + \Phi(-1,-1)} \v{Q}_i + \frac{1}{\Delta x}\Phi(1, -1)\v{Q}_{i+1}
  \end{align*}

  \begin{align*}
    Q_i^k &= -\frac{1}{\Delta x} \dintt{-1}{1}{\sum{l = 1}{M}{R_i^l \phi^l(\xi)} \phi_{\xi}^k(\xi)}{\xi} + \frac{1}{\Delta x}\p{\phi^k(1)\hat{R}_{i+1/2} - \phi^k(-1)\hat{R}_{i-1/2}}\\
    Q_i^k &= -\frac{1}{\Delta x} \p{A \v{R}_i}_k + \frac{1}{\Delta x}\p{\phi^k(1)\v{\phi}^T(1)\v{R}_{i} - \phi^k(-1)\v{\phi}^T(1)\v{R}_{i-1}} \\
    \v{Q}_i &= -\frac{1}{\Delta x} \p{A - \Phi(1, 1)} \v{R}_i - \frac{1}{\Delta x}\Phi(-1, 1)\v{R}_{i-1}
  \end{align*}

\textbf{\large{Proving Stability}}
  In order to prove that this method is $L^2$ stable consider we sum both of the integral equations from before.
  \begin{gather*}
    \dintt{I_j}{}{(u_h)_t v_h}{x} + \dintt{I_j}{}{q_h w_h}{x} = \p{\p{q^+_h v^-_h}_{j+1/2} - \p{q^+_h v^+_h}_{j-1/2}} \\
    + \p{\p{u^-_h w^-_h}_{j+1/2} - \p{u^-_j w^+_h}_{j-1/2}} - \dintt{I_j}{}{q_h (v_h)_x}{x} - \dintt{I_j}{}{u_h (w_h)_x}{x}
  \end{gather*}
  Consider using $v_h = u_h$ and $w_h = q_h$.
  \begin{gather*}
    \dintt{I_j}{}{(u_h)_t u_h}{x} + \dintt{I_j}{}{q_h q_h}{x} = \p{\p{q^+_h u^-_h}_{j+1/2} - \p{q^+_h u^+_h}_{j-1/2}} \\
    + \p{\p{u^-_h q^-_h}_{j+1/2} - \p{u^-_h q^+_h}_{j-1/2}} - \dintt{I_j}{}{q_h (u_h)_x}{x} - \dintt{I_j}{}{u_h (q_h)_x}{x}
  \end{gather*}
  Consider the following shorthand notation
  \begin{gather*}
    B_j = \dintt{I_j}{}{(u_h)_t u_h}{x} + \dintt{I_j}{}{q_h q_h}{x} \\
    B_j = \p{\p{q^+_h u^-_h}_{j+1/2} - \p{q^+_h u^+_h}_{j-1/2}} + \p{\p{u^-_h q^-_h}_{j+1/2} - \p{u^-_h q^+_h}_{j-1/2}} - \dintt{I_j}{}{q_h (u_h)_x}{x} - \dintt{I_j}{}{u_h (q_h)_x}{x}
  \end{gather*}
  %\begin{gather*}
    %B_j(u_h, q_h, u_h, q_h) = \dintt{I_j}{}{(u_h)_t u_h}{x} - \p{\p{\hat{q}_h u^-_h}_{j+1/2} - \p{\hat{q}_h u^+_h}_{j-1/2}} + \dintt{I_j}{}{q_h (u_h)_x}{x} \\
    %+ \dintt{I_j}{}{q_h q_h}{x} - \p{\p{\hat{u}_h q^-_h}_{j+1/2} - \p{\hat{u}_j q^+_h}_{j-1/2}} + \dintt{I_j}{}{u_h (q_h)_x}{x} = 0
  %\end{gather*}
  This can be simplified in several ways.
  First simplify the left hand side.
  \begin{align*}
    B_j &= \dintt{I_j}{}{(u_h)_t u_h}{x} + \dintt{I_j}{}{q_h q_h}{x} \\
    B_j &= \frac{1}{2} \dintt{I_j}{}{\d*{u_h^2}{t}}{x} + \dintt{I_j}{}{q_h^2}{x} \\
    B_j &= \frac{1}{2} \d{}{t} \dintt{I_j}{}{u_h^2}{x} + \dintt{I_j}{}{q_h^2}{x} \\
    B_j &= \frac{1}{2} \d{}{t} \norm[L^2(I_j)]{u_h}^2 + \norm[L^2(I_j)]{q_h}^2 \\
  \end{align*}
  Second the right hand side can be simplified.
  \begin{align*}
    \dintt{I_j}{}{q_h (u_h)_x}{x} + \dintt{I_j}{}{u_h (q_h)_x}{x} &= \dintt{I_j}{}{q_h (u_h)_x + u_h (q_h)_x}{x} \\
    &= \dintt{I_j}{}{(q_h u_h)_x}{x} \\
    &= (q^-_h u^-_h)_{j+1/2} - (q^+_h u^+_h)_{j-1/2}
  \end{align*}
  Now 
  \begin{gather*}
    B_j = \p{\p{q^+_h u^-_h}_{j+1/2} - \p{q^+_h u^+_h}_{j-1/2}} + \p{\p{u^-_h q^-_h}_{j+1/2} - \p{u^-_h q^+_h}_{j-1/2}} - \p{(q^-_h u^-_h)_{j+1/2} - (q^+_h u^+_h)_{j-1/2}} \\
    B_j = \p{q^+_h u^-_h}_{j+1/2} - \p{u^-_h q^+_h}_{j-1/2} \\
  \end{gather*}
  Assuming periodic boundary conditions, and summing $B_j$ over all cells
  \begin{align*}
    \sum{j = 1}{N}{B_j} &= \sum{j=1}{N}{\p{q^+_h u^-_h}_{j+1/2} - \p{u^-_h q^+_h}_{j-1/2}} \\
    &= -\p{u^-_h q^+_h}_{1/2} + \sum{k=1}{N}{\p{q^+_h u^-_h}_{k+1/2} - \p{u^-_h q^+_h}_{k+1/2}} + \p{q^+_h u^-_h}_{N+1/2} \\
    &= 0
  \end{align*}
  This shows that
  \begin{gather*}
    \sum{j = 1}{N}{B_j} = \sum{j = 1}{N}{\frac{1}{2} \d{}{t} \norm[L^2(I_j)]{u_h}^2 + \norm[L^2(I_j)]{q_h}^2} \\
    \frac{1}{2} \d{}{t} \norm[L^2]{u_h}^2 + \norm[L^2]{q_h}^2 = 0 \\
    \d{}{t} \norm[L^2]{u_h}^2 \le 0
  \end{gather*}
\end{document}
