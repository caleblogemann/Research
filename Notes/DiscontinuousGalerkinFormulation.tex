\documentclass[oneside]{article}
\usepackage[letterpaper, margin=2cm]{geometry}
\usepackage{Notes}

\begin{document}
\begin{center}
  \textbf{\Large{Discontinuous Galerkin Formulation for Balance Laws}} \\
\end{center}

\section{Notation}
  We partition the domain \(\Omega \) as a set of elements \(K_i\) and label the
  set \(\Omega_h = \set{K_i}_{i=1}^{N_e}\).
  Each type of element corresponds to a canonical element denoted \(\mcK \).
  The coordinates of the mesh will be denoted \(\v{x}\) and the coordinates in the
  canonical element will be denoted as \(\v{\xi}\).
  There is a linear transformation from each element to the canonical element and
  back denoted \(\v{c}_i(\v{x}): K_i \to \mcK \) and \(\v{b}_i(\v{\xi}): \mcK \to K_i\).
  The metric of the mesh element is denoted as \(m_i\) and is given by
  \begin{equation}
    \dintt{K_i}{}{}{\v{x}} = \dintt{\mcK}{}{m_i}{\v{\xi}}
  \end{equation}
  or \(m_i = \frac{\abs{K_i}}{\abs{\mcK}} = \abs{b_i'(\v{xi})}\).
  Solutions will be elements of the Discontinuous Galerkin space of order \(M\),
  \begin{equation}
    V_h = \set{v \in L^1\p{\Omega} \big| \eval{v}{K_i} \in \PP^M(K_i)}
  \end{equation}
  The basis for this space is given by \(\set{\phi_i^k(\v{x})}\) for \(1 \le i \le N_e\) and
  \(1 \le k \le N_b\).
  The number of basis components, \(N_b\) in each element to space \(\PP^M(K_i)\)
  depends on the dimension and the element shape.
  The polynomial basis on the canonical element is given by
  \(\set{\phi^k(\v{\xi})}_{k=1}{N_b}\).
  The basis on the canonical element can be shifted to be the basis on any element
  \(K_i\) using the transformations \(b_i\) and \(c_i\), that is
  \begin{equation}
    \phi_i^k(\v{x}) = \phi^k(\v{c}_i(\v{x})) \text{ and } \phi^k(\v{\xi}) = \phi_i^k(b_i(\v{\xi})).
  \end{equation}
  The basis can be put into vector form for convenience
  \(\v{\phi} = \br{\phi^1, \cdots \phi^{N_b}}\).
  The mass matrix of a given basis on the canonical element is given by
  \begin{equation}
    M_{ij} = \dintt{\mcK}{}{\phi^i(\v{\xi}) \phi^k(\v{\xi})}{\v{\xi}}
  \end{equation}
  or
  \begin{equation}
    M = \dintt{\mcK}{}{\v{\phi}(\v{\xi}) \v{\phi}^T(\v{\xi})}{\v{\xi}}.
  \end{equation}

  We will also have to evaluate integrals over the faces of the elements.
  In particular we will have to evaluate line integrals over the faces of
  the canonical element in 2D.
  Let \(f\) be a face of \(\mcK \) and let \(\v{r}_f(t)\) be a parameterization of that
  face, then
  \[
    \dintt{\partial K_i}{}{h(x)}{s} = \sum{f \in \mcK}{}{\dintt{}{}{h(\v{b}_i(\v{r}_f(t))) \norm{\M{b}_i'(\v{r}_f(t)) \v{r}_f'(t)}}{t}}
  \]

\section{One Dimension}
  Consider the one dimensional balance law given below.
  \begin{equation}
    \v{q}_t + \v{f}\p{\v{q}, x, t}_x = \v{s}(\v{q}, x, t)
  \end{equation}
  In one dimension the elements are \(K_i = \br{x_{i-1/2}, x_{i+1/2}}\), where the
  center of the element is given by \(x_i\) and
  \(\Delta x_i = \abs{K_i} = x_{i+1/2} - x_{i-1/2}\).
  The canonical element is \(\mcK = \br{-1, 1}\), and the linear transformations are
  \(c_i(x) = \p{x - x_i} \frac{2}{\Delta x_i}\) and
  \(b_i(\xi) = \frac{\Delta x_i}{2} \xi + x_i\).

\section{Two Dimensions}
  Consider the two dimensional balance law given by
  \begin{equation}
    \v{q}_t + \div \v{f}_j\p{\v{q}, \v{x}, t} = \v{s}\p{\v{q}, \v{x}, t}
  \end{equation}
  Note that the flux function is a matrix or two index tensor, so the divergence is a
  vector quantity.
  It could also be written as
  \begin{equation}
    \v{q}_t + \v{f}_1\p{\v{q}, \v{x}, t}_{x_1} + \v{f}_2\p{\v{q}, \v{x}, t}_{x_2}
    = \v{s}\p{\v{q}, \v{x}, t}
  \end{equation}
  The local statements of the weak discontinuous Galerkin form are given by
  \begin{equation}
    \dintt{K_i}{}{\v{q}_t \phi_i^k(\v{x})
    - \v{f}_j\p{\v{q}, \v{x}, t} \phi^k_{i,x_j}(\v{x})}{\v{x}}
    = -\dintt{\partial K_i}{}{\v{n} \cdot \v{f}^*_j \phi_i^k(\v{x})}{s}.
  \end{equation}
  Note that all of the terms related to the flux function \(\v{f}\) are sums over the
  two dimensions.
  It could also be written as
  \begin{equation}
    \dintt{K_i}{}{\v{q}_t \phi_i^k(\v{x})
    - \v{f}_1\p{\v{q}, \v{x}, t} \phi^k_{i, x_1}(\v{x})
    - \v{f}_2\p{\v{q}, \v{x}, t} \phi^k_{i, x_2}(\v{x})}{\v{x}}
    = -\dintt{\partial K_i}{}{\p{n_1 \v{f}^*_1 + n_2 \v{f}^*_2} \phi_i^k(\v{x})}{s}.
  \end{equation}
  We could also consider all of the basis components at one time, by using the test
  function \(\v{\phi}^T_i\) instead of \(\phi^k_i\).
  \begin{gather}
    \dintt{K_i}{}{\v{q}_t \v{\phi}^T_i(\v{x})
    - \v{f}_1\p{\v{q}, \v{x}, t} \v{\phi}^T_{i, x_1}(\v{x})
    - \v{f}_2\p{\v{q}, \v{x}, t} \v{\phi}^T_{i, x_2}(\v{x})}{\v{x}}
    = -\dintt{\partial K_i}{}{\p{n_1 \v{f}^*_1 + n_2 \v{f}^*_2} \v{\phi}_i^T(\v{x})}{s}.
    \intertext{Now using the fact that \(\eval{\v{q}}{K_i}{} = Q_i \v{\phi}_i(\v{x})\)}
    \dintt{K_i}{}{Q_{i,t} \v{\phi}_i(\v{x}) \v{\phi}^T_i(\v{x})
    - \v{f}_1\p{Q_i \v{\phi}_i(\v{x}), \v{x}, t} \v{\phi}^T_{i, x_1}(\v{x})
    - \v{f}_2\p{Q_i \v{\phi}_i(\v{x}), \v{x}, t} \v{\phi}^T_{i, x_2}(\v{x})}{\v{x}}
    = -\dintt{\partial K_i}{}{\p{n_1 \v{f}^*_1 + n_2 \v{f}^*_2} \v{\phi}_i^T(\v{x})}{s}.
    \intertext{Rearranging to solve for \(Q_{i,t}\), and dropping explicit dependence
      on \(\v{x}\) for \(\v{\phi}\)}
    Q_{i,t} \dintt{K_i}{}{\v{\phi}_i \v{\phi}^T_i}{\v{x}} =
    \dintt{K_i}{}{\v{f}_1\p{Q_i \v{\phi}_i, \v{x}, t} \v{\phi}^T_{i, x_1}
    + \v{f}_2\p{Q_i \v{\phi}_i, \v{x}, t} \v{\phi}^T_{i, x_2}}{\v{x}}
    -\dintt{\partial K_i}{}{\p{n_1 \v{f}^*_1 + n_2 \v{f}^*_2} \v{\phi}_i^T}{s}
    \intertext{Transforming to canonical element}
    Q_{i,t} m_i M =
    \dintt{K_i}{}{\v{f}_1\p{Q_i \v{\phi}_i, \v{x}, t} \v{\phi}^T_{i, x_1}
    + \v{f}_2\p{Q_i \v{\phi}_i, \v{x}, t} \v{\phi}^T_{i, x_2}}{\v{x}}
    -\dintt{\partial K_i}{}{\p{n_1 \v{f}^*_1 + n_2 \v{f}^*_2} \v{\phi}_i^T}{s}.
  \end{gather}

  \begin{equation}
    \dintt{K_i}{}{\v{q}_t \phi_i^k(\v{x})
    - \M{f}\p{\v{q}, \v{x}, t} \phi^k_{i,x_j}(\v{x})}{\v{x}}
    = -\dintt{\partial K_i}{}{\v{n} \cdot \M{f}^* \phi_i^k(\v{x})}{s}.
  \end{equation}
  We could also consider all of the basis components at one time, by using the test
  function \(\v{\phi}^T_i\) instead of \(\phi^k_i\).
  \begin{gather}
    \dintt{K_i}{}{\v{q}_t \v{\phi}^T_i(\v{x})
    - \M{f}\p{\v{q}, \v{x}, t} \M{D\phi}^T_{i}(\v{x})}{\v{x}}
    = -\dintt{\partial K_i}{}{\v{n} \cdot \M{f}^*(\v{x}) \v{\phi}_i^T(\v{x})}{s}
    \intertext{Now using the fact that \(\eval{\v{q}}{K_i}{} = Q_i \v{\phi}_i(\v{x})\)}
    \dintt{K_i}{}{\M{Q}_{i,t} \v{\phi}_i(\v{x}) \v{\phi}^T_i(\v{x})
    - \M{f}\p{\M{Q}_i \v{\phi}_i(\v{x}), \v{x}, t} \M{D\phi}^T_{i}(\v{x})}{\v{x}}
    = -\dintt{\partial K_i}{}{\v{n} \cdot \M{f}^*(\v{x}) \v{\phi}_i^T(\v{x})}{s}
    \intertext{Rearranging to solve for \(Q_{i,t}\), and dropping explicit dependence
      on \(\v{x}\) for \(\v{\phi}\)}
    \M{Q}_{i, t} \dintt{K_i}{}{\v{\phi}_i \v{\phi}^T_i}{\v{x}}
    = \dintt{K_i}{}{\M{f}\p{\M{Q}_i \v{\phi}_i, \v{x}, t} \M{D\phi}^T_i}{\v{x}}
    - \dintt{\partial K_i}{}{\v{n} \cdot \M{f}^*\p{\v{x}} \v{\phi}_i^T}{s}
    \intertext{Transforming to canonical element}
    \M{Q}_{i, t} m_i \M{M}
    = \dintt{\mcK}{}{\M{f}\p{\M{Q}_i \v{\phi}, \v{b}_i(\v{\xi}), t} \M{D\phi}^T \M{c}_i' m_i}{\v{\xi}}
    - \sum{f \in \mcK}{}{\dintt{}{}{\v{n} \cdot \M{f}^*\p{b_i(r_f(t))} \v{\phi}^T(r_f(t)) \norm{b_i'(r_f(t)) r_f'(t)}}{t}}
    \intertext{Then the set of ODEs are}
    \M{Q}_{i, t}
    = \dintt{\mcK}{}{\M{f}\p{\M{Q}_i \v{\phi}, \v{b}_i(\v{\xi}), t} \M{D\phi}^T \M{c}_i'}{\v{\xi}} \M{M}^{-1}
    - \sum{f \in \mcK}{}{\dintt{}{}{\v{n} \cdot \M{f}^*\p{b_i(r_f(t))} \v{\phi}^T(r_f(t)) \norm{b_i'(r_f(t)) r_f'(t)}}{t} \frac{1}{m_i} M^{-1}}
  \end{gather}

\subsection{Rectangular Elements}
  Consider if the mesh contain rectangular elements, then
  \(K_i = \br{x_{i-1/2}, x_{i+1/2}} \times \br{y_{i-1/2}, y_{i+1/2}}\).
  The center of the element is \(\p{x_i, y_i}\) with
  \(\Delta x_i = x_{i+1/2} - x_{i-1/2}\) and \(\Delta y_i = y_{i+1/2} - y_{i-1/2}\).
  The canonical element is \(\mcK = \br{-1, 1} \times \br{-1, 1}\) with coordinates
  \(\v{\xi} = \br{\xi, \eta}\).
  The linear transformations are given by
  \begin{gather}
    \v{b}_i(\v{\xi}) = \br{\frac{\Delta x_i}{2} \xi + x_i, \frac{\Delta y_i}{2} \eta + y_i}^T \\
    \v{c}_i(\v{x}) = \br{\frac{2}{\Delta x_i} \p{x - x_i}, \frac{2}{\Delta y_i} \p{y - y_i}}^T \\
  \end{gather}
  with Jacobians
  \begin{gather}
    \M{b}_i' =
    \begin{pmatrix}
      \frac{\Delta x_i}{2} & 0 \\
      0 & \frac{\Delta y_i}{2}
    \end{pmatrix} \\
    \M{c}_i' =
    \begin{pmatrix}
      \frac{2}{\Delta x_i} & 0 \\
      0 & \frac{2}{\Delta y_i}
    \end{pmatrix}
  \end{gather}

  The metric of element i is \(m_i = \frac{\Delta x_i \Delta y_i}{4}\).
  Also the parameterizations of the left, right, bottom, and top faces,
  \(r_l, r_r, r_b, r_t\) respectively, are given by
  \begin{gather}
    r_l(t) = \br{-1, t} \\
    r_r(t) = \br{1, t} \\
    r_b(t) = \br{t, -1} \\
    r_t(t) = \br{t, 1}
  \end{gather}
  for \(t \in \br{-1, 1}\).
  We can easily compute \(\norm{\M{b}_i'(\v{r}_f(t)) \v{r}_f'(t)}\) for each face as well
  \begin{gather}
    \norm{\M{b}_i'(\v{r}_l(t)) \v{r}_l'(t)} = \frac{\Delta y_i}{2} \\
    \norm{\M{b}_i'(\v{r}_r(t)) \v{r}_r'(t)} = \frac{\Delta y_i}{2} \\
    \norm{\M{b}_i'(\v{r}_b(t)) \v{r}_b'(t)} = \frac{\Delta x_i}{2} \\
    \norm{\M{b}_i'(\v{r}_t(t)) \v{r}_t'(t)} = \frac{\Delta x_i}{2}
  \end{gather}
  Substituting all these into the formulation gives,
  \begin{gather}
    \M{Q}_{i, t}
    = \dintt{\mcK}{}{\frac{2}{\Delta x_i}\v{f}_1\p{\M{Q}_i \v{\phi}, \v{b}_i(\v{\xi}), t} \v{\phi}^T_{\xi} +\frac{2}{\Delta y_i}\v{f}_2\p{\M{Q}_i \v{\phi}, \v{b}_i(\v{\xi}), t} \v{\phi}^T_{\eta}}{\v{\xi}} \M{M}^{-1} \\
    + \frac{2}{\Delta x_i} \dintt{-1}{1}{\v{f}^*_1\p{b_i(\xi=-1, \eta)} \v{\phi}^T(\xi=-1, \eta)}{t} \M{M}^{-1}\\
    - \frac{2}{\Delta x_i} \dintt{-1}{1}{\v{f}^*_1\p{b_i(\xi=1, \eta)} \v{\phi}^T(\xi=1, \eta)}{t} \M{M}^{-1}\\
    + \frac{2}{\Delta y_i} \dintt{-1}{1}{\v{f}^*_2\p{b_i(\xi, \eta=-1)} \v{\phi}^T(\xi, \eta=-1)}{t} \M{M}^{-1}\\
    - \frac{2}{\Delta y_i} \dintt{-1}{1}{\v{f}^*_2\p{b_i(\xi, \eta=1)} \v{\phi}^T(\xi, \eta=1)}{t} \M{M}^{-1}\\
  \end{gather}
  For the case of a legendre orthogonal basis with orthogonality condition
  \[
    \frac{1}{4}\dintt{\mcK}{}{\phi^i(\v{\xi}) \phi^j(\v{\xi})}{\xi} = \delta_{ij},
  \]
  then the mass matrix and it's inverse become \(M = 4I\) and \(M = \frac{1}{4}I\).
  So the full method becomes,
  \begin{gather}
    \M{Q}_{i, t}
    = \dintt{\mcK}{}{\frac{1}{2\Delta x_i}\v{f}_1\p{\M{Q}_i \v{\phi}, \v{b}_i(\v{\xi}), t} \v{\phi}^T_{\xi}
    + \frac{1}{2\Delta y_i}\v{f}_2\p{\M{Q}_i \v{\phi}, \v{b}_i(\v{\xi}), t} \v{\phi}^T_{\eta}}{\v{\xi}} \\
    + \frac{1}{2\Delta x_i} \dintt{-1}{1}{\v{f}^*_1\p{b_i(\xi=-1, \eta)} \v{\phi}^T(\xi=-1, \eta)}{t} \\
    - \frac{1}{2\Delta x_i} \dintt{-1}{1}{\v{f}^*_1\p{b_i(\xi=1, \eta)} \v{\phi}^T(\xi=1, \eta)}{t} \\
    + \frac{1}{2\Delta y_i} \dintt{-1}{1}{\v{f}^*_2\p{b_i(\xi, \eta=-1)} \v{\phi}^T(\xi, \eta=-1)}{t} \\
    - \frac{1}{2\Delta y_i} \dintt{-1}{1}{\v{f}^*_2\p{b_i(\xi, \eta=1)} \v{\phi}^T(\xi, \eta=1)}{t} \\
  \end{gather}

\end{document}