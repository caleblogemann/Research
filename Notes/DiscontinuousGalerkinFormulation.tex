\documentclass[oneside]{article}
\usepackage[letterpaper, margin=2cm]{geometry}
\usepackage{Notes}

\begin{document}
\begin{center}
  \textbf{\Large{Discontinuous Galerkin Formulation for Balance Laws}} \\
\end{center}

\section{Notation}
  We partition the domain \(\Omega \) as a set of elements \(K_i\) and label the
  set \(\Omega_h = \set{K_i}_{i=1}^{N_e}\).
  Each type of element corresponds to a canonical element denoted \(\mcK \).
  The coordinates of the mesh will be denoted \(\v{x}\) and the coordinates in the
  canonical element will be denoted as \(\v{\xi}\).
  There is a linear transformation from each element to the canonical element and
  back denoted \(\v{c}_i(\v{x}): K_i \to \mcK \) and \(\v{b}_i(\v{\xi}): \mcK \to K_i\).
  The metric of the mesh element is denoted as \(m_i\) and is given by
  \begin{equation}
    \dintt{K_i}{}{}{\v{x}} = \dintt{\mcK}{}{m_i}{\v{\xi}}
  \end{equation}
  or \(m_i = \frac{\abs{K_i}}{\abs{\mcK}} = \abs{b_i'(\v{xi})}\).
  Solutions will be elements of the Discontinuous Galerkin space of order \(M\),
  \begin{equation}
    V_h = \set{v \in L^1\p{\Omega} \big| \eval{v}{K_i} \in \PP^M(K_i)}
  \end{equation}
  The basis for this space is given by \(\set{\phi_i^k(\v{x})}\) for \(1 \le i \le N_e\) and
  \(1 \le k \le N_b\).
  The number of basis components, \(N_b\) in each element to space \(\PP^M(K_i)\)
  depends on the dimension and the element shape.
  The polynomial basis on the canonical element is given by
  \(\set{\phi^k(\v{\xi})}_{k=1}{N_b}\).
  The basis on the canonical element can be shifted to be the basis on any element
  \(K_i\) using the transformations \(b_i\) and \(c_i\), that is
  \begin{equation}
    \phi_i^k(\v{x}) = \phi^k(\v{c}_i(\v{x})) \text{ and } \phi^k(\v{\xi}) = \phi_i^k(b_i(\v{\xi})).
  \end{equation}
  The basis can be put into vector form for convenience
  \(\v{\phi} = \br{\phi^1, \cdots \phi^{N_b}}\).
  The mass matrix of a given basis on the canonical element is given by
  \begin{equation}
    M_{ij} = \dintt{\mcK}{}{\phi^i(\v{\xi}) \phi^k(\v{\xi})}{\v{\xi}}
  \end{equation}
  or
  \begin{equation}
    M = \dintt{\mcK}{}{\v{\phi}(\v{\xi}) \v{\phi}^T(\v{\xi})}{\v{\xi}}.
  \end{equation}

  We will also have to evaluate integrals over the faces of the elements.
  In particular we will have to evaluate line integrals over the faces of
  the canonical element in 2D.
  Let \(f\) be a face of \(\mcK \) and let \(\v{r}_f(t)\) be a parameterization of that
  face, then
  \[
    \dintt{\partial K_i}{}{h(x)}{s} = \sum{f \in \mcK}{}{\dintt{}{}{h(\v{b}_i(\v{r}_f(t))) \norm{\M{b}_i'(\v{r}_f(t)) \v{r}_f'(t)}}{t}}
  \]

\section{Generic Formulation}
  Consider a balance law in d-dimensions of the form
  \[
    \v{q}_t + \div_{\v{x}} \M{f}\p{\v{q}, \v{x}, t} = \v{s}\p{\v{q}, \v{x}, t}
  \]
  This is a vector equation with \(N_e\) equations.
  \(\M{f}\) is known as the flux function with output of shape \(N_e \times d\).
  The divergence of the flux function is the sum of the spatial derivatives of the
  columns of \(\M{f}\), or in other words the divergence is over the last index of the
  matrix.
  Lastly the source term is given by \(\v{s}\p{\v{q}, \v{x}, t}\) whose range is also
  \(\RR^{N_e}\).
  If the source term is zero then this is known as a conservation law.

  % Weak Solution to PDEs
  % Finite Dimensional Space

  \[
    \dintt{K_i}{}{\v{q}_t \v{\phi}_i^T\p{\v{x}}}{\v{x}}
    = \dintt{K_i}{}{\M{f}\p{\v{q}, \v{x}, t} \p{\v{\phi}_i'\p{\v{x}}}^T}{\v{x}}
    - \dintt{\partial K_i}{}{\M{f}^* \v{n} \v{\phi}_i^T(\v{x})}{s}
  \]

  The local statements of the discontinuous galerkin method
  \begin{equation}
    \dintt{K_i}{}{\v{q}_t \v{\phi}_i^T(\v{x})}{\v{x}}
    = \dintt{K_i}{}{\M{f}\p{\v{q}, \v{x}, t} \p{\v{\phi}_i'(\v{x})}^T}{\v{x}}
    - \dintt{\partial K_i}{}{\M{f}^* \v{n} \v{\phi}_i^T(\v{x})}{s}
    + \dintt{K_i}{}{\v{s}\p{\v{q}, \v{x}, t} \v{\phi}_i^T\p{\v{x}}}{\v{x}}
  \end{equation}
  On each element, \(K_i\) the discontinuous Galerkin solution can be written as an
  expansion of the basis, that is \(\eval{\v{q}}{K_i} = \M{Q}_i \v{\phi}_i(\v{x})\).
  Substituting this expression into the statement of the method gives,
  \begin{equation}
    \dintt{K_i}{}{\M{Q}_{i,t} \v{\phi}_i(\v{x}) \v{\phi}_i^T(\v{x})}{\v{x}}
    = \dintt{K_i}{}{\M{f}\p{\M{Q}_i \v{\phi}_i(\v{x}), \v{x}, t}
      \p{\v{\phi}_i'(\v{x})}^T}{\v{x}}
    - \dintt{\partial K_i}{}{\M{f}^* \v{n} \v{\phi}_i^T(\v{x})}{s}
    + \dintt{K_i}{}{\v{s}\p{\M{Q}_i \v{\phi}_i(\v{x}), \v{x}, t}
      \v{\phi}_i^T\p{\v{x}}}{\v{x}}
  \end{equation}
  Ideally we would like to only work with the basis functions on the canonical element,
  therefore using the function \(\v{b}_i(\v{\xi})\), the integrals can be transformed
  onto the canonical element with a change of variables.
  The integral of the numerical flux on the boundary of the element, will be left on
  the mesh element as in each dimension this integral looks very different.
  More details are given in future sections.
  \begin{gather}
    \dintt{\mcK}{}{\M{Q}_{i,t} \v{\phi}(\v{\xi}) \v{\phi}^T(\v{\xi}) m_i}{\v{\xi}}
    = \dintt{\mcK}{}{\M{f}\p{\M{Q}_i \v{\phi}(\v{\xi}), \v{b}_i(\v{\xi}), t}
      \p{\v{\phi}'(\v{\xi}) \v{c}_i'(\v{b}_i(\v{\xi}))}^T m_i}{\v{\xi}} \\
    - \dintt{\partial K_i}{}{\M{f}^* \v{n} \v{\phi}_i^T(\v{x})}{s}
    + \dintt{\mcK}{}{\v{s}\p{\M{Q}_i \v{\phi}(\v{\xi}), \v{b}_i(\v{\xi}), t}
      \v{\phi}^T\p{\v{\xi}} m_i}{\v{\xi}}
  \end{gather}
  Simplifying and solving for \(\M{Q}_{i,t}\) gives
  \begin{gather}
    \M{Q}_{i,t}
    = \dintt{\mcK}{}{\M{f}\p{\M{Q}_i \v{\phi}(\v{\xi}), \v{b}_i(\v{\xi}), t}
      \p{\v{\phi}'(\v{\xi}) \v{c}_i'(\v{b}_i(\v{\xi}))}^T}{\v{\xi}} \M{M}^{-1} \\
    - \dintt{\partial K_i}{}{\M{f}^* \v{n} \v{\phi}_i^T(\v{x})}{s} \M{M}^{-1} \frac{1}{m_i}
    + \dintt{\mcK}{}{\v{s}\p{\M{Q}_i \v{\phi}(\v{\xi}), \v{b}_i(\v{\xi}), t}
      \v{\phi}^T\p{\v{\xi}}}{\v{\xi}} \M{M}^{-1}
  \end{gather}



\section{One Dimension}
  Consider the one dimensional balance law given below.
  \begin{equation}
    \v{q}_t + \v{f}\p{\v{q}, x, t}_x = \v{s}(\v{q}, x, t)
  \end{equation}
  In one dimension the elements are \(K_i = \br{x_{i-1/2}, x_{i+1/2}}\), where the
  center of the element is given by \(x_i\) and
  \(\Delta x_i = \abs{K_i} = x_{i+1/2} - x_{i-1/2}\).
  The canonical element is \(\mcK = \br{-1, 1}\), and the linear transformations are
  \(c_i(x) = \p{x - x_i} \frac{2}{\Delta x_i}\) and
  \(b_i(\xi) = \frac{\Delta x_i}{2} \xi + x_i\).

\section{Two Dimensions}

  In two dimensions the flux function is a matrix function of size \(N_e \times 2\).
  Often it is denoted as two vector functions \(\v{f}_1\) and \(\v{f}_2\) or \(\v{f}\)
  and \(\v{g}\), however I will denote it as the matrix function
  \(\M{f} = \br{\v{f}_1, \v{f}_2} = \br{\v{f}, \v{g}}\).
  Also in two dimensions the boundary integral of the numerical flux is a line integral.
  A line integral can be expressed as a one dimensional integral through a
  parameterization of that line.
  Suppose we have a line \(L(\v{x}) = 0\), that can be parameterized by
  \(\v{l}(t) = \v{x}\) for \(t \in \br{t_1, t_2}\).
  Then the line integral can be written as
  \begin{equation}
    \dintt{L}{}{h(\v{x})}{s} = \dintt{t_1}{t_2}{h(\v{l}(t)) \norm{\v{l}'(t)}}{t}.
  \end{equation}

  In two dimensions the canonical element will have a set of faces, \(\mcF = \set{f_j}\).
  I will have a parameterization of each face of the canonical element, \(r_j(t)\), with
  \(t \in \br{-1, 1}\).
  Having \(t \in \br{-1, 1}\) is convenient as 1D quadrature rules won't need to be
  transformed from their canonical intervals.
  The actual integral is over the faces of the mesh element, so the actual
  parameterization for the faces of the mesh element will be \(\v{b}_i(\v{r}_j(t))\).
  In this way I will handle the transformation to the canonical element and the
  parameterization of the line in one step.
  Therefore the boundary integral of the numerical flux can be written as
  \begin{equation}
    \dintt{\partial K_i}{}{\M{f}^* \v{n} \v{\phi}_i^T(\v{x})}{s}
    = \sum{f_j \in \mcF}{}{\dintt{-1}{1}{\M{f}^* \v{n} \v{\phi}^T(\v{r}_j(t))
      \norm{\v{b}_i'(\v{r}_j(t)) \v{r}_j'(t)}}{t}}
  \end{equation}

  In two dimensions the discontinuous galerkin formulation is therefore
  \begin{gather}
    \M{Q}_{i,t}
    = \dintt{\mcK}{}{\M{f}\p{\M{Q}_i \v{\phi}(\v{\xi}), \v{b}_i(\v{\xi}), t}
      \p{\v{\phi}'(\v{\xi}) \v{c}_i'(\v{b}_i(\v{\xi}))}^T}{\v{\xi}} \M{M}^{-1} \\
    - \sum{f_j \in \mcF}{}{\dintt{-1}{1}{\M{f}^* \v{n} \v{\phi}^T(\v{r}_j(t))
      \norm{\v{b}_i'(\v{r}_j(t)) \v{r}_j'(t)}}{t}} \M{M}^{-1} \frac{1}{m_i}
    + \dintt{\mcK}{}{\v{s}\p{\M{Q}_i \v{\phi}(\v{\xi}), \v{b}_i(\v{\xi}), t}
      \v{\phi}^T\p{\v{\xi}}}{\v{\xi}} \M{M}^{-1}
  \end{gather}

  % Consider the two dimensional balance law given by
  % \begin{equation}
  %   \v{q}_t + \div \v{f}_j\p{\v{q}, \v{x}, t} = \v{s}\p{\v{q}, \v{x}, t}
  % \end{equation}
  % Note that the flux function is a matrix or two index tensor, so the divergence is a
  % vector quantity.
  % It could also be written as
  % \begin{equation}
  %   \v{q}_t + \v{f}_1\p{\v{q}, \v{x}, t}_{x_1} + \v{f}_2\p{\v{q}, \v{x}, t}_{x_2}
  %   = \v{s}\p{\v{q}, \v{x}, t}
  % \end{equation}
  % The local statements of the weak discontinuous Galerkin form are given by
  % \begin{equation}
  %   \dintt{K_i}{}{\v{q}_t \phi_i^k(\v{x})
  %   - \v{f}_j\p{\v{q}, \v{x}, t} \phi^k_{i,x_j}(\v{x})}{\v{x}}
  %   = -\dintt{\partial K_i}{}{\v{n} \cdot \v{f}^*_j \phi_i^k(\v{x})}{s}.
  % \end{equation}
  % Note that all of the terms related to the flux function \(\v{f}\) are sums over the
  % two dimensions.
  % It could also be written as
  % \begin{equation}
  %   \dintt{K_i}{}{\v{q}_t \phi_i^k(\v{x})
  %   - \v{f}_1\p{\v{q}, \v{x}, t} \phi^k_{i, x_1}(\v{x})
  %   - \v{f}_2\p{\v{q}, \v{x}, t} \phi^k_{i, x_2}(\v{x})}{\v{x}}
  %   = -\dintt{\partial K_i}{}{\p{n_1 \v{f}^*_1 + n_2 \v{f}^*_2} \phi_i^k(\v{x})}{s}.
  % \end{equation}
  % We could also consider all of the basis components at one time, by using the test
  % function \(\v{\phi}^T_i\) instead of \(\phi^k_i\).
  % \begin{gather}
  %   \dintt{K_i}{}{\v{q}_t \v{\phi}^T_i(\v{x})
  %   - \v{f}_1\p{\v{q}, \v{x}, t} \v{\phi}^T_{i, x_1}(\v{x})
  %   - \v{f}_2\p{\v{q}, \v{x}, t} \v{\phi}^T_{i, x_2}(\v{x})}{\v{x}}
  %   = -\dintt{\partial K_i}{}{\p{n_1 \v{f}^*_1 + n_2 \v{f}^*_2} \v{\phi}_i^T(\v{x})}{s}.
  %   \intertext{Now using the fact that \(\eval{\v{q}}{K_i}{} = Q_i \v{\phi}_i(\v{x})\)}
  %   \dintt{K_i}{}{Q_{i,t} \v{\phi}_i(\v{x}) \v{\phi}^T_i(\v{x})
  %   - \v{f}_1\p{Q_i \v{\phi}_i(\v{x}), \v{x}, t} \v{\phi}^T_{i, x_1}(\v{x})
  %   - \v{f}_2\p{Q_i \v{\phi}_i(\v{x}), \v{x}, t} \v{\phi}^T_{i, x_2}(\v{x})}{\v{x}}
  %   = -\dintt{\partial K_i}{}{\p{n_1 \v{f}^*_1 + n_2 \v{f}^*_2} \v{\phi}_i^T(\v{x})}{s}.
  %   \intertext{Rearranging to solve for \(Q_{i,t}\), and dropping explicit dependence
  %     on \(\v{x}\) for \(\v{\phi}\)}
  %   Q_{i,t} \dintt{K_i}{}{\v{\phi}_i \v{\phi}^T_i}{\v{x}} =
  %   \dintt{K_i}{}{\v{f}_1\p{Q_i \v{\phi}_i, \v{x}, t} \v{\phi}^T_{i, x_1}
  %   + \v{f}_2\p{Q_i \v{\phi}_i, \v{x}, t} \v{\phi}^T_{i, x_2}}{\v{x}}
  %   -\dintt{\partial K_i}{}{\p{n_1 \v{f}^*_1 + n_2 \v{f}^*_2} \v{\phi}_i^T}{s}
  %   \intertext{Transforming to canonical element}
  %   Q_{i,t} m_i M =
  %   \dintt{K_i}{}{\v{f}_1\p{Q_i \v{\phi}_i, \v{x}, t} \v{\phi}^T_{i, x_1}
  %   + \v{f}_2\p{Q_i \v{\phi}_i, \v{x}, t} \v{\phi}^T_{i, x_2}}{\v{x}}
  %   -\dintt{\partial K_i}{}{\p{n_1 \v{f}^*_1 + n_2 \v{f}^*_2} \v{\phi}_i^T}{s}.
  % \end{gather}

  % \begin{equation}
  %   \dintt{K_i}{}{\v{q}_t \phi_i^k(\v{x})
  %   - \M{f}\p{\v{q}, \v{x}, t} \phi^k_{i,x_j}(\v{x})}{\v{x}}
  %   = -\dintt{\partial K_i}{}{\v{n} \cdot \M{f}^* \phi_i^k(\v{x})}{s}.
  % \end{equation}
  % We could also consider all of the basis components at one time, by using the test
  % function \(\v{\phi}^T_i\) instead of \(\phi^k_i\).
  % \begin{gather}
  %   \dintt{K_i}{}{\v{q}_t \v{\phi}^T_i(\v{x})
  %   - \M{f}\p{\v{q}, \v{x}, t} \M{D\phi}^T_{i}(\v{x})}{\v{x}}
  %   = -\dintt{\partial K_i}{}{\v{n} \cdot \M{f}^*(\v{x}) \v{\phi}_i^T(\v{x})}{s}
  %   \intertext{Now using the fact that \(\eval{\v{q}}{K_i}{} = Q_i \v{\phi}_i(\v{x})\)}
  %   \dintt{K_i}{}{\M{Q}_{i,t} \v{\phi}_i(\v{x}) \v{\phi}^T_i(\v{x})
  %   - \M{f}\p{\M{Q}_i \v{\phi}_i(\v{x}), \v{x}, t} \M{D\phi}^T_{i}(\v{x})}{\v{x}}
  %   = -\dintt{\partial K_i}{}{\v{n} \cdot \M{f}^*(\v{x}) \v{\phi}_i^T(\v{x})}{s}
  %   \intertext{Rearranging to solve for \(Q_{i,t}\), and dropping explicit dependence
  %     on \(\v{x}\) for \(\v{\phi}\)}
  %   \M{Q}_{i, t} \dintt{K_i}{}{\v{\phi}_i \v{\phi}^T_i}{\v{x}}
  %   = \dintt{K_i}{}{\M{f}\p{\M{Q}_i \v{\phi}_i, \v{x}, t} \M{D\phi}^T_i}{\v{x}}
  %   - \dintt{\partial K_i}{}{\v{n} \cdot \M{f}^*\p{\v{x}} \v{\phi}_i^T}{s}
  %   \intertext{Transforming to canonical element}
  %   \M{Q}_{i, t} m_i \M{M}
  %   = \dintt{\mcK}{}{\M{f}\p{\M{Q}_i \v{\phi}, \v{b}_i(\v{\xi}), t} \M{D\phi}^T \M{c}_i' m_i}{\v{\xi}}
  %   - \sum{f \in \mcK}{}{\dintt{}{}{\v{n} \cdot \M{f}^*\p{b_i(r_f(t))} \v{\phi}^T(r_f(t)) \norm{b_i'(r_f(t)) r_f'(t)}}{t}}
  %   \intertext{Then the set of ODEs are}
  %   \M{Q}_{i, t}
  %   = \dintt{\mcK}{}{\M{f}\p{\M{Q}_i \v{\phi}, \v{b}_i(\v{\xi}), t} \M{D\phi}^T \M{c}_i'}{\v{\xi}} \M{M}^{-1}
  %   - \sum{f \in \mcK}{}{\dintt{}{}{\v{n} \cdot \M{f}^*\p{b_i(r_f(t))} \v{\phi}^T(r_f(t)) \norm{b_i'(r_f(t)) r_f'(t)}}{t} \frac{1}{m_i} M^{-1}}
  % \end{gather}

\subsection{Rectangular Elements}
  Consider if the mesh contain rectangular elements, then
  \(K_i = \br{x_{i-1/2}, x_{i+1/2}} \times \br{y_{i-1/2}, y_{i+1/2}}\).
  The center of the element is \(\p{x_i, y_i}\) with
  \(\Delta x_i = x_{i+1/2} - x_{i-1/2}\) and \(\Delta y_i = y_{i+1/2} - y_{i-1/2}\).
  The canonical element is \(\mcK = \br{-1, 1} \times \br{-1, 1}\) with coordinates
  \(\v{\xi} = \br{\xi, \eta}\).
  The linear transformations are given by
  \begin{gather}
    \v{b}_i(\v{\xi}) = \br{\frac{\Delta x_i}{2} \xi + x_i, \frac{\Delta y_i}{2} \eta + y_i}^T \\
    \v{c}_i(\v{x}) = \br{\frac{2}{\Delta x_i} \p{x - x_i}, \frac{2}{\Delta y_i} \p{y - y_i}}^T \\
  \end{gather}
  with Jacobians
  \begin{gather}
    \M{b}_i' =
    \begin{pmatrix}
      \frac{\Delta x_i}{2} & 0 \\
      0 & \frac{\Delta y_i}{2}
    \end{pmatrix} \\
    \M{c}_i' =
    \begin{pmatrix}
      \frac{2}{\Delta x_i} & 0 \\
      0 & \frac{2}{\Delta y_i}
    \end{pmatrix}
  \end{gather}

  The metric of element i is \(m_i = \frac{\Delta x_i \Delta y_i}{4}\).
  Also the parameterizations of the left, right, bottom, and top faces,
  \(r_l, r_r, r_b, r_t\) respectively, are given by
  \begin{gather}
    r_l(t) = \br{-1, t} \\
    r_r(t) = \br{1, t} \\
    r_b(t) = \br{t, -1} \\
    r_t(t) = \br{t, 1}
  \end{gather}
  for \(t \in \br{-1, 1}\).
  We can easily compute \(\norm{\M{b}_i'(\v{r}_f(t)) \v{r}_f'(t)}\) for each face as well
  \begin{gather}
    \norm{\M{b}_i'(\v{r}_l(t)) \v{r}_l'(t)} = \frac{\Delta y_i}{2} \\
    \norm{\M{b}_i'(\v{r}_r(t)) \v{r}_r'(t)} = \frac{\Delta y_i}{2} \\
    \norm{\M{b}_i'(\v{r}_b(t)) \v{r}_b'(t)} = \frac{\Delta x_i}{2} \\
    \norm{\M{b}_i'(\v{r}_t(t)) \v{r}_t'(t)} = \frac{\Delta x_i}{2}
  \end{gather}
  Substituting all these into the formulation gives,
  \begin{gather}
    \M{Q}_{i, t}
    = \dintt{\mcK}{}{\frac{2}{\Delta x_i}\v{f}_1\p{\M{Q}_i \v{\phi}, \v{b}_i(\v{\xi}), t} \v{\phi}^T_{\xi} +\frac{2}{\Delta y_i}\v{f}_2\p{\M{Q}_i \v{\phi}, \v{b}_i(\v{\xi}), t} \v{\phi}^T_{\eta}}{\v{\xi}} \M{M}^{-1} \\
    + \frac{2}{\Delta x_i} \dintt{-1}{1}{\v{f}^*_1\p{b_i(\xi=-1, \eta)} \v{\phi}^T(\xi=-1, \eta)}{t} \M{M}^{-1}\\
    - \frac{2}{\Delta x_i} \dintt{-1}{1}{\v{f}^*_1\p{b_i(\xi=1, \eta)} \v{\phi}^T(\xi=1, \eta)}{t} \M{M}^{-1}\\
    + \frac{2}{\Delta y_i} \dintt{-1}{1}{\v{f}^*_2\p{b_i(\xi, \eta=-1)} \v{\phi}^T(\xi, \eta=-1)}{t} \M{M}^{-1}\\
    - \frac{2}{\Delta y_i} \dintt{-1}{1}{\v{f}^*_2\p{b_i(\xi, \eta=1)} \v{\phi}^T(\xi, \eta=1)}{t} \M{M}^{-1}\\
  \end{gather}
  For the case of a legendre orthogonal basis with orthogonality condition
  \[
    \frac{1}{4}\dintt{\mcK}{}{\phi^i(\v{\xi}) \phi^j(\v{\xi})}{\xi} = \delta_{ij},
  \]
  then the mass matrix and it's inverse become \(M = 4I\) and \(M^{-1} = \frac{1}{4}I\).
  So the full method becomes,
  \begin{gather}
    \M{Q}_{i, t}
    = \dintt{\mcK}{}{\frac{1}{2\Delta x_i}\v{f}_1\p{\M{Q}_i \v{\phi}, \v{b}_i(\v{\xi}), t} \v{\phi}^T_{\xi}
    + \frac{1}{2\Delta y_i}\v{f}_2\p{\M{Q}_i \v{\phi}, \v{b}_i(\v{\xi}), t} \v{\phi}^T_{\eta}}{\v{\xi}} \\
    + \frac{1}{2\Delta x_i} \dintt{-1}{1}{\v{f}^*_1\p{b_i(\xi=-1, \eta)} \v{\phi}^T(\xi=-1, \eta)}{t} \\
    - \frac{1}{2\Delta x_i} \dintt{-1}{1}{\v{f}^*_1\p{b_i(\xi=1, \eta)} \v{\phi}^T(\xi=1, \eta)}{t} \\
    + \frac{1}{2\Delta y_i} \dintt{-1}{1}{\v{f}^*_2\p{b_i(\xi, \eta=-1)} \v{\phi}^T(\xi, \eta=-1)}{t} \\
    - \frac{1}{2\Delta y_i} \dintt{-1}{1}{\v{f}^*_2\p{b_i(\xi, \eta=1)} \v{\phi}^T(\xi, \eta=1)}{t} \\
  \end{gather}

\subsection{Triangular Elements}
  Consider a mesh with triangular elements.
  That is each mesh element is given by three vertices in \(\RR^2\),
  \(\set{\v{v}_1, \v{v}_2, \v{v}_3}\).
  The coordinates of each vertex are given by \(\v{v}_i = \br{x_i, y_i}\).
  The canonical element that I will use is a right triangle with vertices,
  \(\br{-1, 1}\), \(\br{-1, -1}\), \(\br{1, -1}\).
  The linear transformations between mesh elements and the canonical element are
  given by
  \begin{gather}
    \v{b}_i(\v{\xi}) = \br{b_{00} \xi + b_{01} \eta + b_{02}, b_{10} \xi + b_{11} \eta + b_{12}} \\
    \v{c}_i(\v{x}) = \br{c_{00} x + c_{01} y + c_{02}, c_{10} x + c_{11} y + c_{12}}
  \end{gather}
  where the coefficients are
  \begin{gather}
    b_{00} = \frac{1}{2} \p{x_3 - x_2} \\
    b_{01} = \frac{1}{2} \p{x_1 - x_2} \\
    b_{02} = \frac{1}{2} \p{x_1 + x_3} \\
    b_{10} = \frac{1}{2} \p{y_3 - y_2} \\
    b_{11} = \frac{1}{2} \p{y_1 - y_2} \\
    b_{12} = \frac{1}{2} \p{y_1 + y_3} \\
    c_{00} = \frac{-2(y_1 - y_2)}{y_1 (x_2 - x_3) - x_1 (y_2 - y_3) + y_2 x_3 - x_2 y_3} \\
    c_{01} = \frac{2(x_1 - x_2)}{y_1 (x_2 - x_3) - x_1 (y_2 - y_3) + y_2 x_3 - x_2 y_3} \\
    c_{02} = \frac{y_1 (x_2 + x_3) - x_1 (y_2 + y_3) - y_2 x_3 + x_2 y_3}{y_1 (x_2 -
      x_3) - x_1 (y_2 - y_3) + y_2 x_3 - x_2 y_3} \\
    c_{10} = \frac{-2(y_2 - y_3)}{y_1 (x_2 - x_3) - x_1 (y_2 - y_3) + y_2 x_3 - x_2 y_3} \\
    c_{11} = \frac{2(x_2 - x_3)}{y_1 (x_2 - x_3) - x_1 (y_2 - y_3) + y_2 x_3 - x_2 y_3} \\
    c_{12} = \frac{x_1 (y_2 - y_3) - y_1 (x_2 - x_3) + y_2 x_3 - x_2 y_3}{y_1 (x_2 -
      x_3) - x_1 (y_2 - y_3) + y_2 x_3 - x_2 y_3} \\
  \end{gather}
  These coefficients were found by doing a linear solve such that the vertices of the
  mesh element would be transformed to the vertices of the canonical element.

  The jacobians of the linear transformations are
  \begin{gather}
    \v{b}_i'(\v{\xi}) =
    \begin{pmatrix}
      b_{00} & b_{01} \\
      b_{10} & b_{11}
    \end{pmatrix} \\
    \v{c}_i'(\v{x}) =
    \begin{pmatrix}
      c_{00} & c_{01} \\
      c_{10} & c_{11}
    \end{pmatrix}
  \end{gather}
  The metric of the element will be
  \(m_i = \det{\v{b}_i'} = b_{00}b_{11} - b_{10}b_{01}\).

  Also we can parameterize the left, bottom and hypotenuse faces of the canonical
  element as
  \begin{gather}
    r_l(t) = \br{-1, t} \\
    r_b(t) = \br{t, -1} \\
    r_h(t) = \br{t, -t} \\
  \end{gather}
  for \(t \in \br{-1, 1}\).
  We can easily compute \(\norm{\v{b}_i'(\v{r}_f(t)) \v{r}_f'(t)}\) for each face as
  well
  \begin{gather}
    \norm{\v{b}_i'(\v{r}_l(t)) \v{r}_l'(t)} = \sqrt{b_{01}^2 + b_{11}^2} \\
    \norm{\v{b}_i'(\v{r}_b(t)) \v{r}_b'(t)} = \sqrt{b_{00}^2 + b_{10}^2} \\
    \norm{\v{b}_i'(\v{r}_h(t)) \v{r}_h'(t)} =
      \sqrt{\p{b_{00} - b_{01}}^2 + \p{b_{10} - b_{11}}^2}
  \end{gather}

  \begin{gather}
    \M{Q}_{i,t}
    = \dintt{\mcK}{}{\M{f}\p{\M{Q}_i \v{\phi}(\v{\xi}), \v{b}_i(\v{\xi}), t}
      \p{\v{\phi}'(\v{\xi}) \v{c}_i'(\v{b}_i(\v{\xi}))}^T}{\v{\xi}} \M{M}^{-1} \\
    - \sum{f_j \in \mcF}{}{\dintt{-1}{1}{\M{f}^* \v{n} \v{\phi}^T(\v{r}_j(t))
      \norm{\v{b}_i'(\v{r}_j(t)) \v{r}_j'(t)}}{t}} \M{M}^{-1} \frac{1}{m_i}
    + \dintt{\mcK}{}{\v{s}\p{\M{Q}_i \v{\phi}(\v{\xi}), \v{b}_i(\v{\xi}), t}
      \v{\phi}^T\p{\v{\xi}}}{\v{\xi}} \M{M}^{-1}
  \end{gather}

  For the case of an orthonormal modal basis with orthogonality condition,
  \begin{gather}
    \frac{1}{2} \dintt{\mcK}{}{\phi^i(\v{\xi}) \phi^j(\v{\xi})}{\v{\xi}} = \delta_{ij}
  \end{gather}
  then the mass matrix and it's inverse will be \(\M{M} = 2I\) and
  \(\M{M}^{-1} = \frac{1}{2} I\).

\end{document}