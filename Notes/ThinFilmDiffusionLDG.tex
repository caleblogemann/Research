\documentclass[11pt, oneside]{article}
\usepackage[letterpaper, margin=2cm]{geometry}
\usepackage{Notes}

\begin{document}
\begin{center}
\textbf{\Large{Local Discontinuous Galerkin Method for Thin Film Diffusion
}}
\end{center}

We would like to solve the 1D thin film diffusion equation with a Discontinuous
Galerkin Method.
The equation is given as
\[
  u_t = - \p{u^3 u_{xxx}}_x.
\]

%If we were to naively apply DG methods, we could discretize the domain and
%consider piecewise polynomial approximation.
%We would then multiply by a test function and integrate by parts.
%\begin{gather*}
  %\dintt{I_j}{}{u_t v}{x} = \dintt{I_j}{}{u_{xx} v}{x} \\
  %\dintt{I_j}{}{u_t v}{x} = \p{\p{\hat{u}_x v^-}_{j+1/2} - \p{\hat{u}_x v^+}_{j-1/2}} - \dintt{I_j}{}{u_x v_x}{x}
  %%\dintt{I_j}{}{u_t v}{x} = \p{\p{\hat{u}_x v^-}_{j+1/2} - \p{\hat{u}_x v^+}_{j-1/2}} 
    %%- \p{\p{\hat{u} v^-_x}_{j+1/2} - \p{\hat{u} v^+_x}_{j-1/2}}
    %%+ \dintt{I_j}{}{u v_{xx}}{x}
%\end{gather*}
%and we use the average numerical flux
%\begin{gather*}
  %\p{\hat{u}_x}_{j+1/2} = \frac{\p{u^-_x}_{j+1/2} + \p{u^+_x}_{j+1/2}}{2}
%\end{gather*}
%This method is convergent and stable but it converges to the wrong solution.

\textbf{\large{Local Discontinuous Galerkin Method}}
%The Local Discontinuous Galerkin method proposes a different approach.
First rewrite the diffusion equation as a system of first order equations.
\begin{align*}
  q &= u_x \\
  r &= q_x \\
  s &= u^3 r_x \\
  u_t &= - s_x
\end{align*}

The LDG method becomes the process of finding $u_h, q_h, r_h, s_h \in V_h$ in
the DG solution space, such that for all test functions
$v_h, w_h, y_h, z_h \in V_h$ and for all $j$ the following equations are
satisfied
\begin{align*}
  \dintt{I_j}{}{q_h w_h}{x} &= \dintt{I_j}{}{(u_h)_x w_h}{x} \\
  \dintt{I_j}{}{r_h y_h}{x} &= \dintt{I_j}{}{(q_h)_x y_h}{x} \\
  \dintt{I_j}{}{s_h z_h}{x} &= \dintt{I_j}{}{u_h^3 (r_h)_x z_h}{x} \\
  \dintt{I_j}{}{(u_h)_t v_h}{x} &= -\dintt{I_j}{}{(s_h)_x v_h}{x}
\end{align*}
After integrating by parts, these equations are
\begin{align*}
  \dintt{I_j}{}{q_h w_h}{x} &= \p{\p{\hat{u}_h w^-_h}_{j+1/2} - \p{\hat{u}_j w^+_h}_{j-1/2}} - \dintt{I_j}{}{u_h (w_h)_x}{x} \\
  \dintt{I_j}{}{r_h y_h}{x} &= \p{\p{\hat{q}_h y^-_h}_{j+1/2} - \p{\hat{q}_j y^+_h}_{j-1/2}} - \dintt{I_j}{}{q_h (y_h)_x}{x} \\
  \dintt{I_j}{}{s_h z_h}{x} &= \dintt{I_j}{}{u_h^3 (r_h)_x z_h}{x} \\
  \dintt{I_j}{}{s_h z_h}{x} &= \p{\p{\hat{r}_h z^-_h}_{j+1/2} - \p{\hat{r}_j z^+_h}_{j-1/2}} - \dintt{I_j}{}{u_h^3 r_h (z_h)_x}{x} \\
  \dintt{I_j}{}{(u_h)_t v_h}{x} &= -\p{\p{\hat{s}_h v^-_h}_{j+1/2} - \p{\hat{s}_h v^+_h}_{j-1/2}} + \dintt{I_j}{}{s_h (v_h)_x}{x}
\end{align*}

A common choice of numerical fluxes are the so-called alternating fluxes.
\begin{align*}
  \hat{q}_h &= q^+_h \\
  \hat{u}_h &= u^-_h
\end{align*}

\textbf{\large{Implementation}}
If we consider a single cell $I_j$, do a linear transformation from
$x \in \br{x_{j-1/2}, x_{j+1/2}}$ to $\xi \in \br{-1, 1}$, and consider
specifically the Legendre polynomial basis $\set{\phi^k(\xi)}$ with the
following orthogonality property
\[
  \frac{1}{2}\dintt{-1}{1}{\phi^j(\xi) \phi^k(\xi)}{\xi} = \delta_{jk}
\]
we can form a more concrete LDG method for implementing.
The linear transformation can be expressed as
\[
  x = \frac{\Delta x}{2} \xi + \frac{x_{j-1/2} + x_{j+1/2}}{2}
\]
or
\[
  \xi = \frac{2}{\Delta x} \p{x - \frac{x_{j-1/2} + x_{j+1/2}}{2}}
\]
After this tranformation the diffusion equation become
\[
  u_t = \frac{4}{\Delta x^2} u_{\xi\xi}
\]
on the cell $I_j$.
We can then write this as the following system of first order equations.
\begin{align*}
  u_t &= \frac{2}{\Delta x} q_{\xi} \\
  q &= \frac{2}{\Delta x} u_{\xi}
\end{align*}
With the Legendre basis, the numerical solution on $I_j$ can be written as
\begin{align*}
  u &\approx u_h = \sum{k = 1}{M}{U_k \phi^k(\xi)} \\
  q &\approx q_h = \sum{k = 1}{M}{Q_k \phi^k(\xi)}
\end{align*}
Now plugging these into the system and multiplying by a Legendre basis and integrating gives.
\begin{align*}
  q_h &= \frac{2}{\Delta x} (u_h)_{\xi} \\
  \frac{1}{2} \dintt{-1}{1}{q_h \phi^l}{\xi} &= \frac{1}{\Delta x} \dintt{-1}{1}{(u_h)_{\xi} \phi^l}{\xi} \\
  Q_l &= -\frac{1}{\Delta x} \dintt{-1}{1}{u_h \phi^l_{\xi}}{\xi} + \frac{1}{\Delta x}\p{u^-_{j+1/2} \phi^l(1) - u^-_{j-1/2} \phi^l(-1)} \\
  (u_h)_t &= \frac{2}{\Delta x} (q_h)_{\xi} \\
  \frac{1}{2}\dintt{-1}{1}{(u_h)_t \phi^l}{\xi} &= \frac{1}{\Delta x} \dintt{-1}{1}{(q_h)_{\xi} \phi^l}{\xi} \\
  \dot{U}_l &= -\frac{1}{\Delta x} \dintt{-1}{1}{q_h \phi^l_{\xi}}{\xi} + \frac{1}{\Delta x}\p{q^+_{j+1/2} \phi^l(1) - q^+_{j-1/2} \phi^l(-1)}
\end{align*}
Now this is a system of ODEs, there are $M \times N$ ODEs if $M$ is the spacial
order and $N$ is the number of cells.

\textbf{\large{Proving Stability}}
In order to prove that this method is $L^2$ stable consider we sum both of the integral equations from before.
\begin{gather*}
  \dintt{I_j}{}{(u_h)_t v_h}{x} + \dintt{I_j}{}{q_h w_h}{x} = \p{\p{q^+_h v^-_h}_{j+1/2} - \p{q^+_h v^+_h}_{j-1/2}} \\
  + \p{\p{u^-_h w^-_h}_{j+1/2} - \p{u^-_j w^+_h}_{j-1/2}} - \dintt{I_j}{}{q_h (v_h)_x}{x} - \dintt{I_j}{}{u_h (w_h)_x}{x}
\end{gather*}
Consider using $v_h = u_h$ and $w_h = q_h$.
\begin{gather*}
  \dintt{I_j}{}{(u_h)_t u_h}{x} + \dintt{I_j}{}{q_h q_h}{x} = \p{\p{q^+_h u^-_h}_{j+1/2} - \p{q^+_h u^+_h}_{j-1/2}} \\
  + \p{\p{u^-_h q^-_h}_{j+1/2} - \p{u^-_h q^+_h}_{j-1/2}} - \dintt{I_j}{}{q_h (u_h)_x}{x} - \dintt{I_j}{}{u_h (q_h)_x}{x}
\end{gather*}
Consider the following shorthand notation
\begin{gather*}
  B_j = \dintt{I_j}{}{(u_h)_t u_h}{x} + \dintt{I_j}{}{q_h q_h}{x} \\
  B_j = \p{\p{q^+_h u^-_h}_{j+1/2} - \p{q^+_h u^+_h}_{j-1/2}} + \p{\p{u^-_h q^-_h}_{j+1/2} - \p{u^-_h q^+_h}_{j-1/2}} - \dintt{I_j}{}{q_h (u_h)_x}{x} - \dintt{I_j}{}{u_h (q_h)_x}{x}
\end{gather*}
%\begin{gather*}
  %B_j(u_h, q_h, u_h, q_h) = \dintt{I_j}{}{(u_h)_t u_h}{x} - \p{\p{\hat{q}_h u^-_h}_{j+1/2} - \p{\hat{q}_h u^+_h}_{j-1/2}} + \dintt{I_j}{}{q_h (u_h)_x}{x} \\
  %+ \dintt{I_j}{}{q_h q_h}{x} - \p{\p{\hat{u}_h q^-_h}_{j+1/2} - \p{\hat{u}_j q^+_h}_{j-1/2}} + \dintt{I_j}{}{u_h (q_h)_x}{x} = 0
%\end{gather*}
This can be simplified in several ways.
First simplify the left hand side.
\begin{align*}
  B_j &= \dintt{I_j}{}{(u_h)_t u_h}{x} + \dintt{I_j}{}{q_h q_h}{x} \\
  B_j &= \frac{1}{2} \dintt{I_j}{}{\d*{u_h^2}{t}}{x} + \dintt{I_j}{}{q_h^2}{x} \\
  B_j &= \frac{1}{2} \d{}{t} \dintt{I_j}{}{u_h^2}{x} + \dintt{I_j}{}{q_h^2}{x} \\
  B_j &= \frac{1}{2} \d{}{t} \norm[L^2(I_j)]{u_h}^2 + \norm[L^2(I_j)]{q_h}^2 \\
\end{align*}
Second the right hand side can be simplified.
\begin{align*}
  \dintt{I_j}{}{q_h (u_h)_x}{x} + \dintt{I_j}{}{u_h (q_h)_x}{x} &= \dintt{I_j}{}{q_h (u_h)_x + u_h (q_h)_x}{x} \\
  &= \dintt{I_j}{}{(q_h u_h)_x}{x} \\
  &= (q^-_h u^-_h)_{j+1/2} - (q^+_h u^+_h)_{j-1/2}
\end{align*}
Now 
\begin{gather*}
  B_j = \p{\p{q^+_h u^-_h}_{j+1/2} - \p{q^+_h u^+_h}_{j-1/2}} + \p{\p{u^-_h q^-_h}_{j+1/2} - \p{u^-_h q^+_h}_{j-1/2}} - \p{(q^-_h u^-_h)_{j+1/2} - (q^+_h u^+_h)_{j-1/2}} \\
  B_j = \p{q^+_h u^-_h}_{j+1/2} - \p{u^-_h q^+_h}_{j-1/2} \\
\end{gather*}
Assuming periodic boundary conditions, and summing $B_j$ over all cells
\begin{align*}
  \sum{j = 1}{N}{B_j} &= \sum{j=1}{N}{\p{q^+_h u^-_h}_{j+1/2} - \p{u^-_h q^+_h}_{j-1/2}} \\
  &= -\p{u^-_h q^+_h}_{1/2} + \sum{k=1}{N}{\p{q^+_h u^-_h}_{k+1/2} - \p{u^-_h q^+_h}_{k+1/2}} + \p{q^+_h u^-_h}_{N+1/2} \\
  &= 0
\end{align*}
This shows that
\begin{gather*}
  \sum{j = 1}{N}{B_j} = \sum{j = 1}{N}{\frac{1}{2} \d{}{t} \norm[L^2(I_j)]{u_h}^2 + \norm[L^2(I_j)]{q_h}^2} \\
  \frac{1}{2} \d{}{t} \norm[L^2]{u_h}^2 + \norm[L^2]{q_h}^2 = 0 \\
  \d{}{t} \norm[L^2]{u_h}^2 \le 0
\end{gather*}
\end{document}
