\documentclass[11pt, oneside]{article}
\usepackage[letterpaper, margin=2cm]{geometry}
\usepackage{Notes}

\begin{document}
\begin{center}
\textbf{\Large{Local Discontinuous Galerkin Method for Thin Film Diffusion
}}
\end{center}

  We would like to solve the 1D thin film diffusion equation with a Discontinuous
  Galerkin Method.
  The equation is given as
  \[
    q_t = - \p{q^3 q_{xxx}}_x.
  \]

\textbf{\large{Non Dimensional Form}}
  Let $q = Hq$, $x = Lx$, and $t = Tt$, then
  \begin{gather}
    \frac{H}{T} q_t = - \frac{1}{L} \p{H^3 q^3 \frac{H}{L^3}q_{xxx}}_x \\
    q_t = - \frac{TH^3}{L^4} \p{q^3 q_{xxx}}_x \\
  \end{gather}

\textbf{\large{Local Discontinuous Galerkin Method}}
  First rewrite the diffusion equation as a system of first order equations.
  \begin{align*}
    r &= q_x \\
    s &= r_x = q_{xx}\\
    u &= q^3 s_x = q^3 q_{xxx} \\
    q_t &= - u_x = -\p{q^3 q_{xxx}}_x
  \end{align*}

  The LDG method becomes the process of finding $q_h, r_h, s_h, u_h \in V_h$ in
  the DG solution space, such that for all test functions
  $v_h, w_h, y_h, z_h \in V_h$ and for all $j$ the following equations are
  satisfied
  \begin{align*}
    \dintt{I_j}{}{r_h w_h}{x} &= \dintt{I_j}{}{(q_h)_x w_h}{x} \\
    \dintt{I_j}{}{s_h y_h}{x} &= \dintt{I_j}{}{(r_h)_x y_h}{x} \\
    \dintt{I_j}{}{u_h z_h}{x} &= \dintt{I_j}{}{q_h^3 (s_h)_x z_h}{x} \\
    \dintt{I_j}{}{(q_h)_t v_h}{x} &= -\dintt{I_j}{}{(u_h)_x v_h}{x}
  \end{align*}
  After integrating by parts, these equations are
  \begin{align*}
    \dintt{I_j}{}{r_h w_h}{x} &= \p{\p{\hat{q}_h w^-_h}_{j+1/2} - \p{\hat{q}_j w^+_h}_{j-1/2}} - \dintt{I_j}{}{q_h (w_h)_x}{x} \\
    \dintt{I_j}{}{s_h y_h}{x} &= \p{\p{\hat{r}_h y^-_h}_{j+1/2} - \p{\hat{r}_j y^+_h}_{j-1/2}} - \dintt{I_j}{}{r_h (y_h)_x}{x} \\
    \dintt{I_j}{}{(q_h)_t v_h}{x} &= -\p{\p{\hat{u}_h v^-_h}_{j+1/2} - \p{\hat{u}_h v^+_h}_{j-1/2}} + \dintt{I_j}{}{u_h (v_h)_x}{x}
  \end{align*}
  The third equation is trickier and requires integrating by parts twice.
  \begin{align*}
    \dintt{I_j}{}{u_h z_h}{x} &= \dintt{I_j}{}{q_h^3 (r_h)_x z_h}{x} \\
    \dintt{I_j}{}{u_h z_h}{x} &= \p{\p{\hat{r}_h \hat{q_h^3} z^-_h}_{j+1/2} - \p{\hat{r}_j \hat{q_h^3} z^+_h}_{j-1/2}} - \dintt{I_j}{}{r_h (q_h^3 z_h)_x}{x} \\
    \dintt{I_j}{}{u_h z_h}{x} &= \p{\p{\hat{r}_h \hat{q_h^3} z^-_h}_{j+1/2} - \p{\hat{r}_j \hat{q_h^3} z^+_h}_{j-1/2}} \\
    &-\p{\p{\hat{r}_h \p{q^-_h}^3 z^-_h}_{j+1/2} - \p{\hat{r}_j \p{q^+_h}^3 z^+_h}_{j-1/2}} + \dintt{I_j}{}{(r_h)_x q_h^3 z_h}{x} \\
  \end{align*}

  A common choice of numerical fluxes are the so-called alternating fluxes.
  \begin{align*}
    \hat{u}_h &= u^-_h \\
    \hat{q}_h &= q^+_h \\
    \hat{r}_h &= r^-_h \\
    \hat{s}_h &= s^+_h
  \end{align*}
  or
  \begin{align*}
    \hat{u}_h &= u^+_h \\
    \hat{q}_h &= q^-_h \\
    \hat{r}_h &= r^+_h \\
    \hat{s}_h &= s^-_h
  \end{align*}

\textbf{\large{Implementation}}
  If we consider a single cell $I_j$, do a linear transformation from
  $x \in \br{x_{j-1/2}, x_{j+1/2}}$ to $\xi \in \br{-1, 1}$, and consider
  specifically the Legendre polynomial basis $\set{\phi^k(\xi)}$ with the
  following orthogonality property
  \[
    \frac{1}{2}\dintt{-1}{1}{\phi^j(\xi) \phi^k(\xi)}{\xi} = \delta_{jk}
  \]
  we can form a more concrete LDG method for implementing.
  The linear transformation can be expressed as
  \[
    x = \frac{\Delta x}{2} \xi + \frac{x_{j-1/2} + x_{j+1/2}}{2}
  \]
  or
  \[
    \xi = \frac{2}{\Delta x} \p{x - \frac{x_{j-1/2} + x_{j+1/2}}{2}}
  \]
  After this tranformation the thin film diffusion equation become
  \[
    u_t = -\frac{16}{\Delta x^4} \p{u^3 u_{\xi\xi\xi}}_{\xi}
  \]
  on the cell $I_j$.
  We can then write this as the following system of first order equations.
  \begin{align*}
    r &= \frac{2}{\Delta x} q_{\xi} \\
    s &= \frac{2}{\Delta x} r_{\xi} = \frac{4}{\Delta x^2} q_{\xi\xi} \\
    u &= \frac{2}{\Delta x} q^3 s_{\xi} = \frac{8}{\Delta x^3} q^3 q_{\xi\xi\xi} \\
    q_t &= -\frac{2}{\Delta x} u_{\xi} = - \frac{16}{\Delta x^4} \p{q^3 q_{\xi\xi\xi}}_{\xi}
  \end{align*}
  With the Legendre basis, the numerical solution on $I_i$ can be written as
  \begin{align*}
    \eval{q}{I_i}{} &\approx \eval{q_h}{I_i}{} = \sum{l = 1}{M}{Q_i^l \phi^l(\xi)} \\
    \eval{r}{I_i}{} &\approx \eval{r_h}{I_i}{} = \sum{l = 1}{M}{R_i^l \phi^l(\xi)} \\
    \eval{s}{I_i}{} &\approx \eval{s_h}{I_i}{} = \sum{l = 1}{M}{S_i^l \phi^l(\xi)} \\
    \eval{u}{I_i}{} &\approx \eval{u_h}{I_i}{} = \sum{l = 1}{M}{U_i^l \phi^l(\xi)}
  \end{align*}
  Now plugging these into the system and multiplying by a Legendre basis and integrating over cell $I_i$ gives.
  I will use the following shorthand for numerical fluxes using one of the alternating flux options.
  \begin{align*}
    \hat{Q}_{i+1/2} &= Q^+_{i+1/2} = \sum{l = 1}{M}{Q_{i+1}^l \phi^l(-1)} \\
    \hat{R}_{i+1/2} &= R^-_{i+1/2} = \sum{l = 1}{M}{R_{i}^l \phi^l(1)} \\
    \hat{S}_{i+1/2} &= S^+_{i+1/2} = \sum{l = 1}{M}{S_{i+1}^l \phi^l(-1)} \\
    \hat{U}_{i+1/2} &= U^-_{i+1/2} = \sum{l = 1}{M}{U_{i}^l \phi^l(1)}
  \end{align*}
  \begin{align*}
    r &= \frac{2}{\Delta x} q_{\xi} \\
    \sum{l = 1}{M}{R_i^l \phi^l(\xi)} &= \frac{2}{\Delta x} \sum{l = 1}{M}{Q_i^l \phi_{\xi}^l(\xi)} \\
    \frac{1}{2}\dintt{-1}{1}{\sum{l = 1}{M}{R_i^l \phi^l(\xi)} \phi^k(\xi)}{\xi} &= \frac{1}{\Delta x} \dintt{-1}{1}{\sum{l = 1}{M}{Q_i^l \phi_{\xi}^l(\xi)} \phi^k(\xi)}{\xi} \\
    R_i^k &= \frac{1}{\Delta x} \dintt{-1}{1}{\sum{l = 1}{M}{Q_i^l \phi_{\xi}^l(\xi)} \phi^k(\xi)}{\xi} \\
    R_i^k &= -\frac{1}{\Delta x} \dintt{-1}{1}{\sum{l = 1}{M}{Q_i^l \phi^l(\xi)} \phi_{\xi}^k(\xi)}{\xi} + \frac{1}{\Delta x}\p{\phi^k(1)\hat{Q}_{i+1/2} - \phi^k(-1)\hat{Q}_{i-1/2}} \\
  \end{align*}
  \begin{align*}
    s &= \frac{2}{\Delta x} r_{\xi} \\
    \sum{l = 1}{M}{S_i^l \phi^l(\xi)} &= \frac{2}{\Delta x} \sum{l = 1}{M}{R_i^l \phi_{\xi}^l(\xi)} \\
    \frac{1}{2}\dintt{-1}{1}{\sum{l = 1}{M}{S_i^l \phi^l(\xi)} \phi^k(\xi)}{\xi} &= \frac{1}{\Delta x} \dintt{-1}{1}{\sum{l = 1}{M}{R_i^l \phi_{\xi}^l(\xi)} \phi^k(\xi)}{\xi} \\
    S_i^k &= -\frac{1}{\Delta x} \dintt{-1}{1}{\sum{l = 1}{M}{R_i^l \phi^l(\xi)} \phi_{\xi}^k(\xi)}{\xi} + \frac{1}{\Delta x}\p{\phi^k(1)\hat{R}_{i+1/2} - \phi^k(-1)\hat{R}_{i-1/2}} \\
  \end{align*}
  This is the computation that requires integrating by parts twice.
  On the first integrating by parts I will treat the numerical fluxes normally,
  that is I will use the alternating fluxes for the flux of $S$ and an average
  flux for the flux of $q^3$.
  This allows for the propogation of information over cell interfaces.
  For the second integration by parts I will treat the integrand as existsing
  solely within the cell and thus I will use interior fluxes for everything.
  For shorthand I will use
  \[
    \hat{q}^3_{i+1/2} = \p{\frac{\p{q^+_{i+1/2}}^3 + \p{q^-_{i+1/2}}^3}{2}}
  \]
  and I will explicitly write the interior fluxes as
  \[
    \p{q^-_{i+1/2}}^3 \qquad \p{q^+_{i-1/2}}^3
  \]
  \begin{align*}
    u &= \frac{2}{\Delta x} q^3 s_{\xi} \\
    \sum{l = 1}{M}{U_i^l \phi^l(\xi)} &= \frac{2}{\Delta x} q^3 \sum{l = 1}{M}{S_i^l \phi_{\xi}^l(\xi)} \\
    \frac{1}{2}\dintt{-1}{1}{\sum{l = 1}{M}{U_i^l \phi^l(\xi)}\phi^k(\xi)}{\xi} &= \frac{1}{\Delta x} \dintt{-1}{1}{q^3 \sum{l = 1}{M}{S_i^l \phi_{\xi}^l(\xi)} \phi^k(\xi)}{\xi} \\
    U_i^k &= -\frac{1}{\Delta x} \dintt{-1}{1}{\sum{l = 1}{M}{S_i^l \phi^l(\xi)} \p{q^3\phi^k(\xi)}_{\xi}}{\xi} + \frac{1}{\Delta x}\p{\phi^k(1)\hat{q}^3_{i+1/2}\hat{S}_{i+1/2} - \phi^k(-1)\hat{q}^3_{i-1/2}\hat{S}_{i-1/2}} \\
    U_i^k &= \frac{1}{\Delta x} \dintt{-1}{1}{\sum{l = 1}{M}{S_i^l \phi_{\xi}^l(\xi)} q^3\phi^k(\xi)}{\xi} \\
    &- \frac{1}{\Delta x}\p{\phi^k(1)\p{q^-_{i+1/2}}^3 S^-_{i+0/2} - \phi^k(-1)\p{q^+_{i-1/2}}^3 S^+_{i-1/2}} \\
    &+ \frac{1}{\Delta x}\p{\phi^k(1)\hat{q}^3_{i+1/2}S^+_{i+1/2} - \phi^k(-1)\hat{q}^3_{i-1/2}S^+_{i-1/2}} \\
    U_i^k &= \frac{1}{\Delta x} \dintt{-1}{1}{\sum{l = 1}{M}{S_i^l \phi_{\xi}^l(\xi)} q^3\phi^k(\xi)}{\xi} \\
    &- \frac{1}{\Delta x}\phi^k(1)\p{q^-_{i+1/2}}^3 S^-_{i+1/2} \\
    &+ \frac{1}{\Delta x}\phi^k(-1)\p{\p{q^+_{i-1/2}}^3 - \hat{q}^3_{i-1/2}} S^+_{i-1/2} \\
    &+ \frac{1}{\Delta x}\phi^k(1)\hat{q}^3_{i+1/2} S^+_{i+1/2}
  \end{align*}
  \begin{align*}
    q_t &= -\frac{2}{\Delta x} u_{\xi} \\
    \sum{l = 1}{M}{\dot{Q}_i^l \phi^l(\xi)}  &= -\frac{2}{\Delta x} \sum{l = 1}{M}{U_i^l \phi_{\xi}^l(\xi)} \\
    \frac{1}{2}\dintt{-1}{1}{\sum{l = 1}{M}{\dot{Q}_i^l \phi^l(\xi)} \phi^k(\xi)}{\xi}  &= -\frac{1}{\Delta x} \dintt{-1}{1}{\sum{l = 1}{M}{U_i^l \phi_{\xi}^l(\xi)}\phi^k(\xi)}{\xi} \\
    \dot{Q}_i^k &= \frac{1}{\Delta x} \dintt{-1}{1}{\sum{l = 1}{M}{U_i^l \phi_{\xi}^l(\xi)}\phi^k(\xi)}{\xi} - \frac{1}{\Delta x}\p{\phi^k(1)\hat{U}_{i+1/2} - \phi^k(-1)\hat{U}_{i-1/2}} \\
  \end{align*}
  
  %\begin{align*}
    %q_h &= \frac{2}{\Delta x} (u_h)_{\xi} \\
    %\frac{1}{2} \dintt{-1}{1}{q_h \phi^l}{\xi} &= \frac{1}{\Delta x} \dintt{-1}{1}{(u_h)_{\xi} \phi^l}{\xi} \\
    %Q_l &= -\frac{1}{\Delta x} \dintt{-1}{1}{u_h \phi^l_{\xi}}{\xi} + \frac{1}{\Delta x}\p{u^-_{j+1/2} \phi^l(1) - u^-_{j-1/2} \phi^l(-1)} \\
    %(u_h)_t &= \frac{2}{\Delta x} (q_h)_{\xi} \\
    %\frac{1}{2}\dintt{-1}{1}{(u_h)_t \phi^l}{\xi} &= \frac{1}{\Delta x} \dintt{-1}{1}{(q_h)_{\xi} \phi^l}{\xi} \\
    %\dot{U}_l &= -\frac{1}{\Delta x} \dintt{-1}{1}{q_h \phi^l_{\xi}}{\xi} + \frac{1}{\Delta x}\p{q^+_{j+1/2} \phi^l(1) - q^+_{j-1/2} \phi^l(-1)}
  %\end{align*}
  Now this is a system of ODEs, there are $M \times N$ ODEs if $M$ is the spacial
  order and $N$ is the number of cells.

\textbf{\large{Matrix Representation}}
  Some common matrices and vectors that appear in these equations are
  \begin{align*}
    \v{Q}_{i} &= \br{Q_i^l}_{l=1}^M \\
    \v{\phi}(\xi) &= \br{\phi^k(\xi)}_{k = 1}^{M} \\
    \Phi(\xi_1,\xi_2) &= \v{\phi}(\xi_1) \v{\phi}^T(\xi_2) \\
    A &= \br{a_{kl}}_{k,l=1}^M \\
    a_{kl} &= \dintt{-1}{1}{\phi_{\xi}^k(\xi) \phi^l(\xi)}{\xi} \\
    B_i &= \br{b_{kl}}_{k,l=1}^M \\
    b_{kl} &= \dintt{-1}{1}{q_i^3(\xi) \phi^k(\xi) \phi_{\xi}^l(\xi)}{\xi}
  \end{align*}
  For example if $M = 5$, then
  \[
    \v{\phi}(\xi) =
    \begin{bmatrix}
      \phi^1(\xi) \\
      \phi^2(\xi) \\
      \phi^3(\xi) \\
      \phi^4(\xi) \\
      \phi^5(\xi)
    \end{bmatrix}
  \]
  and
  \[
    A =
    \begin{bmatrix}
      0 & 0 & 0 & 0 & 0 \\
      2 \sqrt{3} & 0 & 0 & 0 & 0 \\
      0 & 2 \sqrt{3}\sqrt{5} & 0 & 0 & 0 \\
      2 \sqrt{7} & 0 & 2 \sqrt{5}\sqrt{7} & 0 & 0 \\
      0 & 6 \sqrt{3} & 0 & 6 \sqrt{7} & 0 \\
    \end{bmatrix}
  \]
  Also the numerical fluxes can be written as the following dot products
  \begin{align*}
    \hat{Q}_{i+1/2} &= \sum{l = 1}{M}{Q_{i+1}^l \phi^l(-1)} \\
    &= \v{\phi}^T(-1)\v{Q}_{i+1} \\
    \hat{R}_{i+1/2} &= \sum{l = 1}{M}{R_{i}^l \phi^l(1)} \\
    &= \v{\phi}^T(1)\v{R}_{i} \\
    \hat{S}_{i+1/2} &= \sum{l = 1}{M}{S_{i+1}^l \phi^l(-1)} \\
    &= \v{\phi}^T(-1)\v{S}_{i+1} \\
    \hat{U}_{i+1/2} &= \sum{l = 1}{M}{U_{i}^l \phi^l(1)} \\
    &= \v{\phi}^T(1)\v{U}_{i}
  \end{align*}

  \begin{align*}
    R_i^k &= -\frac{1}{\Delta x} \dintt{-1}{1}{\sum{l = 1}{M}{Q_i^l \phi^l(\xi)} \phi_{\xi}^k(\xi)}{\xi} + \frac{1}{\Delta x}\p{\phi^k(1)\hat{Q}_{i+1/2} - \phi^k(-1)\hat{Q}_{i-1/2}} \\
    R_i^k &= -\frac{1}{\Delta x} \sum{l=1}{M}{Q_i^l\dintt{-1}{1}{\phi^l(\xi) \phi_{\xi}^k(\xi)}{\xi}} + \frac{1}{\Delta x}\p{\phi^k(1)\hat{Q}_{i+1/2} - \phi^k(-1)\hat{Q}_{i-1/2}} \\
    R_i^k &= -\frac{1}{\Delta x} \p{A \v{Q}_i}_k + \frac{1}{\Delta x}\p{\phi^k(1)\v{\phi}^T(-1)\v{Q}_{i+1} - \phi^k(-1)\v{\phi}^T(-1)\v{Q}_{i}} \\
    \v{R}_i &= -\frac{1}{\Delta x} A \v{Q}_i + \frac{1}{\Delta x}\p{\v{\phi}(1)\v{\phi}^T(-1)\v{Q}_{i+1} - \v{\phi}(-1)\v{\phi}^T(-1)\v{Q}_{i}} \\
    \v{R}_i &= -\frac{1}{\Delta x} A \v{Q}_i + \frac{1}{\Delta x}\p{\Phi(1, -1)\v{Q}_{i+1} - \Phi(-1, -1)\v{Q}_{i}} \\
    \v{R}_i &= -\frac{1}{\Delta x} \p{A + \Phi(-1,-1)} \v{Q}_i + \frac{1}{\Delta x}\Phi(1, -1)\v{Q}_{i+1}
  \end{align*}

  \begin{align*}
    S_i^k &= -\frac{1}{\Delta x} \dintt{-1}{1}{\sum{l = 1}{M}{R_i^l \phi^l(\xi)} \phi_{\xi}^k(\xi)}{\xi} + \frac{1}{\Delta x}\p{\phi^k(1)\hat{R}_{i+1/2} - \phi^k(-1)\hat{R}_{i-1/2}}\\
    S_i^k &= -\frac{1}{\Delta x} \p{A \v{R}_i}_k + \frac{1}{\Delta x}\p{\phi^k(1)\v{\phi}^T(1)\v{R}_{i} - \phi^k(-1)\v{\phi}^T(1)\v{R}_{i-1}} \\
    \v{S}_i &= -\frac{1}{\Delta x} \p{A - \Phi(1, 1)} \v{R}_i - \frac{1}{\Delta x}\Phi(-1, 1)\v{R}_{i-1}
  \end{align*}

  Note that I am treating the $q^3$ fluxes as constants, they don't depend on
  the unknowns $Q_i^l$
  \begin{align*}
    U_i^k &= \frac{1}{\Delta x} \dintt{-1}{1}{\sum{l = 1}{M}{S_i^l \phi_{\xi}^l(\xi)} q_i^3\phi^k(\xi)}{\xi} \\
    &- \frac{1}{\Delta x}\p{\phi^k(1)\p{q^-_{i+1/2}}^3S^-_{i+1/2} - \phi^k(-1)\p{q^+_{i-1/2}}^3 S^+_{i-1/2}}\\
    &+ \frac{1}{\Delta x}\p{\phi^k(1)\hat{q}^3_{i+1/2}\hat{S}_{i+1/2} - \phi^k(-1)\hat{q}^3_{i-1/2}\hat{S}_{i-1/2}} \\
    U_i^k &= \frac{1}{\Delta x} \sum{l = 1}{M}{S_i^l\dintt{-1}{1}{q_i^3 \phi^k(\xi) \phi_{\xi}^l(\xi)}{\xi}} \\
    &- \frac{1}{\Delta x}\p{\phi^k(1)\p{q^-_{i+1/2}}^3\v{\phi}^T(1)\v{S}_{i} - \phi^k(-1)\p{q^+_{i-1/2}}^3 \v{\phi}^T(-1)\v{S}_{i}}\\
    &+ \frac{1}{\Delta x}\p{\phi^k(1)\hat{q}^3_{i+1/2}\v{\phi}^T(-1)\v{S}_{i+1} - \phi^k(-1)\hat{q}^3_{i-1/2}\v{\phi}^T(-1)\v{S}_{i}} \\
    U_i^k &= \frac{1}{\Delta x} \p{B_i\v{S}_i}_k - \frac{1}{\Delta x}\p{\phi^k(1)\p{q^-_{i+1/2}}^3\v{\phi}^T(1)\v{S}_{i} - \phi^k(-1)\p{q^+_{i-1/2}}^3 \v{\phi}^T(-1)\v{S}_{i}}\\
    &+ \frac{1}{\Delta x}\p{\phi^k(1)\hat{q}^3_{i+1/2}\v{\phi}^T(-1)\v{S}_{i+1} - \phi^k(-1)\hat{q}^3_{i-1/2}\v{\phi}^T(-1)\v{S}_{i}} \\
    \v{U}_i &= \frac{1}{\Delta x} B_i\v{S}_i - \frac{1}{\Delta x}\p{\v{\phi}(1)\p{q^-_{i+1/2}}^3\v{\phi}^T(1)\v{S}_{i} - \v{\phi}(-1)\p{q^+_{i-1/2}}^3 \v{\phi}^T(-1)\v{S}_{i}}\\
    &+ \frac{1}{\Delta x}\p{\v{\phi}(1)\hat{q}^3_{i+1/2}\v{\phi}^T(-1)\v{S}_{i+1} - \v{\phi}(-1)\hat{q}^3_{i-1/2}\v{\phi}^T(-1)\v{S}_{i}} \\
    \v{U}_i &= \frac{1}{\Delta x} B_i\v{S}_i - \frac{1}{\Delta x}\p{\p{q^-_{i+1/2}}^3\Phi(1, 1)\v{S}_{i} - \p{q^+_{i-1/2}}^3 \Phi(-1, -1)\v{S}_{i}}\\
    &+ \frac{1}{\Delta x}\p{\hat{q}^3_{i+1/2}\Phi(1, -1)\v{S}_{i+1} - \hat{q}^3_{i-1/2}\Phi(-1, -1)\v{S}_{i}} \\
    \v{U}_i &= \frac{1}{\Delta x} \p{B_i - \p{q^-_{i+1/2}}^3\Phi(1, 1) + \p{\p{q^+_{i-1/2}}^3 - \hat{q}^3_{i-1/2}} \Phi(-1, -1)} \v{S}_i \\
    &+ \frac{1}{\Delta x}\p{\hat{q}^3_{i+1/2}\Phi(1, -1)\v{S}_{i+1}} \\
  \end{align*}

  \begin{align*}
    \dot{Q}_i^k &= \frac{1}{\Delta x} \dintt{-1}{1}{\sum{l = 1}{M}{U_i^l \phi_{\xi}^l(\xi)}\phi^k(\xi)}{\xi} - \frac{1}{\Delta x}\p{\phi^k(1)\hat{U}_{i+1/2} - \phi^k(-1)\hat{U}_{i-1/2}} \\
    \dot{Q}_i^k &= \frac{1}{\Delta x} \p{A\v{U}_i}_k - \frac{1}{\Delta x}\p{\phi^k(1)\v{\phi}^T(1)\v{U}_{i} - \phi^k(-1)\v{\phi}^T(1)\v{U}_{i-1}} \\
    \dot{\v{Q}}_i^k &= \frac{1}{\Delta x} \p{A - \Phi(1, 1)}\v{U}_i + \frac{1}{\Delta x}\Phi(-1, 1)\v{U}_{i-1} \\
  \end{align*}

\end{document}
