\documentclass[oneside]{article}
\usepackage[letterpaper, margin=2cm]{geometry}
\usepackage{Notes}

\begin{document}
  \begin{center}
    \textbf{\Large{Hyperbolicity of Shallow Water Equations}} \\
  \end{center}

  The one dimensional inviscid shallow water equations in primitive variables are given as
  \begin{equation}
    \begin{bmatrix}
      h \\
      hu
    \end{bmatrix}_t +
    \begin{bmatrix}
      hu \\
      hu^2 + \frac{1}{2}gh^2
    \end{bmatrix}_x
    =
    \begin{bmatrix}
      0 \\
      0
    \end{bmatrix},
  \end{equation}
  where \(h\) is the height of the water, \(u\) is the horizontal velocity, and \(g\)
  is the gravitational constant.
  Often it may be better to use the conserved variables of mass and momentum instead of
  height and velocity.
  Let \(q_1 = h\) be the mass and \(q_2 = hu\) be the momentum of a cross section of the
  water.
  In the conserved variables the shallow water equations are
  \begin{equation}
    \begin{bmatrix}
      q_1 \\
      q_2
    \end{bmatrix}_t +
    \begin{bmatrix}
      q_2 \\
      \frac{q_2^2}{q_1} + \frac{1}{2}g q_1^2
    \end{bmatrix}_x
    =
    \begin{bmatrix}
      0 \\
      0
    \end{bmatrix},
  \end{equation}
  or in vector form
  \begin{equation}
    \v{q}_t + \v{f}\p{\v{q}}_x = \v{0},
  \end{equation}
  where
  \begin{equation}
    \v{q} =
    \begin{bmatrix}
      q_1 \\
      q_2
    \end{bmatrix} \quad \text{ and } \quad
    \v{f}\p{\v{q}} =
    \begin{bmatrix}
      q_2 \\
      \frac{q_2^2}{q_1} + \frac{1}{2}g q_1^2
    \end{bmatrix}.
  \end{equation}

  For smooth solutions this can be expressed in its quasilinear form,
  \begin{equation}
    \v{q}_t + \v{f}'\p{\v{q}} \v{q}_x = \v{0},
  \end{equation}
  where \(\v{f}'\p{\v{q}}\) is the Jacobian matrix.
  We say that a system of conservation laws is hyperbolic if the Jacobian matrix is
  diagonalizable with real eigenvalues.
  To this end, consider the Jacobian matrix of the shallow water equations,
  \begin{equation}
    \v{f}'\p{\v{q}} =
    \begin{bmatrix}
      0 & 1 \\
      -\frac{q_2^2}{q_1^2} + gq_1 & 2\frac{q_2}{q_1}
    \end{bmatrix}
  \end{equation}
  or in primitive variables
  \begin{equation}
    \v{f}'\p{\v{q}} =
    \begin{bmatrix}
      0 & 1 \\
      -u^2 + gh & 2u
    \end{bmatrix}.
  \end{equation}
  Note that the Jacobian must be computed with respect to the conserved variables, but
  can then be expressed in terms of the primitive variables.
  In order to determine if these equations are hyperbolic, we must find the eigenvalues
  of this Jacobian matrix.
  This is easiest to do when expressed in terms of the primitive variables.
  The eigenvalues are computed as follows
  \begin{align*}
    \begin{vmatrix}
      -\lambda & 1 \\
      -u^2 + gh & 2u - \lambda
    \end{vmatrix} &= 0 \\
    -\lambda\p{2u - \lambda} - \p{-u^2 + gh} &= 0 \\
    \lambda^2 - 2u\lambda + u^2 - gh &= 0 \\
    \lambda &= \frac{2u \pm \sqrt{4u^2 - 4\p{u^2 - gh}}}{2} \\
    \lambda &= \frac{2u \pm \sqrt{4u^2 - 4u^2 + 4gh}}{2} \\
    \lambda &= \frac{2u \pm \sqrt{4gh}}{2} \\
    \lambda &= u \pm \sqrt{gh}.
  \end{align*}
  These eigenvalues are real if and only if \(h \ge 0\).
  As a negative height of the water does not make physical sense this is an intuitive
  result.
  In conserved variables the eigenvalues are
  \begin{equation}
    \lambda = \frac{q_2}{q_1} \pm \sqrt{g q_1}.
  \end{equation}

  Next we will find the eigenvectors for these eigenvalues.
  For the eigenvalue \(u + \sqrt{gh}\), we can compute the eigenvector as follows
  \begin{align*}
    \begin{bmatrix}
      -u - \sqrt{gh} & 1 \\
      -u^2 + gh & 2u - u - \sqrt{gh}
    \end{bmatrix}
    \begin{bmatrix}
      v_1 \\
      v_2
    \end{bmatrix} &=
    \begin{bmatrix}
      0 \\
      0
    \end{bmatrix} \\
    \p{-u - \sqrt{gh}}v_1 + v_2 &= 0 \\
    \p{-u^2 + gh}v_1 + \p{u - \sqrt{gh}}v_2 &= 0
    \intertext{Multiplying the first equation by \(u - \sqrt{gh}\) gives}
    \p{-u^2 + \sqrt{gh}}v_1 + \p{u - \sqrt{gh}}v_2 &= 0 \\
    v_1 &= 1 \\
    v_2 &= u + \sqrt{gh} \\
    \v{v} &=
    \begin{bmatrix}
      1 \\
      u + \sqrt{gh}
    \end{bmatrix}
  \end{align*}
  In conservative form this eigenvector is
  \begin{equation}
    \v{v} =
    \begin{bmatrix}
      1 \\
      \frac{q_2}{q_1} + \sqrt{g q_1}
    \end{bmatrix}.
  \end{equation}

  For the eigenvalue \(u - \sqrt{gh}\) the eigenvector is computed as follows,
  \begin{align*}
    \begin{bmatrix}
      -u + \sqrt{gh} & 1 \\
      -u^2 + gh & 2u - u + \sqrt{gh}
    \end{bmatrix}
    \begin{bmatrix}
      v_1 \\
      v_2
    \end{bmatrix}
    &=
    \begin{bmatrix}
      0 \\
      0
    \end{bmatrix} \\
    \p{-u + \sqrt{gh}}v_1 + v_2 &= 0 \\
    \p{-u^2 + gh}v_1 + \p{u + \sqrt{gh}}v_2 &= 0
    \intertext{Multiplying the first equation by \(\p{u + \sqrt{gh}}\) gives}
    \p{-u^2 + gh}v_1 + \p{u + \sqrt{gh}}v_2 &= 0 \\
    v_1 &= 1 \\
    v_2 &= u - \sqrt{gh} \\
    \v{v} &=
    \begin{bmatrix}
      1 \\
      u - \sqrt{gh}
    \end{bmatrix}.
  \end{align*}
  In conservative form this eigenvector is
  \begin{equation}
    \v{v} =
    \begin{bmatrix}
      1 \\
      \frac{q_2}{q_1} - \sqrt{g q_1}
    \end{bmatrix}.
  \end{equation}


\end{document}