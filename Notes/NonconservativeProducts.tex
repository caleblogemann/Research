\documentclass{article}
\usepackage[letterpaper, margin=2cm]{geometry}
\usepackage{Notes}

\begin{document}
  \begin{center}
    \textbf{\Large{Nonconservative Products}} \\
  \end{center}

  \section{Definition}
    Consider the nonconservative product
    \[
      g(\v{q}) \d{\v{q}}{x},
    \]
    where \(g(\v{q}): \RR^p \to \RR^p \times \RR^p\) is continuous, but \(\v{q}\) is
    possibly discontinuous.
    In this case, the product is traditionally not well-defined at the discontinuities
    of \(\v{q}\).
    In order to define this product for discontinuous functions, \(\v{q}\), it is
    possible to regularize \(\v{q}\) with a path \(\phi \) at discontinuities according
    to the theory laid out by Dal Maso, Le Floch, and Murat.
    To this end consider Lipschitz continuous paths,
    \(\v{\psi}:\br{0, 1} \times \RR^p \times \RR^p \to \RR^p \), that satisfy the
    following properties.
    \begin{enumerate}
      \item \(\forall \v{q}_L, \v{q}_R \in \RR^p\),
        \(\v{\psi}(0, \v{q}_L, \v{q}_R) = \v{q}_L\) and
        \(\v{\psi}(1, \v{q}_L, \v{q}_R) = \v{q}_R\)
      \item \(\exists k > 0\), \(\forall \v{q}_L, \v{q}_R \in \RR^p\),
        \(\forall s \in \br{0, 1}\), \(\abs{\pd{\v{\psi}}{s}(s, \v{q}_L, \v{q}_R)}
        \le k \abs{\v{q}_L - \v{q}_R}\) elementwise
      \item \(\exists k > 0\), \(\forall \v{q}_L, \v{q}_R, \v{u}_L, \v{u}_R \in \RR^p\),
        \(\forall s \in \br{0, 1}\), elementwise
        \[
          \abs{\pd{\v{\psi}}{s}(s, \v{q}_L, \v{q}_R) - \pd{\v{\psi}}{s}(s, \v{u}_L, \v{u}_R)}
          \le k \p{\abs{\v{q}_L - \v{u}_L} + \abs{\v{q}_R - \v{u}_R}}
        \]
    \end{enumerate}
    Once we have these paths, \(\v{\psi} \), we can define the nonconservative product.

    Let \(\v{q}:\br{a, b} \to \RR^p\) be a function of bounded variation, let
    \(g: \RR^p \to \RR^p \times \RR^p \) be a continuous function, and let \(\v{\psi} \)
    satisfy the properties given above.
    Then there exists a unique real-valued bounded Borel measure \(\mu \) on
    \(\br{a, b}\) characterized by the two following properties.
    \begin{enumerate}
      \item If \(q\) is continuous on a Borel set \(B \subset \br{a, b}\), then
        \[
          \mu(B) = \dintt{B}{}{g(\v{q}) \d{\v{q}}{x}}{x}
        \]
      \item If \(q\) is discontinuous at a point \(x_0 \in \br{a, b}\), then
        \[
          \mu({x_0}) = \dintt{0}{1}{g(\psi(s; q(x_0^-), q(x_0^+))) \pd{\psi}{s}(s; q(x_0^-), q(x_0^+))}{s}
        \]
    \end{enumerate}
    By definition, this measure \(\mu \) is the nonconservative product \(g(q) \d{q}{x}\)
    and will be denoted by
    \[
      \mu = \br[\psi]{g(q) \d{q}{x}}
    \]

    \noindent In higher dimensions the paths, \(\v{\psi}\) must also have the property that
    \begin{enumerate}
      \item[4.] \(\v{\psi}(s, \v{q}_L, \v{q}_R) = \v{\psi}(1 - s, \v{q}_L, \v{q}_R)\)
    \end{enumerate}
    Then the following Thereom can be given in spacetime
    Let \(\v{q}: \Omega \to \RR^m\) be a bounded function of bounded variation defined
    on an open subset \(\Omega \) of \(\RR^{n+1}\) and \(\v{t}: \RR^m \to \RR^m\) be
    a locally bounded Borel function.
    Then there existsa unique family of real-valued bounded Borel measures \(\mu_i\)
    on \(\Omega \), \(i = 1, 2, \ldots, m\) such that
    \begin{enumerate}
      \item if \(B\) is a continuous Borel subset of \(\Omega \), then
        \[
          \mu_i(B) = \dintt{B}{}{t_{ik}(q) \v{q}_{x_k}}{\lambda}
        \]
        where \(\lambda \) is the Borel measure;

      \item if \(B\) is a discontinuous subset of \(\Omega \) of approximate jump, then
        \[
          \mu_i(B) = \dintt{B}{}{\dintt{0}{1}{t_{ik}(\v{\psi}(s, \v{q}^L, \v{q}^R))
            \pd{\v{\psi}}{s}(s, \v{q}^L, \v{q}^R)}{s} \v{n}_k^L}{H^n}
        \]
        with \(\v{q}^L\) and \(\v{q}^R\) the left and right traces at the discontinuity,
        where \(H^n\) is the n-dimensional Hausdorf measure and where we choose
        \(\v{n}^L\) the outward normal with respect to the left state,

      \item if \(B\) is an irregular Borel subset of \(\Omega \), then \(\mu_i(B) = 0\)
    \end{enumerate}
    \noindent This is given in Rhebergen without proof, but I haven't found any outside
    original references for this Theorem.
    Mostly this reflects the one dimensional theorem, but I don't understand the
    appearance of the outward facing normal.
    It appears to have been an arbitrary choice between \(\v{n}^L\) and \(\v{n}^R\).
    Also I am not sure why it is necessary at all.

  \section{Weak Solutions}
    A function \(\v{q}\) of bounded variation is a weak solution to
    \begin{gather}
      \v{q}_t + g(\v{q}) \v{q}_x = 0
    \end{gather}
    if
    \begin{gather}
      \v{q}_t + \br[\phi]{g(\v{q}) \v{q}_x} = 0
    \end{gather}
    as a bounded Borel measure on \(\RR \times \RR_+\).
    This is equivalent to finding \(\v{q}\) that satisfies,
    \begin{gather}
      \dintt{\RR_+}{}{\dintt{\RR}{}{v_t(t, x) \v{q}(t, x)}{x}}{t}
      + \dintt{\RR_+}{}{\dint{\RR}{}{v(t, \cdot) \br[\psi]{g(\v{q}(t, \cdot)) \v{q}_x(t, \cdot)}}}{t}
      = \v{0}
    \end{gather}
    for all functions \(v \in C^{\infty}_0\p{\RR_t \times \RR}\).

  \section{DG Weak Formulation}

  \subsection{Rhebergen Weak Formulation}
    Find \(\v{q} \in V_h\) such that for all \(\v{v} \in V_h\),
    \begin{gather}
      \sum{j}{}{\dintt{K_j}{}{\v{v}^T \v{q}_t - \v{v}^T_x \v{f}(\v{q})
        + \v{v}^T g(\v{q}) \v{q}_x}{x}}
      + \sum{S}{}{\dintt{S}{}{\p{\v{v}^L - \v{v}^R}^T \hat{\v{P}}^{nc}}{S}} \\
      + \sum{S}{}{\dintt{S}{}{\frac{1}{2}\p{\v{v}^R + \v{v}^L}^T
        \dintt{0}{1}{g\p{\v{\psi}\p{\tau, \v{q}^L, \v{q}^R}}
        \pd{\v{\psi}}{\tau}(\tau, \v{q}^L, \v{q}^R)}{\tau}}{S}}
    \end{gather}
    where \(\hat{\v{P}}^{nc}\) is the nonconservative numerical flux, if symmetrical
    wave speeds are assumed, then the Rusanov or Local Lax Friedrichs flux can be used,
    otherwise the nonconservative product will affect the numerical flux.

  \subsection{Pure Nonconservative DG Formulation}
    Consider the 1D nonconservative equation shown below,
    \[
      \v{q}_t + g(\v{q}) \v{q}_x = \v{0} \qquad x \in \br{a, b}, 0 < t < T
    \]
    Let \(\set{K_j}\) be a mesh of the domain \(\br{a, b}\).
    Also denote the DG space as
    \[
      V_h = \set{v \in L^1\p{\br{a, b}} \big| \eval{v}{K_j} \in \PP^M(K_j)}
    \]
    Now the semi discrete DG formulation for this problem becomes finding
    \(\v{q}_h \in V_h\) for all \(v_h \in V_h\) that satisfies
    \begin{gather}
      \dintt{a}{b}{v_h \v{q}_{h,t}}{x} + \dintt{a}{b}{v_h \br[\psi]{g(\v{q}_h)\v{q}_{h,x}}}{x} = \v{0} \\
      \sum{j}{}{\dintt{K_j}{}{v_h \v{q}_{h,t}}{x}}
      + \sum{j}{}{\dintt{K_j}{}{v_h g(\v{q}_h) \v{q}_{h,x}}{x}}
      + \sum{I}{}{\hat{v}_h \dintt{0}{1}{g(\psi(s, \v{q}_h^L, \v{q}_h^R)) \pd{\psi}{s}(s, \v{q}_h^L, \v{q}_h^R)}{s}}
      = 0
    \end{gather}

    Consider the case where there exists a function \(\v{f}\p{\v{q}_h}\) such that
    \(\v{f}'\p{\v{q}_h} = g(\v{q})\).
    \begin{gather}
      \sum{j}{}{\dintt{K_j}{}{v_h \v{q}_{h,t}}{x}}
      + \sum{j}{}{\dintt{K_j}{}{v_h g(\v{q}_h) \v{q}_{h,x}}{x}}
      + \sum{I}{}{\hat{v}_h \dintt{0}{1}{g(\psi(s, \v{q}_h^L, \v{q}_h^R)) \pd{\psi}{s}(s, \v{q}_h^L, \v{q}_h^R)}{s}}
      = 0
    \end{gather}

\end{document}