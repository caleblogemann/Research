\documentclass[oneside]{article}
\usepackage[letterpaper, margin=2cm]{geometry}
\usepackage{Notes}

\newcommand{\dOmega}{\partial\Omega}
\newcommand{\eh}{\varepsilon_h}
\newcommand{\eho}{\varepsilon_h^0}
\newcommand{\ehd}{\varepsilon_h^{\partial}}

\begin{document}
\begin{center}
\textbf{\Large{Hybridized Discontinuous Galerkin Method \\}}
\end{center}

These notes are intended to give background on Hyridized Discontinuous Galerkin
methods and then explore how to apply this method to Thin Film Equations

\section{Introduction/Main Idea}
  To start we will consider Poisson's equation with Dirichlet boundary conditions
  \begin{align*}
    -\Delta u &= f \qquad \text{in } \Omega \\
    u &= g \qquad \text{on } \dOmega 
  \end{align*}
  We will consider this in mixed form by introducing the auxilliary variable
  $\v{q} = -\nabla u$,  then the equation becomes
  \begin{align*}
    \v{q} + \nabla u &= 0 \qquad \text{in } \Omega \\
    \Div \v{q} &= f \qquad \text{in } \Omega \\
    u &= g \qquad \text{on } \dOmega
  \end{align*}
  This problem has an exact solution that can be found analytically assuming
  some nice properties.

  I will introduce a triangulation of $\Omega$, $\mcT_h$, and reformulate the
  problem on this triangulation that will give the same exact solution.

  \subsection{Notation}
    First some notation,
    \begin{align*}
      \partial\mcT_h &= \set{\partial K: K \in \mcT_h} \\
      F &= \partial K \cap \dOmega \text{ for } K \in \mcT_h \\
      F &= \partial K^+ \cap \partial K^- \text{ for } K^+, K^- \in \mcT_h \\
    \end{align*}
    Let $\eh$ be the set of all faces, $F$, and $\eho$ be interior faces, and
    $\ehd$ be boundary faces.

    Let $\v{n}^+$ and $\v{n}^-$ be the outward unit normals of $\partial K^+$ and
    $\partial K^-$ respectively, and $\p{\v{q}^{\pm}, u^{\pm}}$ be the interior
    values of $\p{\v{q}, u}$ on $F$ for $K^{\pm}$.
    Define
    \begin{align*}
      \bbr{\v{q} \cdot n} = \v{q}^+ \cdot \v{n}^+ + \v{q}^- \cdot \v{n}^- \\
      \bbr{u\v{n}} = u^+\v{n}^+ + u^-\v{n}^- \\
      \set{\v{q}} = \frac{\v{q}^+ + \v{q}^-}{2} \\
      \set{u} = \frac{u^+ + u^-}{2}
    \end{align*}

  \subsection{Reformulation}
    Now we can reformulate the original Poisson's problem on $\mcT_h$ as
    a local problem for each $K$
    \begin{align*}
      \v{q} + \nabla u &= 0 \\
      \Div \v{q} &= f
    \end{align*}
    a transmission condition on each interior face, $F \in \eho$
    \begin{align*}
      \bbr{u\v{n}} &= \v{0} \\
      \bbr{\v{q} \cdot \v{n}} &= 0
    \end{align*}
    and the boundary condition on each boundary face, $F \in \ehd$
    \begin{align*}
      u = g
    \end{align*}
    This problem is equivalent to the original problem on $\Omega$.
    The $(\v{q}, u)$ that satisfies this problem also solve the original problem.

    We would like to be able to solve the local problem locally, but this requires
    boundary conditions on each element $K$ for the local problem to be solved.
    Therefore consider the local problem
    \begin{align*}
      \v{q} + \nabla u &= 0 \quad \text{in } K \\
      \Div \v{q} &= f \quad \text{in } K \\
      u &= \hat{u} \quad \text{on } \partial K
    \end{align*}
    We have introduced another unknown $\hat{u}$ on each interior face
    $F \in \eho$.
    This unknown automatically makes us satisfy $\bbr{u\v{n}} = \v{0}$, so the
    transmission condition becomes 
    \begin{align*}
      \bbr{\v{q}\cdot\v{n}} = 0 \quad \text{ on } F \in \eho
    \end{align*}
    and we still have the boundary condition
    \begin{align*}
      u = g \quad \text{ on } F \in \ehd
    \end{align*}

    Now solving for $\p{\v{q}, u, \hat{u}}$ will give the same solution as the
    original problem, however $\v{q}$ and $u$ can be solved locally and only
    $\hat{u}$ needs to be solved globally.

  \subsection{General Algorithm}
    Here is the outline for solving for $\v{q}$, $u$, and $\hat{u}$.
    First split the local problem in two, so that one part depends on $f$ and the
    other part depends on $\hat{u}$, that is let
    $\v{q} = \v{Q}_f + \v{Q}_{\hat{u}}$ and $u = U_f + U_{\hat{u}}$, where
    \begin{align*}
      \v{Q}_f + \nabla U_f &= 0 \quad \text{in } K \\
      \Div \v{Q}_f &= f \quad \text{in } K \\
      U_f &= 0 \quad \text{on } \partial K
    \end{align*}
    and
    \begin{align*}
      \v{Q}_{\hat{u}}+ \nabla U_{\hat{u}} &= 0 \quad \text{in } K \\
      \Div \v{Q}_{\hat{u}} &= 0 \quad \text{in } K \\
      U_{\hat{u}} &= \hat{u} \quad \text{on } \partial K
    \end{align*}
    Now the transmission condition becomes
    \begin{align*}
      \bbr{\v{Q}_{\hat{u}}} = -\bbr{\v{Q}_f}
    \end{align*}

    First solve for $\v{Q}_f$ exactly, then solve for $\v{Q}_{\hat{u}}$ in terms
    of $\hat{u}$.
    Now the transmission condition gives a global linear algebra problem
    \begin{align*}
      \bbr{\v{Q}_{\hat{u}}} = -\bbr{\v{Q}_f}
    \end{align*}
    since $\bbr{\v{Q}_{\hat{u}}}$ is a linear system in $\hat{u}$ and
    $-\bbr{\v{Q}_f}$ is known.

    After this linear algebra problem is solved, the values of $U_{\hat{u}}$ and
    $\v{Q}_{\hat{u}}$ can be found/reconstructed locally.
    The full solution is then $\v{q} = \v{Q}_f + \v{Q}_{\hat{u}}$ and
    $u = U_f + U_{\hat{u}}$.

  \subsection{1D example}
    In 1D, the problem becomes
    \begin{align*}
      Q_f + U_f' &= 0 \quad \text{in } K \\
      Q_f' &= f \quad \text{in } K \\
      U_f &= 0 \quad \text{on } \partial K
    \end{align*}
    and
    \begin{align*}
      Q_{\hat{u}} + U_{\hat{u}}' &= 0 \quad \text{in } K \\
      Q_{\hat{u}}' &= 0 \quad \text{in } K \\
      U_{\hat{u}} &= \hat{u} \quad \text{on } \partial K
    \end{align*}
    and
    \begin{align*}
      Q_{\hat{u}}(x^-_{j+1/2}) - Q_{\hat{u}}(x^+_{j+1/2})  = -Q_f(x^-_{j+1/2}) + Q_f(x^+_{j+1/2}
    \end{align*}
    where the uniform mesh is given by $x_i$ at cell centers, $x_{i+1/2}$ at
    cell interfaces, and spacing $h$.

    Solving the $\hat{u}$ system we see that
    \begin{align*}
      Q_{\hat{u}}' &= 0 \\
      Q_{\hat{u}} &= c
    \end{align*}
    and
    \begin{align*}
      Q_{\hat{u}} + U_{\hat{u}}' &= 0 \\
      U_{\hat{u}}' &= -c \\
      U_{\hat{u}} &= -cx + b \\
    \end{align*}
    with the boundary conditions, we know $U_{\hat{u}}$ is a line from
    $\hat{u}_{j-1/2}$ to $\hat{u}_{j+1/2}$, and $Q_{\hat{u}}$ is the opposite of
    the slope of this line.
    \begin{align*}
      U_{\hat{u}} &= \frac{\hat{u}_{j+1/2} - \hat{u}_{j-1/2}}{h}\p{x - x_{j-1/2}} - \hat{u}_{j-1/2} \\
      Q_{\hat{u}} &= -\frac{\hat{u}_{j+1/2} - \hat{u}_{j-1/2}}{h}
    \end{align*}

    Now we can form the linear system given by the transmission condition
    \begin{align*}
      Q_{\hat{u}}(x^-_{j+1/2}) - Q_{\hat{u}}(x^+_{j+1/2}) &= -Q_f(x^-_{j+1/2}) + Q_f(x^+_{j+1/2} \\
      -\frac{\hat{u}_{j+1/2} - \hat{u}_{j-1/2}}{h} + \frac{\hat{u}_{j+3/2} - \hat{u}_{j+1/2}}{h} &= -Q_f(x^-_{j+1/2}) + Q_f(x^+_{j+1/2} \\
      \frac{\hat{u}_{j-1/2} - 2\hat{u}_{j+1/2} + \hat{u}_{j+3/2}}{h} &= -Q_f(x^-_{j+1/2}) + Q_f(x^+_{j+1/2}
    \end{align*}
    After solving this linear system we already have expressions for $Q_{\hat{u}}$
    and $U_{\hat{u}}$ in terms of $\hat{u}$.

\section{Hybridizable Discontinuous Galerkin Method}
  \subsection{General Form}
  \subsection{1D example}
\end{document}
