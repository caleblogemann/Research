\documentclass[oneside]{article}
\usepackage[letterpaper, margin=2cm]{geometry}
\usepackage{Notes}

\begin{document}
  \begin{center}
    \textbf{\Large{Hyperbolicity of Generalized Shallow Water Equations}} \\
  \end{center}

  The one dimensional first order inviscid generalized shallow water equations in
  primitive variables are given as
  \begin{equation}
    \begin{bmatrix}
      h \\
      hu \\
      hs
    \end{bmatrix}_t +
    \begin{bmatrix}
      hu \\
      hu^2 + \frac{1}{2}gh^2 + \frac{1}{3}hs^2 \\
      2hus
    \end{bmatrix}_x
    =
    \begin{bmatrix}
      0 & 0 & 0 \\
      0 & 0 & 0 \\
      0 & 0 & u
    \end{bmatrix}
    \begin{bmatrix}
      h \\
      hu \\
      hs
    \end{bmatrix}_x,
  \end{equation}
  where \(h\) is the height of the water, \(u\) is the mean horizontal velocity, \(s\)
  is linear scaling of the horizontal velocity, and \(g\) is the gravitational constant.
  Often it may be better to use the conserved variables of mass and momentum instead of
  height and velocity.
  Let \(q_1 = h\) be the mass and \(q_2 = hu\) be the momentum of a cross section of the
  water, and \(q_3 = hs\) be the final conserved variable.
  In the conserved variables the shallow water equations are
  \begin{equation}
    \begin{bmatrix}
      q_1 \\
      q_2 \\
      q_3
    \end{bmatrix}_t +
    \begin{bmatrix}
      q_2 \\
      \frac{q_2^2}{q_1} + \frac{1}{2}g q_1^2 + \frac{1}{3} \frac{q_3^2}{q_1} \\
      2 \frac{q_2 q_3}{q_1}
    \end{bmatrix}_x
    =
    \begin{bmatrix}
      0 & 0 & 0 \\
      0 & 0 & 0 \\
      0 & 0 & \frac{q_2}{q_1}
    \end{bmatrix}
    \begin{bmatrix}
      q_1 \\
      q_2 \\
      q_3
    \end{bmatrix}_x,
  \end{equation}
  or in vector form
  \begin{equation}
    \v{q}_t + \v{f}\p{\v{q}}_x = Q\v{q}_x,
  \end{equation}
  where
  \begin{equation}
    \v{q} =
    \begin{bmatrix}
      q_1 \\
      q_2 \\
      q_3
    \end{bmatrix}, \quad
    \v{f}\p{\v{q}} =
    \begin{bmatrix}
      q_2 \\
      \frac{q_2^2}{q_1} + \frac{1}{2}g q_1^2 + \frac{1}{3} \frac{q_3^2}{q_1} \\
      2 \frac{q_2 q_3}{q_1}
    \end{bmatrix}, \quad
    Q =
    \begin{bmatrix}
      0 & 0 & 0 \\
      0 & 0 & 0 \\
      0 & 0 & \frac{q_2}{q_1}
    \end{bmatrix}
  \end{equation}

  For smooth solutions this can be expressed in its quasilinear form,
  \begin{equation}
    \v{q}_t + \p{\v{f}'\p{\v{q}} - Q} \v{q}_x = \v{0},
  \end{equation}
  where \(\v{f}'\p{\v{q}}\) is the Jacobian matrix.
  A system of balance laws is said to be hyperbolic if the matrix
  \(A = \v{f}'\p{\v{q}} - Q\) is diagonalizable with real eigenvalues.
  To show that this system is hyperbolic we proceed to find the eigenvalues of the
  matrix \(A\), where the flux Jacobian is
  \begin{equation}
    \v{f}'\p{\v{q}} =
    \begin{bmatrix}
       0 & 1 & 0 \\
       -\frac{q_2^2}{q_1^2} + gq_1 - \frac{1}{3} \frac{q_3^2}{q_1^2} & 2 \frac{q_2}{q_1} & \frac{2}{3} \frac{q_3}{q_1} \\
       -2 \frac{q_2 q_3}{q_1} & 2 \frac{q_3}{q_1} & 2\frac{q_2}{q_1}
    \end{bmatrix}
  \end{equation}
  or in primitive variables
  \begin{equation}
    \v{f}'\p{\v{q}} =
    \begin{bmatrix}
      0 & 1 & 0 \\
      -u^2 + gh - \frac{1}{3}s^2 & 2u & \frac{2}{3}s \\
      -2us & 2s & 2u
    \end{bmatrix}
  \end{equation}
  So the matrix \(A\) is thus
  \begin{equation}
    A =
    \begin{bmatrix}
       0 & 1 & 0 \\
       -\frac{q_2^2}{q_1^2} + gq_1 - \frac{1}{3} \frac{q_3^2}{q_1^2} & 2 \frac{q_2}{q_1} & \frac{2}{3} \frac{q_3}{q_1} \\
       -2 \frac{q_2 q_3}{q_1} & 2 \frac{q_3}{q_1} & \frac{q_2}{q_1}
    \end{bmatrix} =
    \begin{bmatrix}
      0 & 1 & 0 \\
      -u^2 + gh - \frac{1}{3}s^2 & 2u & \frac{2}{3}s \\
      -2us & 2s & u
    \end{bmatrix}.
  \end{equation}
  The eigenvalues, \(\lambda \) of \(A\) satisfy \(\det{A - \lambda I} = 0\).
  This equation can be solved as follows,
  \begin{align*}
    \det{A - \lambda I} &= 0 \\
    \begin{vmatrix}
      -\lambda & 1 & 0 \\
      -u^2 + gh - \frac{1}{3}s^2 & 2u - \lambda & \frac{2}{3}s \\
      -2us & 2s & u - \lambda
    \end{vmatrix} &= 0 \\
    \begin{vmatrix}
      -\lambda & 1 & 0 \\
      -u^2 + gh - \frac{1}{3}s^2 & 2u - \lambda & \frac{2}{3}s \\
      -2us & 2s & u - \lambda
    \end{vmatrix} &= 0 \\
    -\lambda \p{\p{2u - \lambda}\p{u - \lambda} - \frac{4}{3}s^2} - \p{\p{-u^2 + gh - \frac{1}{3}s^2}\p{u - \lambda} + \frac{4}{3}s^2 u} &= 0 \\
    \p{u - \lambda}\lambda \p{\lambda - 2u} + \frac{4}{3}s^2\lambda - \frac{4}{3}s^2 u + \p{u - \lambda}\p{u^2 - gh + \frac{1}{3}s^2} &= 0 \\
    \p{u - \lambda}\p{\lambda \p{\lambda - 2u} - \frac{4}{3}s^2 + u^2 - gh + \frac{1}{3}s^2} &= 0 \\
    \p{u - \lambda}\p{\lambda^2 - 2u\lambda + u^2 - gh - s^2} &= 0
  \end{align*}
  Thus the eigenvalues of \(A\) are
  \begin{align*}
    \lambda &= u \\
    \lambda &= \frac{2u \pm \sqrt{4u^2 - 4\p{u^2 - gh - s^2}}}{2} \\
    \lambda &= u \pm \sqrt{gh + s^2}
  \end{align*}

  % TODO: Compute eigenvectors

  % TODO: Second Order
  % TODO: Third Order
\end{document}